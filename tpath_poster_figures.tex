% !TEX root = poster.tex

\usepackage{tikz,tikz-cd} % 
\usetikzlibrary{arrows,shapes,trees}
\usetikzlibrary{backgrounds}
\usetikzlibrary{calc,decorations.pathmorphing,patterns}
\usetikzlibrary{decorations.markings}

% new for poster
%\usetikzlibrary{decorations.pathreplacing,shapes,arrows}
\usetikzlibrary{positioning} % to use right= of
%\usetikzlibrary{calc} 


%%% start of NEW NEW
\tikzstyle{NooseArrowAtPosStyleTwo}=[decoration={ markings, mark=at position 0.2 with {\arrow[line width=1.5pt]{stealth};}}, postaction={decorate}]
\tikzstyle{NooseArrowAtPosStyleThree}=[decoration={ markings, mark=at position 0.3 with {\arrow[line width=1.5pt]{stealth};}}, postaction={decorate}]
\tikzstyle{NooseArrowAtPosStyleFour}=[decoration={ markings, mark=at position 0.4 with {\arrow[line width=1.5pt]{stealth};}}, postaction={decorate}]
\tikzstyle{NooseArrowAtPosStyleFive}=[decoration={ markings, mark=at position 0.5 with {\arrow[line width=1.5pt]{stealth};}}, postaction={decorate}]
\tikzstyle{NooseArrowAtPosStyleSix}=[decoration={ markings, mark=at position 0.6 with {\arrow[line width=1.5pt]{stealth};}}, postaction={decorate}]
\tikzstyle{NooseArrowAtPosStyleSeven}=[decoration={ markings, mark=at position 0.7 with {\arrow[line width=1.5pt]{stealth};}}, postaction={decorate}]

\newcommand\PathArrowLineWidth{1.8pt}
\newcommand\PathArrowStyle{to}
\tikzstyle{PathArrowAtPosStyleOne}=[decoration={ markings, mark=at position 0.1 with {\arrow[line width=\PathArrowLineWidth]{\PathArrowStyle};}}, postaction={decorate}]

\tikzstyle{PathArrowAtPosStyleTwo}=[decoration={ markings, mark=at position 0.2 with {\arrow[line width=\PathArrowLineWidth]{\PathArrowStyle};}}, postaction={decorate}]

\tikzstyle{PathArrowAtPosStyleThree}=[decoration={ markings, mark=at position 0.3 with {\arrow[line width=\PathArrowLineWidth]{\PathArrowStyle};}}, postaction={decorate}]

\tikzstyle{PathArrowAtPosStyleFour}=[decoration={ markings, mark=at position 0.4 with {\arrow[line width=\PathArrowLineWidth]{\PathArrowStyle};}}, postaction={decorate}]
\tikzstyle{PathArrowAtPosStyleFourSmall}=[decoration={ markings, mark=at position 0.4 with {\arrow[line width=0.7*\PathArrowLineWidth]{\PathArrowStyle};}}, postaction={decorate}]

\tikzstyle{PathArrowAtPosStyleFive}=[decoration={ markings, mark=at position 0.5 with {\arrow[line width=\PathArrowLineWidth]{\PathArrowStyle};}}, postaction={decorate}]
\tikzstyle{PathArrowAtPosStyleSix}=[decoration={ markings, mark=at position 0.6 with {\arrow[line width=\PathArrowLineWidth]{\PathArrowStyle};}}, postaction={decorate}]
\tikzstyle{PathArrowAtPosStyleSixSmall}=[decoration={ markings, mark=at position 0.6 with {\arrow[line width=0.7*\PathArrowLineWidth]{\PathArrowStyle};}}, postaction={decorate}]

\tikzstyle{PathArrowAtPosStyleSeven}=[decoration={ markings, mark=at position 0.7 with {\arrow[line width=\PathArrowLineWidth]{\PathArrowStyle};}}, postaction={decorate}]
\tikzstyle{PathArrowAtPosStyleSevenSmall}=[decoration={ markings, mark=at position 0.7 with {\arrow[line width=0.7*\PathArrowLineWidth]{\PathArrowStyle};}}, postaction={decorate}]

\tikzstyle{PathArrowAtPosStyleEight}=[decoration={ markings, mark=at position 0.8 with {\arrow[line width=\PathArrowLineWidth]{\PathArrowStyle};}}, postaction={decorate}]
\tikzstyle{PathArrowAtPosStyleEightSmall}=[decoration={ markings, mark=at position 0.8 with {\arrow[line width=0.7*\PathArrowLineWidth]{\PathArrowStyle};}}, postaction={decorate}]

\tikzstyle{PathArrowAtPosStyleNine}=[decoration={ markings, mark=at position 0.9 with {\arrow[line width=\PathArrowLineWidth]{\PathArrowStyle};}}, postaction={decorate}]
\tikzstyle{PathArrowAtPosStyleEnd}=[decoration={ markings, mark=at position 1 with {\arrow[line width=\PathArrowLineWidth]{\PathArrowStyle};}}, postaction={decorate}]

\tikzstyle{PathArrowAtPosStyleReversedFive}=[decoration={ markings, mark=at position 0.5 with {\arrow{to reversed};}}, postaction={decorate}]

\tikzstyle{NonOrientedTriangulationStyle}=[style=thick]

\tikzstyle{TPathStyle}=[blue, very thick, 
%line width=5pt, 
	style=dashed]

\tikzstyle{orientedTriangulationStyle}=[style=thick]

\tikzstyle{inducedmapTPathStyle}=[
	blue, very thick,%line width = 2pt, 
	->, style=dashed, rounded corners = 20pt,
	decoration={ markings, mark=at position .6 with {\arrow[blue, line width=2.5pt]{to};}}]
	 
\tikzstyle{inducedmapTPathonCnStyle}=[
	blue, very thick, %line width = 2pt, 
	style=dashed, rounded corners = 30pt,
	decoration={ markings, mark=at position .6 with {\arrow[blue, line width=2.5pt]{to};}}]



%%% start of new
\tikzstyle{RedGammaStyle}=[very thick, dashed, red]
\tikzstyle{CyanGammaStyle}=[very thick, dashed, cyan]
\tikzstyle{CyanBetaStyle}=[very thick, dashed, cyan]
\tikzstyle{CyanTilBetaStyle}=[line width=2pt, dashed, cyan]
\tikzstyle{RedTilSigmaStyle}=[line width=2pt, dashed, red]
\tikzstyle{RedSigmaStyle}=[very thick, dashed, red]
\newcommand\TikzPunctureSize{0.04}
\newcommand\TikzPointSize{0.03}
\newcommand\TikzIntersectionSize{0.03}

%%% ending triangle_0 is self-folded, including \a and \b
\newcommand \MacroLiftedTriangleOneTwoNoArrow {
\coordinate (P) at (0,0); \coordinate (v1) at (\Tikzn,0); \coordinate (v2) at (0,\Tikzn); \coordinate (y) at (\Tikzn,\Tikzn);

\draw (P) --(v1);% node[pos=0.5, below] {$r$};
\draw (P) -- (v2);% node[pos=0.5,left] {$\ccr$}; 

\draw (v2) -- (y);% node[pos=0.5,above]{$\a$};
\draw (v1) -- (y);% node[pos=0.5,right]{$\b$}; 

\draw (v2) -- (v1);% node[pos=0.7,right] {$\ell$}; 

\foreach \pos in {(v1),(v2)}
	{\draw [fill=olive] \pos circle (\TikzPointSize*\Tikzn);}

\foreach \pos in {(P),(y)}
	{\draw [fill=violet] \pos circle (\TikzPunctureSize*\Tikzn);}
}

\newcommand \MacroLiftedTriangleOneTwo {
\coordinate (P) at (0,0); \coordinate (v1) at (\Tikzn,0); \coordinate (v2) at (0,\Tikzn); \coordinate (y) at (\Tikzn,\Tikzn);

\draw (P) --(v1) node[pos=0.5, below] {$r$};
\draw (P) -- (v2) node[pos=0.5,left] {$\ccr$}; 

\draw (v2) -- (y) node[pos=0.5,above]{$\a$};
\draw (v1) -- (y) node[pos=0.5,right]{$\b$}; 

\draw[NooseArrowAtPosStyleSix] (v2) -- (v1) node[pos=0.7,right] {$\ell$}; 

\foreach \pos in {(v1),(v2),(y)}
	{\draw [fill=olive] \pos circle (\TikzPointSize*\Tikzn);}

\foreach \pos in {(P)}
	{\draw [fill=violet] \pos circle (\TikzPunctureSize*\Tikzn);}
}

\newcommand\TikzLiftedTriangleOneTwoCrossesA[1]{
\begin{tikzpicture}[scale=#1, DoubleTriangleStyle]
\newcommand\Tikzn{2}
\MacroLiftedTriangleOneTwo

\draw [RedGammaStyle, ->] (P) --(0.8*\Tikzn,1.2*\Tikzn) node[pos=0.4,left] {$\ga$};

\begin{scope}[gray]
\draw (\Tikzn * 0.4,\Tikzn * 0.22) node {$\tilblacktri{0}$};
\draw (\Tikzn * 0.75,\Tikzn *0.72) node {$\tilblacktri{1}$};
\end{scope}
\end{tikzpicture}
}

\newcommand\TikzLiftedTriangleOneTwo[1]{
\begin{tikzpicture}[scale=#1, DoubleTriangleStyle]
\newcommand\Tikzn{2}
\MacroLiftedTriangleOneTwo

\draw [RedGammaStyle, <-] (0.5*\Tikzn,0.8*\Tikzn) -- (0.1,0.1);

\begin{scope}[gray]
\draw (\Tikzn * 0.4,\Tikzn * 0.22) node {$\tilblacktri{0}$};
\draw (\Tikzn * 0.75,\Tikzn *0.72) node {$\tilblacktri{1}$};
\end{scope}
\end{tikzpicture}
}
\newcommand\TikzLiftedTriangleD[1]{
\begin{tikzpicture}[scale=#1, DoubleTriangleStyle]
\newcommand\Tikzn{2}
\MacroLiftedTriangleOneTwo

\draw [RedGammaStyle, ->] (0.5*\Tikzn,0.7*\Tikzn) -- (0.1,0.1);

\begin{scope}[gray]
\draw (\Tikzn * 0.4,\Tikzn * 0.22) node {$\tilblacktri{d}$};
\draw (\Tikzn * 0.75,\Tikzn *0.72) node {$\tilblacktri{d-1}$};
\end{scope}
\end{tikzpicture}
}
\newcommand\TikzCrossingNooseOne[1]{
\begin{tikzpicture}[scale={#1}]
\newcommand\Tikzn{2}
\MacroCrossingLRAB
%\coordinate (P) at (0,\Tikzn);
\draw[RedGammaStyle, <- ] (-0.2*\Tikzn,1.6*\Tikzn) -- (P) node[blue,left, pos=0.4] {$p_1$};
\newcommand\TikzHeight{1.3}
\draw[fill=blue] (-0.1*\Tikzn,\Tikzn*\TikzHeight) circle (0.2*\Tikzn ex); 
\end{tikzpicture}
}
\newcommand\TikzCrossingNooseOneAndA[1]{
\begin{tikzpicture}[scale={#1}]
\newcommand\Tikzn{2}
\MacroCrossingLRAB
%\coordinate (P) at (0,\Tikzn);
\draw[RedGammaStyle, -> ]% (-0.2*\Tikzn,1.6*\Tikzn) 
(P) -- (-0.6*\Tikzn,2.1*\Tikzn) node[pos=0.45,left]{$\ga$};
\newcommand\TikzHeight{1.3}
\end{tikzpicture}
}

\newcommand\TikzCrossingNooseD[1]{
\begin{tikzpicture}[scale={#1}]
\newcommand\Tikzn{2}
\MacroCrossingLRAB
%\coordinate (P) at (0,\Tikzn);
\draw[RedGammaStyle, -> ] (-0.2*\Tikzn,1.6*\Tikzn) -- (P) node[blue,left, pos=0.4] {$p_d$};
\newcommand\TikzHeight{1.3}
\draw[fill=blue] (-0.1*\Tikzn,\Tikzn*\TikzHeight) circle (0.2*\Tikzn ex); 
\end{tikzpicture}
}
%%% ending triangle_0 is self-folded, including \a and \b

%% begins crossing self-folded triangle, including \a and \b
\newcommand\MacroCrossingLRAB{
\begin{scope}[thick, gray]
	\coordinate (v) at (0,0); 
	\coordinate (P) at (0,0.9*\Tikzn);
	\coordinate (y) at (0,2*\Tikzn);
	\draw[thick] (0,0) -- (P);
	\draw (0.08*\Tikzn,0.4*\Tikzn) node {$r$};
	\draw (v) node[below,olive] {$v$}; \draw (y) node[below,olive] {$y$};
	
\begin{scope}[gray]
	\newcommand\TikzLowerPointerLeft{(-\Tikzn*0.5,\Tikzn*0.5)} 
	\newcommand	\TikzLowerPointerRight{(\Tikzn*0.5,\Tikzn*0.5)}
	\newcommand\TikzUpperPointerLeft{(-\Tikzn*0.5,\Tikzn)} \newcommand\TikzUpperPointerRight{(\Tikzn*0.5,\Tikzn)}

	\draw[style = thick, rounded corners = 20pt,NooseArrowAtPosStyleSix]	
	(v) .. controls \TikzLowerPointerLeft and \TikzUpperPointerLeft  .. (0,\Tikzn*1.5)
	.. controls \TikzUpperPointerRight and \TikzLowerPointerRight .. (v) node[pos=0.3,right=3pt] {$\ell$};
\end{scope}
	
	\newcommand\TikzLowerPointA{(-1.3*\Tikzn,0.3*\Tikzn)} \newcommand\TikzUpperPointA{(-\Tikzn,1.8*\Tikzn)}
	\newcommand\TikzLowerPointB{(1.3*\Tikzn,0.3*\Tikzn)} \newcommand\TikzUpperPointB{(\Tikzn,1.8*\Tikzn)}
	\draw (v) .. controls \TikzLowerPointA and \TikzUpperPointA .. (y) node[pos=0.6,left] {$\a$};
	\draw (v) .. controls \TikzLowerPointB and \TikzUpperPointB .. (y) node[pos=0.6,right] {$\b$};
\end{scope}
	
	\draw [fill=violet, violet] (P) circle (\TikzPunctureSize*\Tikzn); 
	\foreach \pos in {(v),(y)}
	{\draw [fill=olive, olive] \pos circle (\TikzPointSize*\Tikzn );}
}

\newcommand\TikzCrossingCounterclockwiseLRAB[1]{
\begin{tikzpicture}[scale={#1}]
\newcommand\Tikzn{2}
\MacroCrossingLRAB

\newcommand\TikzHeight{0.6}
\draw [RedGammaStyle, dashed, ->] (-\Tikzn *0.6, \Tikzn * \TikzHeight) -- (\Tikzn *0.7, \Tikzn * \TikzHeight) node[blue,below, pos=0.1] {$p_k$};
\draw[fill=blue] (-0.4*\Tikzn,\Tikzn*\TikzHeight) circle (0.2*\Tikzn ex);
\end{tikzpicture}
}

\newcommand\TikzCrossingClockwiseLRAB[1]{
\begin{tikzpicture}[scale={#1}]
\newcommand\Tikzn{2}
\MacroCrossingLRAB
\newcommand\TikzHeight{0.6}
\draw [RedGammaStyle, dashed, <-] (-\Tikzn *0.7, \Tikzn * \TikzHeight) -- (\Tikzn *0.7, \Tikzn * \TikzHeight) 
node[blue,below, pos=0.9] {$p_k$};
\draw[fill=blue] (0.4*\Tikzn,\Tikzn*\TikzHeight) circle (0.2*\Tikzn ex);
\end{tikzpicture}
}

%%% end of crossing self-folded triangle, including \a and \b

%%%
\newcommand\MacroDoubleCrossingActualThetaOneThetaH{
\draw[gray, thick, rounded corners=5pt] (v) -- (-2.3*\Tikzn,0) -- (-2.3*\Tikzn,2*\Tikzn) -- (y)
(-2*\Tikzn,0) -- (-1*\Tikzn,2*\Tikzn) node [black,pos=0.5, left] {$\theta_1$};
}
\newcommand\TikzDoubleCrossingActualCrossingLR[1]{
	\begin{tikzpicture}[scale={#1}]
	\newcommand\Tikzn{2}
	\MacroCrossingLRAB
	\MacroDoubleCrossingActualThetaOneThetaH
	
	\newcommand\TikzHeightBelow{0.6}
	\newcommand\TikzHeightExtraBelow{0.3}
	\newcommand\TikzHeightAbove{1.5}
	\draw [RedGammaStyle, dashed, ->, rounded corners=15pt] (-\Tikzn *2, \Tikzn * \TikzHeightBelow) -- (\Tikzn *0.2, \Tikzn * \TikzHeightBelow) -- (\Tikzn *0.2, \Tikzn * \TikzHeightAbove) -- (-2*\Tikzn, \Tikzn*\TikzHeightAbove)
	node[pos=0.8, above] {$\si$};
	
	\draw [CyanBetaStyle, dashed, ->] (-\Tikzn *2, \Tikzn * \TikzHeightExtraBelow) -- (\Tikzn *0.45, \Tikzn * \TikzHeightExtraBelow) 
	node[pos=0.2, below] {$\ga$};
	
%	\draw[fill=blue] (-0.4*\Tikzn,\Tikzn*\TikzHeightBelow) circle (0.2*\Tikzn ex);
	\end{tikzpicture}
}

\newcommand\TikzDoubleCrossingActualRhoOneRhoF[1]{
\begin{tikzpicture}[scale={#1}]
\newcommand\Tikzn{2}
\begin{scope}[thick, gray]
	\coordinate (v) at (0,0); 
	\coordinate (P) at (0,0.9*\Tikzn);
	\coordinate (y) at (0,2*\Tikzn);
	\draw[thick] (0,0) -- (y) ;
	\draw[black] (-0.1*\Tikzn,0.4*\Tikzn) node {$\rho_1$};
	\draw[black] (-0.2*\Tikzn,1.3*\Tikzn) node {$\rho_f$};
	\draw (v) node[below,white] {};%{$v$};% \draw (y) node[below,olive] {$y$};
	
	\newcommand\TikzLowerPointerLeft{(-\Tikzn*0.5,\Tikzn*0.5)} 
	\newcommand\TikzLowerPointerRight{(\Tikzn*0.5,\Tikzn*0.5)}
	\newcommand\TikzUpperPointerLeft{(-\Tikzn*0.5,\Tikzn)} \newcommand\TikzUpperPointerRight{(\Tikzn*0.5,\Tikzn)}
	
%	\draw[style = thick, rounded corners = 20pt,NooseArrowAtPosStyleSix]	
%	(v) .. controls \TikzLowerPointerLeft and \TikzUpperPointerLeft  .. (0,\Tikzn*1.5)
%	.. controls \TikzUpperPointerRight and \TikzLowerPointerRight .. (v) node[pos=0.3,right=3pt] {$\ell$};
	
	\newcommand\TikzLowerPointA{(-1.3*\Tikzn,0.3*\Tikzn)} \newcommand\TikzUpperPointA{(-\Tikzn,1.8*\Tikzn)}
	\newcommand\TikzLowerPointB{(1.3*\Tikzn,0.3*\Tikzn)} \newcommand\TikzUpperPointB{(\Tikzn,1.8*\Tikzn)}
	\draw (v) .. controls \TikzLowerPointA and \TikzUpperPointA .. (y) node[pos=0.5,black] {$\theta_h$};
	\draw (v) .. controls \TikzLowerPointB and \TikzUpperPointB .. (y);% node[pos=0.6,right] {$\b$};
\end{scope}
	
	\draw [fill=violet, violet] (P) circle (\TikzPunctureSize*\Tikzn); 
	\foreach \pos in {(v),(y)}
	{\draw [fill=olive, olive] \pos circle (\TikzPointSize*\Tikzn );}

\MacroDoubleCrossingActualThetaOneThetaH

\newcommand\TikzHeightBelow{0.6}
\newcommand\TikzHeightExtraBelow{0.3}
\newcommand\TikzHeightAbove{1.5}
\draw [RedGammaStyle, dashed, ->, rounded corners=15pt] (-\Tikzn *2, \Tikzn * \TikzHeightBelow) -- (\Tikzn *0.5, \Tikzn * \TikzHeightBelow) -- (\Tikzn *0.5, \Tikzn * \TikzHeightAbove) -- (-2*\Tikzn, \Tikzn*\TikzHeightAbove)
node[pos=0.8, above] {$\si$};
\draw [CyanBetaStyle, dashed, ->] (-\Tikzn *2, \Tikzn * \TikzHeightExtraBelow) -- (\Tikzn *0.3, \Tikzn * \TikzHeightExtraBelow) 
node[pos=0.2, below] {$\ga$};
\end{tikzpicture}
}

%%
\newcommand\MacroTwoFoldCoverCrossingRL{
\begin{scope}[thick, gray]
	\coordinate (v) at (0,0); 
	\coordinate (P) at (0,1*\Tikzn);
	\coordinate (y) at (0,2*\Tikzn);
	\draw[thick] (0,0) -- (y);
	\draw[black] (0.0*\Tikzn,0.6*\Tikzn) node {$r^1$};
	\draw[black] (-0.0*\Tikzn,1.6*\Tikzn) node {$r^2$};
	\draw (v) node[below,olive] {$v^1$}; \draw (y) node[above,olive] {$v^2$};
	
	\newcommand\TikzLowerPointerLeft{(-\Tikzn*0.6,\Tikzn*0.5)} 
	\newcommand\TikzLowerPointerRight{(\Tikzn*0.6,\Tikzn*0.5)}
	\newcommand\TikzUpperPointerLeft{(-\Tikzn*0.6,1.5*\Tikzn)} 
	\newcommand\TikzUpperPointerRight{(\Tikzn*0.6,1.5*\Tikzn)}
	
	\draw[style = thick, NooseArrowAtPosStyleThree,NooseArrowAtPosStyleSeven]	
	(v) .. controls \TikzLowerPointerLeft and \TikzUpperPointerLeft  .. (y) node[black,pos=0.7] {$\ell^1$}
	(y) .. controls \TikzUpperPointerRight and \TikzLowerPointerRight .. (v) node[black,pos=0.5] {$\ell^2$};
	
	\newcommand\TikzLowerPointA{(-1.3*\Tikzn,0.3*\Tikzn)} \newcommand\TikzUpperPointA{(-0.7*\Tikzn,1.8*\Tikzn)}
	\newcommand\TikzLowerPointB{(0.7*\Tikzn,0.2*\Tikzn)} \newcommand\TikzUpperPointB{(1.3*\Tikzn,1.7*\Tikzn)}
	\draw (v) .. controls \TikzLowerPointA and \TikzUpperPointA .. (-1.4,2*\Tikzn) node[black,pos=0.6] {$\a^1$};
	\draw (1.4,0*\Tikzn)  .. controls \TikzLowerPointB and \TikzUpperPointB .. (y) node[black,pos=0.6] {$\a^2$};
	\draw[fill=olive] (-1.4,2*\Tikzn) circle (\TikzPointActualSize) node [olive, above] {$y^2$};
	\draw (-0.8,2.1*\Tikzn) node {$\b$};
	\draw[fill=olive] (1.4,0*\Tikzn) circle (\TikzPointActualSize) node [olive, below] {$y^1$};
	\draw (0.8,-0.1*\Tikzn) node {$\b$};
	
\end{scope}
	
	\draw [fill=violet, violet] (P) circle (\TikzPunctureActualSize); 
	\foreach \pos in {(v),(y)}
	{\draw [fill=olive, olive] \pos circle (\TikzPointSize*\Tikzn );}
}


\newcommand\TikzTwoFoldCoverCrossingLR[1]{
\begin{tikzpicture}[scale={#1}]
\newcommand\Tikzn{2}
\MacroTwoFoldCoverCrossingRL
\newcommand\TikzWidth{2}
\draw[gray, thick,rounded corners=5pt] (v) -- (-\TikzWidth*\Tikzn,0) -- (-\TikzWidth*\Tikzn,2*\Tikzn) -- (y)
(-1.8*\Tikzn,0) -- (-1*\Tikzn,2*\Tikzn) node [black,pos=0.7, left] {$\theta_1^1$};
\draw[gray, thick,rounded corners=5pt] (v) -- (\TikzWidth*\Tikzn,0) -- (\TikzWidth*\Tikzn,2*\Tikzn) -- (y)
(1.8*\Tikzn,2*\Tikzn) -- (1*\Tikzn,0) node [black,pos=0.3, left] {$\theta_1^2$};

\newcommand\TikzHeightBelow{0.4}
\newcommand\TikzHeightAbove{0.8}
\draw [RedTilSigmaStyle, dashed, ->] (-\Tikzn *2, \Tikzn * \TikzHeightAbove) -- (\Tikzn *0.7, \Tikzn * \TikzHeightAbove) -- (2*\Tikzn, \Tikzn*\TikzHeightAbove) node[pos=0.7,above] {$\tilsi$};
\draw[fill=olive] (-\Tikzn *2, \Tikzn * \TikzHeightAbove) circle (\TikzPunctureActualSize) node[olive,left] {$\tils$};
\draw[fill=olive] (2*\Tikzn, \Tikzn*\TikzHeightAbove) circle (\TikzPunctureActualSize) node[olive,right] {$\tilt$};

\draw [CyanTilBetaStyle, dashed, ->] (-\Tikzn *1.5, \Tikzn * \TikzHeightBelow) -- (0.5*\Tikzn, \Tikzn*\TikzHeightBelow) node[pos=0.2, below] {$\tilbe$};

%node[blue,below, pos=0.1] {$p_k$};
%\draw[fill=blue] (-0.4*\Tikzn,\Tikzn*\TikzHeightBelow) circle (0.2*\Tikzn ex);
\end{tikzpicture}
}


\newcommand\MacroTwoFoldCoverCrossingRhoOneRhoF{
\begin{scope}[thick, gray]
	\coordinate (v) at (0,0); 
	\coordinate (P) at (0,1*\Tikzn);
	\coordinate (y) at (0,2*\Tikzn);
	\draw[thick] (0,0) -- (y);
	\draw[black] (0.0*\Tikzn,0.6*\Tikzn) node {$\rho_1^1$};
	\draw[black] (-0.0*\Tikzn,1.6*\Tikzn) node {$\rho_1^2$};
	
	\draw  (-0.7*\Tikzn,2*\Tikzn)  --  
	(0.7*\Tikzn,0) 
	node[black,pos=0.25] {$\rho_f^2$} node[black,pos=0.8] {$\rho_f^1$};
	\newcommand\TikzLowerPointerLeft{(-\Tikzn*0.6,\Tikzn*0.5)} 
	\newcommand\TikzLowerPointerRight{(\Tikzn*0.6,\Tikzn*0.5)}
	\newcommand\TikzUpperPointerLeft{(-\Tikzn*0.6,1.5*\Tikzn)} 
	\newcommand\TikzUpperPointerRight{(\Tikzn*0.6,1.5*\Tikzn)}
	
	\newcommand\TikzLowerPointA{(-0.8*\Tikzn,0.3*\Tikzn)} \newcommand\TikzUpperPointA{(-0.8*\Tikzn,1.8*\Tikzn)}
	\newcommand\TikzLowerPointB{(1*\Tikzn,0.3*\Tikzn)} \newcommand\TikzUpperPointB{(0.5*\Tikzn,1.8*\Tikzn)}
	\draw (v) .. controls \TikzLowerPointA and \TikzUpperPointA .. (-0.7*\Tikzn,2*\Tikzn) node[black,pos=0.6] {$\theta_{h}^1$};
	\draw (0.7*\Tikzn,0) .. controls \TikzLowerPointB and \TikzUpperPointB .. (y) node[black,pos=0.6] {$\theta_{h}^2$};
\end{scope}
	
	\draw [fill=violet, violet] (P) circle (\TikzPunctureActualSize); 
	\foreach \pos in {(v),(y),(0.7*\Tikzn,0), (-0.7*\Tikzn,2*\Tikzn)}
	{\draw [fill=olive, olive] \pos circle (\TikzPointSize*\Tikzn );}
}

%
%\newcommand\TikzTwoFoldCoverCrossingRhoOneRhoF[1]{
%\begin{tikzpicture}[scale={#1}]
%\newcommand\Tikzn{2}
%\MacroTwoFoldCoverCrossingRhoOneRhoF
%\newcommand\TikzWidth{2}
%\draw[gray, thick,rounded corners=5pt] (v) -- (-\TikzWidth*\Tikzn,0) -- (-\TikzWidth*\Tikzn,2*\Tikzn) -- (y)
%(-1.8*\Tikzn,0) -- (-1*\Tikzn,2*\Tikzn) node [black,pos=0.7, left] {$\theta_1^1$};
%\draw[gray,thick, rounded corners=5pt] (v) -- (\TikzWidth*\Tikzn,0) -- (\TikzWidth*\Tikzn,2*\Tikzn) -- (y)
%(1.8*\Tikzn,2*\Tikzn) -- (1*\Tikzn,0) node [black,pos=0.25, left] {$\theta_1^2$};
%
%\newcommand\TikzHeightBelow{0.4}
%\newcommand\TikzHeightAbove{0.8}
%\draw [RedTilSigmaStyle, dashed, ->] (-\Tikzn *2, \Tikzn * \TikzHeightAbove) -- (\Tikzn *0.7, \Tikzn * \TikzHeightAbove) -- (2*\Tikzn, \Tikzn*\TikzHeightAbove) node[pos=0.7,above] {$\tilsi$};
%\draw[fill=olive] (-\Tikzn *2, \Tikzn * \TikzHeightAbove) circle (\TikzPunctureActualSize) node[olive,left] {$\tils$};
%\draw[fill=olive] (2*\Tikzn, \Tikzn*\TikzHeightAbove) circle (\TikzPunctureActualSize) node[olive,right] {$\tilt$};
%\draw [CyanTilBetaStyle, dashed, ->] (-\Tikzn *1.5, \Tikzn * \TikzHeightBelow) -- (0.3*\Tikzn, \Tikzn*\TikzHeightBelow) node[pos=0.2, below] {$\tilbe$};
%
%\draw (-1.4,2*\Tikzn) node [white, above] {$y^2$};
%\draw (1.4,0*\Tikzn) node [white, below] {$y^1$};
%\end{tikzpicture}
%}


%%%% current 
\tikzstyle{LiftedPathStyle} = [blue, line width=2pt, dashed]

% begin of paths for k=1 or k=d
%% begins crossing self-folded triangle, including \a and \b


\newcommand\TikzNooseOneRLMinusA[1]{
\begin{tikzpicture}[scale=#1, DoubleTriangleLightStyle]
\newcommand\Tikzn{2}
\MacroLiftedTriangleOneTwoNoArrow
\draw [LiftedPathStyle, PathArrowAtPosStyleTwo, PathArrowAtPosStyleFour, PathArrowAtPosStyleEnd, rounded corners=5pt] 
(P) --(v1) node[pos=0.4, below] {$r$}
-- (0.1,0.9*\Tikzn) node[pos=0.6,right] {$\ell$}
(0.1,0.9*\Tikzn)
-- (\Tikzn,0.9*\Tikzn) node[pos=0.3,above=3pt]{$\a$}
-- (\Tikzn,1.1*\Tikzn)
-- (0.6*\Tikzn,1.1*\Tikzn);
\end{tikzpicture}
}

\newcommand\TikzNooseOneRLPlusB[1]{
\begin{tikzpicture}[scale=#1, DoubleTriangleLightStyle]
\newcommand\Tikzn{2}
\MacroLiftedTriangleOneTwoNoArrow
\draw [LiftedPathStyle, PathArrowAtPosStyleTwo, PathArrowAtPosStyleFive, PathArrowAtPosStyleEnd, rounded corners=5pt] 
 (P) -- (v2) node[pos=0.3,left] {$\ccr$}
-- (v1) node[pos=0.6,right] {$\ell$}
-- (y) node[pos=0.5,right]{$\b$}
-- (0.3,\Tikzn) node[pos=0.5,above]{$\a$};
\end{tikzpicture}
}

\newcommand\TikzNooseOneRLPlusLMinus[1]{
\begin{tikzpicture}[scale=#1, DoubleTriangleLightStyle]
\newcommand\Tikzn{2}
\MacroLiftedTriangleOneTwoNoArrow
\draw [LiftedPathStyle, PathArrowAtPosStyleOne, PathArrowAtPosStyleSeven, PathArrowAtPosStyleEnd, rounded corners=5pt] 
(P) -- (0,0.8*\Tikzn) node[pos=0.5,left] {$\ccr$} 
-- (v1)  -- (0.2,\Tikzn)  node[pos=0.6,right] {$\ell$}
 -- (y) node[pos=0.5,above]{$\a$};
\end{tikzpicture}
}

\newcommand\MacroCrossingLRABPlain{
\begin{scope}[thick, gray]
	\coordinate (v) at (0,0); 
	\coordinate (P) at (0,0.9*\Tikzn);
	\coordinate (y) at (0,2*\Tikzn);
	\draw[thick] (0,0) -- (P);
	\draw (0.08*\Tikzn,0.4*\Tikzn) node {$r$};
	\draw (v) node[below,olive] {$v$}; \draw (y) node[below,olive] {$y$};
	
	\newcommand\TikzLowerPointerLeft{(-\Tikzn*0.5,\Tikzn*0.5)} 
	\newcommand	\TikzLowerPointerRight{(\Tikzn*0.5,\Tikzn*0.5)}
	\newcommand\TikzUpperPointerLeft{(-\Tikzn*0.5,\Tikzn)} \newcommand\TikzUpperPointerRight{(\Tikzn*0.5,\Tikzn)}

	\draw[style = thick, rounded corners = 20pt]	
	(v) .. controls \TikzLowerPointerLeft and \TikzUpperPointerLeft  .. (0,\Tikzn*1.5)
	.. controls \TikzUpperPointerRight and \TikzLowerPointerRight .. (v);% node[pos=0.3,right=3pt] {$\ell$};
	
	\newcommand\TikzLowerPointA{(-1.3*\Tikzn,0.3*\Tikzn)} \newcommand\TikzUpperPointA{(-\Tikzn,1.8*\Tikzn)}
	\newcommand\TikzLowerPointB{(1.3*\Tikzn,0.3*\Tikzn)} \newcommand\TikzUpperPointB{(\Tikzn,1.8*\Tikzn)}
	\draw (v) .. controls \TikzLowerPointA and \TikzUpperPointA .. (y) ;%node[pos=0.6,left] {$\a$};
	\draw (v) .. controls \TikzLowerPointB and \TikzUpperPointB .. (y) ;%node[pos=0.6,right] {$\b$};
\end{scope}
	
	\draw [fill=violet, violet] (P) circle (\TikzPunctureSize*\Tikzn); 
	\foreach \pos in {(v),(y)}
	{\draw [fill=olive, olive] \pos circle (\TikzPointSize*\Tikzn );}
}

\newcommand\TikzRLMinusA[1]{
\begin{tikzpicture}[scale={#1}] 	
\newcommand\Tikzn{2}
	\MacroCrossingLRABPlain

	\newcommand\TikzLowerPointerLeft{(-\Tikzn*0.5,\Tikzn*0.5)} 
	\newcommand	\TikzLowerPointerRight{(\Tikzn*0.5,\Tikzn*0.5)}
	\newcommand\TikzUpperPointerLeft{(-\Tikzn*0.5,\Tikzn)} \newcommand\TikzUpperPointerRight{(\Tikzn*0.5,\Tikzn)}
	\newcommand\TikzLowerPointA{(-1.3*\Tikzn,0.3*\Tikzn)} \newcommand\TikzUpperPointA{(-\Tikzn,1.8*\Tikzn)}
	%\newcommand\TikzLowerPointB{(1.3*\Tikzn,0.3*\Tikzn)} \newcommand\TikzUpperPointB{(\Tikzn,1.8*\Tikzn)}
	
\draw [LiftedPathStyle, PathArrowAtPosStyleOne,PathArrowAtPosStyleFour,rounded corners=15pt] 
	(P) -- (0.1,0.1)
	(0.1,0.1) .. controls \TikzLowerPointerRight and \TikzUpperPointerRight .. (0,\Tikzn*1.5)
	node[pos=0.7,right=3pt] {$\ell_-$}
	.. controls \TikzUpperPointerLeft and \TikzLowerPointerLeft .. (-0.3,0.2);
	
\draw [LiftedPathStyle, PathArrowAtPosStyleNine,rounded corners=15pt] 
	 (-0.3,0.2) .. controls \TikzLowerPointA and \TikzUpperPointA .. (y) node[pos=0.4,left] {$\a$};
\draw [LiftedPathStyle, PathArrowAtPosStyleEnd,rounded corners=15pt] 
	(y) --  (-0.4*\Tikzn,2.2*\Tikzn) -- (-1*\Tikzn,1.5*\Tikzn); 
	 
\draw[LiftedPathStyle] (0.08*\Tikzn,0.4*\Tikzn) node {$r$};	

\end{tikzpicture}
}

\newcommand\TikzRLPlusB[1]{
\begin{tikzpicture}[scale={#1}] 	
\newcommand\Tikzn{2}
	\MacroCrossingLRABPlain

	\newcommand\TikzLowerPointerLeft{(-\Tikzn*0.5,\Tikzn*0.5)} 
	\newcommand	\TikzLowerPointerRight{(\Tikzn*0.5,\Tikzn*0.5)}
	\newcommand\TikzUpperPointerLeft{(-\Tikzn*0.5,\Tikzn)} \newcommand\TikzUpperPointerRight{(\Tikzn*0.5,\Tikzn)}
	\newcommand\TikzLowerPointA{(-1.3*\Tikzn,0.3*\Tikzn)} \newcommand\TikzUpperPointA{(-\Tikzn,1.8*\Tikzn)}
	\newcommand\TikzLowerPointB{(1.3*\Tikzn,0.3*\Tikzn)} \newcommand\TikzUpperPointB{(\Tikzn,1.8*\Tikzn)}
	
\draw [LiftedPathStyle, PathArrowAtPosStyleOne, PathArrowAtPosStyleSeven, rounded corners=15pt] 
	(P) -- (0,0.3)
	(0,0.3) .. controls \TikzLowerPointerLeft and \TikzUpperPointerLeft  .. (0,\Tikzn*1.5)
	.. controls \TikzUpperPointerRight and \TikzLowerPointerRight .. (0.2,0) node[pos=0.3,right=3pt] {$\ell_+$};

\draw [LiftedPathStyle, PathArrowAtPosStyleNine, rounded corners=15pt] 
	 (0.2,0) .. controls \TikzLowerPointB and \TikzUpperPointB .. (y) node[pos=0.6,right] {$\b$};
	 
\draw [LiftedPathStyle, PathArrowAtPosStyleNine] 
	 (y) .. controls \TikzUpperPointA and \TikzLowerPointA .. (-0.8,0.3) node[pos=0.6,left] {$\a$};	 

\draw[LiftedPathStyle] (0.08*\Tikzn,0.4*\Tikzn) node {$r$};
\end{tikzpicture}
}

\newcommand\TikzRLPlusLMinusA[1]{
\begin{tikzpicture}[scale={#1}] 	
\newcommand\Tikzn{2}
	\MacroCrossingLRABPlain

	\newcommand\TikzLowerPointerLeft{(-\Tikzn*0.5,\Tikzn*0.5)} 
	\newcommand	\TikzLowerPointerRight{(\Tikzn*0.5,\Tikzn*0.5)}
	\newcommand\TikzUpperPointerLeft{(-\Tikzn*0.5,\Tikzn)} \newcommand\TikzUpperPointerRight{(\Tikzn*0.5,\Tikzn)}
	\newcommand\TikzLowerPointA{(-1.2*\Tikzn,0.3*\Tikzn)} \newcommand\TikzUpperPointA{(-\Tikzn,1.8*\Tikzn)}
	%\newcommand\TikzLowerPointB{(1.3*\Tikzn,0.3*\Tikzn)} \newcommand\TikzUpperPointB{(\Tikzn,1.8*\Tikzn)}
	
\draw [LiftedPathStyle, PathArrowAtPosStyleOne, PathArrowAtPosStyleFour, rounded corners=15pt] 
	(P) -- (0,0.3)
	(0,0.3) .. controls \TikzLowerPointerLeft and \TikzUpperPointerLeft  .. (0,\Tikzn*1.4)
	.. controls \TikzUpperPointerRight and \TikzLowerPointerRight .. (0.3,0.2); %node[pos=0.3,right=3pt] {$\ell_+$};


	\renewcommand\TikzLowerPointerLeft{(-\Tikzn*0.7,\Tikzn*0.5)} 
	\renewcommand\TikzLowerPointerRight{(\Tikzn*0.7,\Tikzn*0.5)}
	\renewcommand\TikzUpperPointerLeft{(-\Tikzn*0.8,\Tikzn)} 
	\renewcommand\TikzUpperPointerRight{(\Tikzn*0.8,\Tikzn)}
	
\draw [LiftedPathStyle,PathArrowAtPosStyleFour,rounded corners=15pt] 
	(0.3,0.2) .. controls \TikzLowerPointerRight and \TikzUpperPointerRight .. (0,\Tikzn*1.6)
	node[pos=0.7,right=3pt] {$\ell$}
	.. controls \TikzUpperPointerLeft and \TikzLowerPointerLeft .. (-0.5,0.3);

\draw [LiftedPathStyle, PathArrowAtPosStyleNine,rounded corners=15pt] 
	 (-0.5,0.3) .. controls \TikzLowerPointA and \TikzUpperPointA .. (y) node[pos=0.6,left] {$\a$};

\draw[LiftedPathStyle] (0.08*\Tikzn,0.4*\Tikzn) node {$r$};
\end{tikzpicture}
}

%%%%% begin of double lifted triangles %%%%%%%%
\tikzstyle{DoubleTriangleStyle}=[thick]
\tikzstyle{DoubleTriangleLightStyle}=[gray!90]

\newcommand\MacroDoubleTriangle{
\newcommand\Tikzn{2}
\coordinate (P) at (0,\Tikzn);

\draw (0,0) -- (P) node[pos=0.3, left] {$r$};

\draw (-\Tikzn,0) -- (-\Tikzn,\Tikzn) node[pos=0.6,left]{$\b$}; %left 
\draw (0,0) -- (-\Tikzn,0) node[pos=0.5,below]{$\a$}; %left 
\draw (\Tikzn,0) -- (\Tikzn,\Tikzn) node[pos=0.6,right]{$\a$}; %right 
\draw (0,0) -- (2,0) node[pos=0.5,below]{$\b$}; %right 

\draw [fill=violet] (P) circle (\TikzPunctureSize*\Tikzn);
\foreach \pos in {(0,0),(-\Tikzn,\Tikzn) ,(\Tikzn,\Tikzn), (-\Tikzn,0) ,(\Tikzn,0)}
	{\draw [fill=olive] \pos circle (\TikzPointSize*\Tikzn);}
}

\newcommand\MacroDoubleTriangleWithArrow{
\MacroDoubleTriangle
\draw[NooseArrowAtPosStyleSeven] (0,0) -- (-\Tikzn,\Tikzn) node[pos=0.3,left] {$\ell$}; % left
\draw[NooseArrowAtPosStyleFour] (\Tikzn,\Tikzn) -- (0,0) node[pos=0.7,right] {$\ell$}; % right
}

\newcommand\MacroDoubleTriangleNoArrow{
\MacroDoubleTriangle
\draw (0,0) -- (-\Tikzn,\Tikzn) node[pos=0.3,left] {$\ell$}; % left
\draw (\Tikzn,\Tikzn) -- (0,0) node[pos=0.7,right] {$\ell$}; % right
}

\newcommand\TikzDoubleTriangleCounterclockwise[1]{
\begin{tikzpicture}[scale=#1, DoubleTriangleStyle]
\MacroDoubleTriangleWithArrow
\draw (-\Tikzn,\Tikzn) -- (P) node[pos=0.5,above]{$\dot{r}$};
\draw (\Tikzn,\Tikzn) -- (P) node[pos=0.5,above]{$\ddot{r}$};
\draw [RedGammaStyle, ->] (-\Tikzn *0.65, \Tikzn *0.5) -- (\Tikzn *0.8, \Tikzn *0.5);

\begin{scope}[gray]
\draw (-\Tikzn * 0.35,\Tikzn * 0.72) node {$\tilblacktri{k}$};
\draw (\Tikzn * 0.35,\Tikzn *0.72) node {$\tilblacktri{k+1}$};
\end{scope}

\end{tikzpicture}
}

\newcommand\TikzDoubleTriangleClockwise[1]{
\begin{tikzpicture}[scale=#1, DoubleTriangleStyle]
\MacroDoubleTriangleWithArrow

\draw (-\Tikzn,\Tikzn) -- (P) node[pos=0.5,above]{$\ddot{r}$};
\draw (\Tikzn,\Tikzn) -- (P) node[pos=0.5,above]{$\dot{r}$};

\draw [RedGammaStyle, <-] (-\Tikzn *0.8, \Tikzn *0.5) -- (\Tikzn *0.7, \Tikzn *0.5);

\begin{scope}[gray]
\draw (-\Tikzn * 0.35,\Tikzn * 0.72) node {$\tilblacktri{k+1}$};
\draw (\Tikzn * 0.35,\Tikzn *0.72) node {$\tilblacktri{k}$};
\end{scope}

\end{tikzpicture}
}



\newcommand\TikzDoubleTriangleKPlusPlus[1]{
\begin{tikzpicture}[scale=#1, DoubleTriangleLightStyle]
\MacroDoubleTriangleNoArrow
\draw (-\Tikzn,\Tikzn) -- (P) node[pos=0.5,above]{$\ddot{r}$};
\draw (\Tikzn,\Tikzn) -- (P) node[pos=0.5,above]{$\dot{r}$};

\coordinate (pk) at (-0.6*\Tikzn,0.6*\Tikzn);
\coordinate (pk1) at (0.2,0.3*\Tikzn);
\draw [LiftedPathStyle, PathArrowAtPosStyleTwo, PathArrowAtPosStyleEnd, rounded corners=5pt] (pk) -- (-0.1,0) -- ($(P)+(-0.1,0)$) -- (pk1);

%\foreach \pos in {(pk),(pk1)}
%	{\draw [black,fill=blue] \pos circle (\TikzIntersectionSize*\Tikzn);}
\end{tikzpicture}
}
\newcommand\TikzEllMinusRR[1]{ %
\begin{tikzpicture}[scale={#1}] 	
	\MacroChubbySelfFoldedTriangle
	\MacroEllMinusRRLong
\end{tikzpicture}
}

\newcommand\TikzDoubleTriangleKKPlus[1]{
\begin{tikzpicture}[scale=#1, DoubleTriangleLightStyle]
\MacroDoubleTriangleNoArrow
\draw (-\Tikzn,\Tikzn) -- (P) node[pos=0.5,above]{$\ddot{r}$};
\draw (\Tikzn,\Tikzn) -- (P) node[pos=0.5,above]{$\dot{r}$};

\coordinate (pk) at (-0.2+-0.5*\Tikzn,0.4*\Tikzn);
\coordinate (pk1) at (0,0.9*\Tikzn);
\draw [LiftedPathStyle, PathArrowAtPosStyleSix, PathArrowAtPosStyleEnd, rounded corners=5pt] (pk) -- (-\Tikzn+0.2,\Tikzn) -- (0,0) -- (pk1);

%\foreach \pos in {(pk),(pk1)}
%	{\draw [black,fill=blue] \pos circle (\TikzIntersectionSize*\Tikzn);}
\end{tikzpicture}
}

\newcommand\TikzEllPlusEllMinusR[1]{ %
\begin{tikzpicture}[scale={#1}] 	
	\MacroChubbySelfFoldedTriangle
	\MacroEllPlusEllMinusR
\end{tikzpicture}
}

\newcommand\TikzDoubleTriangleKThirdEdgePlus[1]{
\begin{tikzpicture}[scale=#1, DoubleTriangleLightStyle]
\MacroDoubleTriangleNoArrow
\draw (-\Tikzn,\Tikzn) -- (P) node[pos=0.5,above]{$\ddot{r}$};
\draw (\Tikzn,\Tikzn) -- (P) node[pos=0.5,above]{$\dot{r}$};

\coordinate (pk) at (-0.3*\Tikzn,0.3*\Tikzn);
\coordinate (pk1) at (0,0.1*\Tikzn);
\draw [LiftedPathStyle, PathArrowAtPosStyleTwo, PathArrowAtPosStyleSix, PathArrowAtPosStyleEnd, rounded corners=5pt] (pk) -- (-\Tikzn,\Tikzn) -- (P) -- (pk1);

%\foreach \pos in {(pk),(pk1)}
%	{\draw [black,fill=blue] \pos circle (\TikzIntersectionSize*\Tikzn);}
\end{tikzpicture}
}

%%%%%%% from om_2k+2 to om_2k+4

\newcommand\TikzDoubleTriangleRRL[1]{
\begin{tikzpicture}[scale=#1, DoubleTriangleLightStyle]
\MacroDoubleTriangleNoArrow
\draw (-\Tikzn,\Tikzn) -- (P) node[pos=0.5,above]{$\ddot{r}$};
\draw (\Tikzn,\Tikzn) -- (P) node[pos=0.5,above]{$\dot{r}$};

\coordinate (pk2) at (0.6*\Tikzn,0.6*\Tikzn);
\draw [LiftedPathStyle, PathArrowAtPosStyleSix, PathArrowAtPosStyleEnd, rounded corners=5pt]  (-0.2,0.3) -- ($(P)+(-0.1,-0.2)$) -- (0.2,0.1) -- (pk2);
\end{tikzpicture}
}

\newcommand\TikzDoubleTriangleRLL[1]{
\begin{tikzpicture}[scale=#1, DoubleTriangleLightStyle]
\MacroDoubleTriangleNoArrow
\draw (-\Tikzn,\Tikzn) -- (P) node[pos=0.5,above]{$\ddot{r}$};
\draw (\Tikzn,\Tikzn) -- (P) node[pos=0.5,above]{$\dot{r}$};

\draw [LiftedPathStyle, PathArrowAtPosStyleTwo, PathArrowAtPosStyleEnd, rounded corners=15pt] 
(0,0.9*\Tikzn) --  (0,0.2) -- (\Tikzn-0.2,\Tikzn) -- (0.7,0.5) ;
\end{tikzpicture}
}

\newcommand\TikzRREllMinus[1]{ %
\begin{tikzpicture}[scale={#1}] 	
	\MacroChubbySelfFoldedTriangle
	\MacroRREllMinus
\end{tikzpicture}
}


\newcommand\TikzDoubleTriangleRddotRL[1]{
\begin{tikzpicture}[scale=#1, DoubleTriangleLightStyle]
\MacroDoubleTriangleNoArrow
\draw (-\Tikzn,\Tikzn) -- (P) node[pos=0.5,above]{$\ddot{r}$};
\draw (\Tikzn,\Tikzn) -- (P) node[pos=0.5,above]{$\dot{r}$};

\draw [LiftedPathStyle, PathArrowAtPosStyleTwo, PathArrowAtPosStyleEnd, rounded corners=5pt] (0,0) -- (0,\Tikzn)  -- (\Tikzn,\Tikzn) -- (0.3,0.3) ;

\end{tikzpicture}
}

\newcommand\TikzRREllPlusnonBT[1]{ %
\begin{tikzpicture}[scale={#1}] 	
	\MacroChubbySelfFoldedTriangle
	\MacroRREllPlusnonBT
\end{tikzpicture}
}
\newcommand\TikzRLMinLPlus[1]{ %
\begin{tikzpicture}[scale={#1}] 	
	\MacroChubbySelfFoldedTriangle
	\MacroRLMinLPlus
\end{tikzpicture}
}
%%%%% end of double lifted triangles %%%%%%%%

\newcommand\WheelLocalTriangulation[1]{ % USED
\begin{tikzpicture}[scale=#1]
	\path (0:0) coordinate (origin);
	\path (1*50:2) coordinate (1);
	\path (2*50:2) coordinate (2);
	\path (3*50:2) coordinate (3);
	\path (4*50:2) coordinate (4);
	\path (5*50:2) coordinate (5);
	\path (6*50:2) coordinate (6);
	\path (7*50:2) coordinate (7);
	
	\draw [fill=gray!20, style=dashed] (origin) circle (2);
	
	\begin{scope}[NonOrientedTriangulationStyle]
	\draw  [fill=white] (1) -- (2) -- (3) -- (4) -- (5) -- (6) -- (7) -- cycle;
	\draw [postaction={decorate}] (1) -- (origin);
	\draw [postaction={decorate}] (2) -- (origin);
	\draw [postaction={decorate}] (3) -- (origin);
	\draw [postaction={decorate}] (4) -- (origin);
	\draw [postaction={decorate}] (5) -- (origin);
	\draw [postaction={decorate}] (6) -- (origin);
	\draw [postaction={decorate}] (7) -- (origin);
	\draw [fill=violet] (origin) circle (1.2*\TikzPunctureActualSize);
	\draw (origin) node[above=4pt, right=2pt, violet] {\tiny{$P$}};
	\node at (0:3) {}; \node at (180:3) {};
	\end{scope}
\end{tikzpicture}
  }
  
\newcommand \SelfFoldedLocalTriangulation[1] { % USED
	\begin{tikzpicture}[scale={#1}]
	\node at (-4,1) {};  \node at (4,1) {};  
	\coordinate (v1L) at (-2.5,1);  
	\coordinate (v1R) at (2.5,1); 
	\coordinate (v2L) at (-2.5,3.5); 
	\coordinate (v2R) at (2.5,3);

	\begin{scope}[NonOrientedTriangulationStyle]
		\draw [style = dashed, fill=gray!20]  % left gray area
		(0,0) .. controls (-3,2) and (-2,3.5) .. (0,4)  -- (v2L) -- (v1L) -- (0,0);
		
		\draw [style = dashed, fill=gray!20]  % right gray area
		(0,0) .. controls (3,2) and (2,3.5) .. (0,4)  -- (v2R) -- (v1R) -- (0,0);
	
		\draw[style = thick, postaction={decorate}, fill=white] (0,0) -- (0,2.2);
		\draw (0.2,1) node {$r$};
		\draw[style = thick, postaction={decorate}] (0,0) .. controls (-3,4) and (3,4) .. (0,0) 
		node[pos = 0.7, above]{$\ell$};
		\draw[style = thick, postaction={decorate}] (0,0) .. controls (-3,2) and (-2,3.5) .. (0,4); 
		%node[pos = 0.5,right=0.02pt]{$\a$};
		\draw (-2.1,2.7) node {$\a$};
		\draw[style = thick, postaction={decorate}] (0,0) .. controls (3,2) and (2,3.5) .. (0,4); 
		%node[pos = 0.4, right]{$\b$};
		\draw (2.05,2) node {\tiny $\b$};
	\end{scope}

	\coordinate (Ps) at (0,2.2);  \coordinate (xs) at (0,0); \coordinate (ys) at (0,4);
	\foreach \pos in {(0,0),(0,4)}
		{\draw[fill=olive] \pos circle (1.2*\TikzPointActualSize);}
	\draw [fill=violet] (Ps) circle (1.2*\TikzPunctureActualSize);  \draw[violet] (0,2.2) node[above] {\tiny{$P$}}; %Puncture
		
	\end{tikzpicture}
}  


 \newcommand\MacroTtwoTpathChubby{
	 \begin{scope}[thick, gray!60]
	 \newcommand\localfontsize{\small}
	 \draw[rounded corners = 20pt] (-4,4)  -- (0,7) node[pos=0.8,above,black]{\localfontsize $1$} -- (4,4);
	 
	 \draw[rounded corners = 15pt] (0,0) -- 
	 (-2.5,4) %-- (-3,4) 
	 --  (0,6) node[pos=0.8,right,black]{\localfontsize $2$}  -- (4,4);
	 
	 \draw[thick] (0,0) -- (0,2.9) node[pos=0.8, right=2pt,black]{\localfontsize $r$};
	
	\draw[style = thick, rounded corners = 20pt]
	(0,0) .. controls (-1.5,2) and (-2,3) .. (0,5) .. 
	controls (2,3)  and (2,2) .. (0,0) node[pos=0.4, above=2pt,black] {\localfontsize $\ell$};
	
	\draw
	(0,0) -- (4,4) -- (0,8) -- node[pos=0.5, left] {} (-4,4) -- (0,0);	
	\end{scope}
	\draw [fill=violet] (0,2.9) circle (.3ex);
 }

\newcommand\MacroCounterClockRadius[4]{
\draw[TPathStyle, %blue,style = dashed, thick, %line width = 1pt, 
rounded corners = 5pt,
	decoration={ markings, mark=at position .3 with {\arrow[line width=1pt]{to};},
	mark=at position .8 with {\arrow[line width=1pt]{to};}}, postaction={decorate}]
	({#1},{#2}) .. controls (0.2,3) and (0.4,3) .. (0,3.5) %node[pos = 0.4, right]{$w_{j+1}$} 
	 .. controls (-0.4,3) and (-0.2,3) .. ({#3},{#4});% node[pos = 0.6, left]{$w_{j}$};
}

\newcommand\MacroClockLoop[4]{
	\draw[TPathStyle, %line width = 1.5pt, 
	rounded corners = 20pt
	,decoration={ markings, mark=at position .73 with {\arrow[blue, line width=1pt]{to};}},
	postaction={decorate}]
	({#1},{#2}) .. controls (-1.5,2) and (-2,3) .. (0,5) .. 
	controls (2,3)  and (2,2) .. ({#3},{#4}); %node[pos=0.5, right] {$\ell$};
}

\newcommand\MacroClockRadius[4]{
\draw[TPathStyle, %blue,style = dashed, thick, %line width = 1pt, 
rounded corners = 5pt,
	decoration={ markings, mark=at position .3 with {\arrow[line width=1pt]{to};},
	mark=at position .8 with {\arrow[line width=1pt]{to};}}, postaction={decorate}]
	({#1},{#2}) .. controls  (-0.2,3) and (-0.4,3) .. (0,3.5)  
	.. controls (0.4,3) and (0.2,3)  .. ({#3},{#4});
}

\newcommand\MacroCounterClockLoop[4]{
	\draw[TPathStyle, %blue, thick, %line width = 1.5pt, 
	rounded corners = 20pt
	,decoration={ markings, mark=at position .73 with {\arrow[blue, line width=1pt]{to};}},
	postaction={decorate}]
	({#1},{#2}) .. controls (2,2) and (2,3) .. (0,5) ..
	controls  (-2,3) and (-1.5,2)  ..   ({#3},{#4}); %node[pos=0.5, right] {$\ell$};
}

\newcommand\TikzPosterTpathSixNineFontSize{\small}
\newcommand\TikzPosterTpathSixNineCaptionSize{\small}
\newcommand\TikzPosterTpathSixNineCaptionLocation{(-3,-2.6)}

\newcommand\TikzTpathSix[2]{
	\begin{tikzpicture} [scale = {#1}]
	\begin{scope}[rotate=45]
	\MacroTtwoTpathChubby
	\begin{scope}[TPathStyle]
	\draw[rounded corners = 20pt]  (0,8) -- (3,5) -- (0,7) -- (-4,4);
	\draw (-4,4) -- (0.2*0,-0.1*0);
	
	\MacroCounterClockRadius{0.2*0}{-0.1*0}{-0.5}{1}	
	\MacroClockLoop{-0.5}{1}{1}{1}

	\draw[->] (1,1) -- (3.8,3.8);
	\end{scope}
	\end{scope}
	%\draw (-3,7) node {($6$)};
	\node at (-4,6.3) {\TikzPosterTpathSixNineFontSize $b_4$};
	\node at (-3,-0.5) {\TikzPosterTpathSixNineFontSize $b_2$};
	\node at (0.5,3) {\TikzPosterTpathSixNineFontSize $b_3$};
	\node at \TikzPosterTpathSixNineCaptionLocation {\TikzPosterTpathSixNineCaptionSize #2};
	\end{tikzpicture}
}

\newcommand\TikzTpathSixBad[1]{
	\begin{tikzpicture} [scale = {#1}]
	\begin{scope}[rotate=45]
	\MacroTtwoTpathChubby
	\begin{scope}[TPathStyle]
	\draw[rounded corners = 20pt]  (0,8) -- (3,5) -- (0,7) -- (-4,4) ;
	\draw (-4,4) -- (-0.2,0);
	
	\MacroClockRadius{-0.2}{0}{0.5}{1.5}	
	\MacroCounterClockLoop{0.5}{1.5}{0.3}{0.3}%{0.5}{0}

	%\draw[->] (0.5,0) -- (3.8,3.8);
	\draw[->] (0.3,0.3) -- (3.8,3.8);
	\end{scope}
	\end{scope}
	%\draw (-3,7) node {($6'$)};
	\node at (-4,6.3) {\TikzPosterTpathSixNineFontSize $b_4$};
	\node at (-3,-0.5) {\TikzPosterTpathSixNineFontSize $b_2$};
	\node at (0.5,3) {\TikzPosterTpathSixNineFontSize $b_3$};
	%\node[white] at (-6.2,3) {$b_1$};
	\node[] at (-8,3){}; \node[] at (2,3){};
	\end{tikzpicture}
}

\newcommand\TikzTpathSeven[2]{
	\begin{tikzpicture} [scale = {#1}]
	\begin{scope}[rotate=45]
	\MacroTtwoTpathChubby
	\begin{scope}[TPathStyle]
	\draw[rounded corners = 20pt]  (0,8) -- (3,5) -- (0,7) -- (-4,4);
	\draw  (-4,4) -- (-0.6,0.6);

	\MacroClockLoop{-0.6}{0.6}{0.6}{1.3}
	\MacroCounterClockRadius{0.6}{1.3}{-0.2*0}{0}
	\draw[->] (0,0) -- (3.8,3.8);
	\end{scope}
	\end{scope}
	%\draw (-3,7) node {($7$)};
	\node at (-4,6.3) {\TikzPosterTpathSixNineFontSize $b_4$};
	\node at (-3,-0.5) {\TikzPosterTpathSixNineFontSize $b_2$};
	\node at (0.5,3) {\TikzPosterTpathSixNineFontSize $b_3$};
	\node at \TikzPosterTpathSixNineCaptionLocation {\TikzPosterTpathSixNineCaptionSize #2};
	\end{tikzpicture}
}

\newcommand\MacroTikzTpathEight[1]{
	\begin{scope} [scale = {#1}]
	\begin{scope}[rotate=45]
	\MacroTtwoTpathChubby
	\begin{scope}[TPathStyle]
	\draw[rounded corners = 20pt]  (0,8) -- (-3,5) -- (0,7) -- (3,5) -- 
	(0,6) -- (-3,4) -- (0.1*0,-0.1*0);
	
	\MacroCounterClockRadius{0.1*0}{-0.1*0}{-0.5}{1.5}
	\MacroClockLoop{-0.5}{1.5}{1}{1}
	\draw[->] (1,1) -- (3.8,3.8);
	\end{scope}
	\end{scope}
	%\draw (-3,7) node {($8$)};
	\end{scope}
}

\newcommand\TikzTpathEight[2]{
	\begin{tikzpicture} [scale = {#1}]
	\MacroTikzTpathEight{1}
	\node at (-6.2,3) {\TikzPosterTpathSixNineFontSize $b_1$};
	\node at (0.5,3) {\TikzPosterTpathSixNineFontSize $b_3$};
	\node at (-4,6.3) {};
	\node at (-3,-0.5) {};
	\node at \TikzPosterTpathSixNineCaptionLocation {\TikzPosterTpathSixNineCaptionSize #2};
	\end{tikzpicture}
}

\newcommand\TikzTpathNine[2]{
	\begin{tikzpicture} [scale = {#1}]
	\begin{scope}[rotate=45]
	\MacroTtwoTpathChubby
	\begin{scope}[TPathStyle]
	\draw[rounded corners = 20pt]  (0,8) -- (-3,5) -- (0,7) -- (3,5) -- 
	(0,6) -- (-3,4) -- (-1,1);
	\MacroClockLoop{-1}{1}{0.3}{1}
	\MacroCounterClockRadius{0.3}{1}{0}{0}
	\draw[->] (0,0) -- (3.8,3.8);
	\end{scope}
	\end{scope}
	%\draw (-3,7) node {($9$)};
	\node at (-6.2,3) {\TikzPosterTpathSixNineFontSize $b_1$};
	\node at (0.5,3) {\TikzPosterTpathSixNineFontSize $b_3$};
	\node at (-4,6.3) {};
	\node at (-3,-0.5) {};
	\node at \TikzPosterTpathSixNineCaptionLocation {\TikzPosterTpathSixNineCaptionSize #2};
	\end{tikzpicture}
}


%%%%%%%%%%%%%%%%%%%%%%%%%%%%%%
%%%%%% making five paths on lifted polygon %%%%%%%

\newcommand\MacroTtwoLiftedPolygon{ % the macro lifted polygon for the T-path
	\begin{scope}[gray!75,style=thick]
	\tilTtwoIdealTriangulationCoordinates
	\draw %[style = thick, postaction={decorate}]
	 (x1) -- (-4,4) node[pos = 0.6, left=2pt]{$\ell$};
	\draw %[style = thick, postaction={decorate}] 
	(-4,4) -- (0,4); %node[pos=0.5,right] {$\dot{r}$}; %{$\tiltg{3}=\dot{r}$};
	\draw (-2,4.4) node {$\dot{r}$};
	\draw %[style = thick, postaction={decorate}] 
	(0,0) -- (0,4) node[pos=0.6, above]
	{$r$}; %{$\tiltaui{4}= r$}; 
	\draw %[style = thick, postaction={decorate}]
	(4,4) -- (0,0) node[pos = 0.5, right=2pt]{$\ell$};
	\draw %[style = thick, postaction={decorate}] 
	(4,4) -- (0,4) ;%node[pos=0.5,right] {$\ddot{r}$} ;%{$\tiltg{4}= \ddot{r}$};
	\draw (2,4.4) node {$\ddot{r}$};

	% The rest of the top rectangle
	\draw (x2) -- (t1) node[pos=0.5, above] {$b_3$};
	\draw (t1) -- (x1) node[pos=0.5, left] {$2$};
	\draw (x1) -- (t2) node[pos=0.6, left] {$b_3$}; %{$\tiltg{6}=b_3$};
	\draw (t2) -- (x3) node[pos=0.5, below] {$2$}; %{$\tiltg{5} = 2$};
	
	\draw (t1) -- (s) node[pos=0.5, above] {$b_4$}; %{$\tiltg{-1}=b_4$};
	\draw (s) -- (z); %node[pos=0.3, left] {$b_1$}; %{$\tiltg{0}=b_1$};
	\draw (-2,-4.4) node {$b_1$};
	\draw (z) -- (x1) node[pos=0.5, below] {$b_2$}; %{$\tiltg{1}=b_2$};
	\draw (t1) -- (z) node[pos=0.5, left] {$1$};

	\draw [fill=violet] (P) circle (1ex); 
	\end{scope}
}

\newcommand\rdPolygonTpath{10}
\newcommand \TikzPolygonTpathSix[1]{
\begin{tikzpicture}[scale={#1}, rotate=-90]
	\MacroTtwoLiftedPolygon
	\begin{scope}[TPathStyle]
	\draw [->, rounded corners = \rdPolygonTpath pt] (-4,-4) -- (-4,0) -- (0,-4) -- (0,0) [rounded corners = 0] --  (0,4)  [rounded corners= \rdPolygonTpath pt] --  (4,4) -- (0,0) -- (4,0);
	\end{scope}
	
	\end{tikzpicture}
}

\newcommand \TikzPolygonTpathSeven[1]{
\begin{tikzpicture}[scale={#1}, rotate=-90]
	\MacroTtwoLiftedPolygon
	\begin{scope}[TPathStyle]
	\draw [->, rounded corners = \rdPolygonTpath pt] (-4,-4) -- (-4,0) -- (0,-4) -- (0,0) -- (-4,4)  [rounded corners=0] -- (0,4)  [rounded corners=\rdPolygonTpath pt] -- (0,0) --(4,0);
	\end{scope}
	
	\end{tikzpicture}
}

\newcommand \MacroTikzPolygonTpathEight[1]{
\begin{scope}[scale={#1}, rotate=-90]
	\MacroTtwoLiftedPolygon
	\begin{scope}[TPathStyle]
	\draw [TPathStyle,->, rounded corners = \rdPolygonTpath pt] (-4,-4) -- (0,-4) -- (-4,0) -- (0,0)  [rounded corners=0] -- (0,4)  [rounded corners=\rdPolygonTpath pt] -- (4,4) -- (0,0) -- (4,0);
	\end{scope}
	\end{scope}
}
\newcommand \TikzPolygonTpathEight[1]{
\begin{tikzpicture}[scale={#1}]
\MacroTikzPolygonTpathEight{1}
\end{tikzpicture}
}

\newcommand \TikzPolygonTpathNine[1]{
\begin{tikzpicture}[scale={#1}, rotate=-90]
	\MacroTtwoLiftedPolygon
	\begin{scope}[TPathStyle]
	\draw [->, rounded corners = \rdPolygonTpath pt] (-4,-4) -- (0,-4) -- (-4,0) -- (0,0) -- (-4,4)  [rounded corners=0]-- (0,4)  [rounded corners=\rdPolygonTpath pt] -- (0,0) -- (4,0);
	\end{scope}
	\end{tikzpicture}
}

%%%%%
%%%%%

\newcommand\MacroChubbySelfFoldedTriangleTau{
	\draw[thick] (0,0) -- (0,2.9) node[pos=0.4, left=4pt]{$\tau$};
	\draw[style = thick, rounded corners = 20pt]%,gray!60]
	(0,0) .. controls (-2,1) and (-1.5,3) .. (0,4.2) .. 
	controls (2,3)  and (2.5,1) .. (0,0);
	\draw [fill=violet] (0,2.9) circle (1ex);
	\draw [olive] (0,0) node[below] {$v$};
}

\newcommand \TikzBacktrackNonbacktrackBacktrack[1] {
\begin{tikzpicture}[scale={#1} ]
	\MacroChubbySelfFoldedTriangleTau
	\draw[blue,style = dashed, line width = 1pt, rounded corners = 5pt,
	decoration={ markings, mark=at position .3 with {\arrow[line width=1pt]{to};},
	mark=at position .8 with {\arrow[line width=1pt]{to};}}, postaction={decorate}]
 	(-0.1,0) .. controls (-0.2,2) and (-0.3,2.2) .. (0,2.5) node[pos = 0.4, left]{$\om_j$} 
	node[pos=0.8,style=thick,fill=blue,blue,circle,minimum width=4pt,inner sep=0pt,draw] {}
	.. controls (0.2,2) and (0.1,1) .. (0.1,0) node[pos = 0.5, right]{$\om_{j+1}$};
	\draw[blue] (-0.3,2.5) node {$\puncture'$};
	\draw [fill=olive] (0,0) circle (.6ex); 
	\draw [violet] (-0.2,3.1) node {$\puncture$}; 
\end{tikzpicture}
}


\newcommand\MacroQuasiBacktrackBackground{
\newcommand\Tikzn{2}
\newcommand\Tikzwidth{1.2*\Tikzn}
\coordinate (P) at (0,2.9);
\coordinate (v) at (0,0);
\coordinate (left_top) at (-\Tikzwidth,4); 
\coordinate (right_top) at (\Tikzwidth,4); 
\coordinate (left_bottom) at (-\Tikzwidth,0);
\coordinate (right_bottom) at (\Tikzwidth,0);
\coordinate (left_mid) at (-0.8*\Tikzwidth,0.3*\Tikzn);
\coordinate (right_mid) at (0.8*\Tikzwidth,0.3*\Tikzn);
\draw [gray!40,fill=gray!40, rounded corners=2pt] (P) -- (left_top) -- (right_top) -- cycle;
\draw [gray!40,fill=gray!40, rounded corners=10pt] (left_bottom) -- (left_top) -- (left_mid) -- (v);
\draw [gray!40,fill=gray!40, rounded corners=10pt] (right_bottom) -- (right_top) -- (right_mid) -- (v) ;
\draw[dashed, rounded corners =10pt] (left_top) -- (left_mid) -- (v);
\draw[dashed, rounded corners =10pt] (right_top) -- (right_mid) -- (v);

\draw [rounded corners=2pt] (P) -- (left_top) ; 
\draw[dashed, rounded corners =2pt] (P) -- (right_top) -- (left_top);

	\draw[thick] (0,0) -- (0,2.9);% node[pos=0.8,red]{$k$};
\draw[rounded corners=2pt] (v) -- (-\Tikzwidth,0) -- (-\Tikzwidth,4) -- (\Tikzwidth,4) -- (\Tikzwidth,0) -- (v);

\draw [fill=olive] (v) circle (.6ex); \draw [fill=olive] (left_top) circle (.6ex); 
	\draw [fill=violet] (P) circle (1ex);
	\draw [olive] (v) node[below] {$v$}; %\draw [olive] (left_top) node[above] {$y$};
}

\newcommand\TikzBacktrackOddFirst[1]{
\begin{tikzpicture}[scale={#1},NonOrientedTriangulationStyle]	
	\MacroQuasiBacktrackBackground
	
	\begin{scope}
	\tikzset{anchor=west}
  	\node[rotate=90,red] at (0.2,1.5) {\tiny $k+1$};
	\end{scope}
        
	\draw[blue,style = dashed, line width = 1pt, rounded corners = 1pt,
	decoration={ markings, mark=at position .4 with {\arrow[line width=1pt]{to reversed};}},
	decoration={ markings, mark=at position .7 with {\arrow[line width=1pt]{to reversed};}},
	postaction={decorate}]
	(v) .. controls (-0.1,2) and (-0.3,2.5) .. (-0.5,2.8) node[pos = 0.2, left]{$\om_{2k+1}$}
	node[pos=1,style=thick,fill=blue,blue,circle,minimum width=4pt,inner sep=0pt,draw] {}
	.. controls (-0.7,2.3) and (-0.3,2) .. (-0.1,0) node[pos = 0.8, right=2pt]{$\om_{2k+2}$};
	\draw[blue] (-0.8,2.9) node {$\puncture'$};

	\draw[gray] (-0.5*\Tikzn,1*\Tikzn) node {$\blacktri{k}$};
	\draw[gray] (0.5*\Tikzn,1*\Tikzn) node {$\blacktri{k+1}$};
\end{tikzpicture}
}



\newcommand\TikzBacktrackEvenFirst[1]{
\begin{tikzpicture}[scale={#1},NonOrientedTriangulationStyle]
\MacroQuasiBacktrackBackground

	\draw (-0.15,2.3) node[red] {\tiny $k$};
        
	\draw[blue,style = dashed, line width = 1pt, rounded corners = 1pt,
	decoration={ markings, mark=at position .4 with {\arrow[line width=1pt]{to};}},
	decoration={ markings, mark=at position .7 with {\arrow[line width=1pt]{to};}},
	postaction={decorate}]
	(v) .. controls (0.1,2) and (0.3,2.5) .. (0.5,2.7) node[pos = 0.3, left]{$\om_{2k}$}
	node[pos=1,style=thick,fill=blue,blue,circle,minimum width=4pt,inner sep=0pt,draw] {}
	.. controls (0.7,2.3) and (0.5,2) .. (0.1,0) node[pos = 0.8, right]{$\om_{2k+1}$};
	\draw[blue] (0.8,3) node {$\puncture'$};

	\draw[gray] (-0.5*\Tikzn,1*\Tikzn) node {$\blacktri{k-1}$};
	\draw[gray] (0.5*\Tikzn,1*\Tikzn) node {$\blacktri{k}$};
\end{tikzpicture}
}


\newcommand\TikzQuasiBacktrackOddFirst[1]{
\begin{tikzpicture}[scale={#1},NonOrientedTriangulationStyle]
	\begin{scope}
	\tikzset{anchor=west}
  	\node[rotate=90,red] at (0.2,1.5) {\tiny $k+1$};
	\end{scope}
  
	\MacroQuasiBacktrackBackground
	
        \draw[blue,style = dashed, line width = 1pt, rounded corners = 5pt,
	decoration={ markings, mark=at position .6 with {\arrow[line width=1pt]{to};}},
	postaction={decorate}]
	(-0.3,2.7) .. controls (-0.2,2.5) and (-0.2,2) .. (-0.1,0) node[pos = 0.7, left]{$\om_{2k+1}$}
	node[pos=0,style=thick,fill=blue,blue,circle,minimum width=4pt,inner sep=0pt,draw] {};
	\draw[blue] (-0.6,2.8) node {$\puncture'$};
        
	\draw[blue,style = dashed, line width = 1pt, rounded corners = 5pt,
	decoration={ markings, mark=at position .4 with {\arrow[line width=1pt]{to};}},
	postaction={decorate}]
	(0.1,0) .. controls (0.1,2) and (0.7,2.5) .. (0.8,2.7) node[pos = 0.3, right]{$\om_{2k+2}$}
	node[pos=1,style=thick,fill=blue,blue,circle,minimum width=4pt,inner sep=0pt,draw] {};
	\draw[blue] (1.2,3) node {$\puncture''$};

	\draw [fill=olive] (v) circle (.6ex); \draw [fill=olive] (left_top) circle (.6ex); 
	\draw [fill=violet] (P) circle (1ex);
	\draw [olive] (v) node[below] {$v$}; %\draw [olive] (left_top) node[above] {$y$};
	%\draw [violet] (-0.4,3.1);
%	\draw[gray] (-0.5*\Tikzn,1*\Tikzn) node {$\blacktri{k}$};
%	\draw[gray] (0.5*\Tikzn,1*\Tikzn) node {$\blacktri{k+1}$};
	
\end{tikzpicture}
}

\newcommand\TikzQuasiBacktrackEvenFirst[1]{
\begin{tikzpicture}[scale={#1},NonOrientedTriangulationStyle]
	\MacroQuasiBacktrackBackground
	\draw (0.15,2.3) node[red] {\tiny $k$};
        \draw[blue,style = dashed, line width = 1pt, rounded corners = 5pt,
	decoration={ markings, mark=at position .6 with {\arrow[line width=1pt]{to};}},
	postaction={decorate}]
	(-0.3,2.7) .. controls (-0.2,2.5) and (-0.1,2.4) .. (-0.1,0) node[pos = 0.7, left]{$\om_{2k}$}
	node[pos=0,style=thick,fill=blue,blue,circle,minimum width=4pt,inner sep=0pt,draw] {};
	\draw[blue] (-0.6,2.8) node {$\puncture'$};
        
	\draw[blue,style = dashed, line width = 1pt, rounded corners = 5pt,
	decoration={ markings, mark=at position .4 with {\arrow[line width=1pt]{to};}},
	postaction={decorate}]
	(0.1,0) .. controls (0.1,2) and (0.6,2.5) .. (0.6,2.7) node[pos = 0.3, right]{$\om_{2k+1}$}
	node[pos=1,style=thick,fill=blue,blue,circle,minimum width=4pt,inner sep=0pt,draw] {};
	\draw[blue] (1,3.1) node {$\puncture''$};

%	\draw [fill=olive] (v) circle (.6ex); \draw [fill=olive] (left_top) circle (.6ex); 
%	\draw [fill=violet] (P) circle (1ex);
%	\draw [olive] (v) node[below] {$v$}; %\draw [olive] (left_top) node[above] {$y$};

	\draw[gray] (-0.5*\Tikzn,1*\Tikzn) node {$\blacktri{k-1}$};
	\draw[gray] (0.5*\Tikzn,1*\Tikzn) node {$\blacktri{k}$};
\end{tikzpicture}
}

%\newcommand \TikzNonloopBacktrack[1] {
%\begin{tikzpicture}[scale={#1} ]
%        \MacroChubbySelfFoldedTriangleTau
%        \draw[blue,style = dashed, line width = 1pt, rounded corners = 5pt,
%	decoration={ markings, mark=at position .6 with {\arrow[line width=1pt]{to};}},
%	postaction={decorate}]
%	(-0.3,3.2) .. controls (-0.4,3) and (-0.2,3) .. (-0.1,0) node[pos = 0.7, left]{$\om_k$}
%	node[pos=0,style=thick,fill=blue,blue,circle,minimum width=4pt,inner sep=0pt,draw] {};
%	\draw[blue] (-0.6,3.3) node {\tiny{$P'$}};
%        
%	\draw[blue,style = dashed, line width = 1pt, rounded corners = 5pt,
%	decoration={ markings, mark=at position .4 with {\arrow[line width=1pt]{to};}},
%	postaction={decorate}]
%	(0.1,0) .. controls (0.2,3) and (0.4,3) .. (0.3,3.3) node[pos = 0.3, right]{$\om_{k+1}$}
%	node[pos=1,style=thick,fill=blue,blue,circle,minimum width=4pt,inner sep=0pt,draw] {};
%	\draw[blue] (0.7,3.3) node {\tiny{$P''$}};
%
%	\draw [fill=olive] (0,0) circle (.6ex); 
%	\draw [violet] (-0.4,3.1);
%\end{tikzpicture}
%}

\newcommand \TikzBacktrackNonbacktrackCounterclock[1] {
\begin{tikzpicture}[scale={#1}]
	\MacroChubbySelfFoldedTriangleTau
	\draw[blue,style = dashed, line width = 1pt, rounded corners = 5pt,
	decoration={ markings, mark=at position .3 with {\arrow[line width=1pt]{to};},
	mark=at position .8 with {\arrow[line width=1pt]{to};}}, postaction={decorate}]
	(0.1,0) .. controls (0.2,3) and (0.4,3) .. (0,3.5) node[pos = 0.4, right]{$\om_{j+1}$} 
	node[pos=0.9,style=thick,fill=blue,blue,circle,minimum width=4pt,inner sep=0pt,draw] {}
	 .. controls (-0.4,3) and (-0.2,3) .. (-0.1,0) node[pos = 0.6, left]{$\om_j$};
	 \draw[blue] (0.4,3.6) node {\tiny{$P'$}};
	\draw [fill=olive] (0,0) circle (.6ex); 
	\draw [violet] (-0.4,3.1);
	\node at (3.5,0) {}; \node at (-3.5,0) {};
\end{tikzpicture}
}

\newcommand \TikzBacktrackNonbacktrackClock[1] {
\begin{tikzpicture}[scale={#1}]
	\MacroChubbySelfFoldedTriangleTau
	\draw[blue,style = dashed, line width = 1pt, rounded corners = 5pt,
	decoration={ markings, mark=at position .3 with {\arrow[line width=1pt]{to};},
	mark=at position .8 with {\arrow[line width=1pt]{to};}}, postaction={decorate}]
	(-0.1,0) .. controls (-0.2,3) and (-0.4,3) .. (0,3.5) node[pos = 0.4, left]{$\om_j$}
	.. controls (0.4,3) and (0.2,3) .. (0.1,0) node[pos = 0.6, right]{$\om_{j+1}$}
	node[pos=0.1,style=thick,fill=blue,blue,circle,minimum width=4pt,inner sep=0pt,draw] {};
	\draw[blue] (0.4,3.6) node {\tiny{$P'$}};
	\draw [fill=olive] (0,0) circle (.6ex); 
	\draw [violet] (-0.4,3.1); 
	\node at (3,0) {}; \node at (-3,0) {};
\end{tikzpicture}
}

%%%%% end of new
 \newcommand \MacroSelfFoldedLocalTriangulation {
	\coordinate (v1L) at (-2.2,1);  %\coordinate (v1L) at (-3,1); 
	\coordinate (v1R) at (3,1); 
	%\coordinate (v2L) at (-3.5,3); \coordinate (v2R) at (3.5,3);
	\coordinate (v2L) at (-2.2,3.5); \coordinate (v2R) at (3,3);

	\begin{scope}[orientedTriangulationStyle,gray]
		\draw [style = dashed, fill=gray!20]  % left gray area
		(0,0) .. controls (-3,2) and (-2,3.5) .. (0,4)  -- (v2L) -- (v1L) -- (0,0);
		\draw [style = dashed, fill=gray!20]  % right gray area
		(0,0) .. controls (3,2) and (2,3.5) .. (0,4)  -- (v2R) -- (v1R) -- (0,0);
	
		\draw[style = thick, postaction={decorate}] (0,0) -- (0,2.2) node[pos=0.6, right]{r};
		\draw[style = thick, NooseArrowAtPosStyleSix] (0,0) .. controls (-3,4) and (3,4) .. (0,0) 
		node[pos = 0.7, right=1.5pt,black]{$\ell$};
		\draw[style = thick, postaction={decorate}] (0,0) .. controls (-3,2) and (-2,3.5) .. (0,4) 
		node[pos = 0.3, left,black]{$\a$};
		\draw[style = thick, postaction={decorate}] (0,0) .. controls (3,2) and (2,3.5) .. (0,4) 
		node[pos = 0.3, right,black]{\tiny $\b$};
		\draw[fill=violet] (0,2.2) circle (0.6ex);
	\end{scope}
} 
  
\newcommand \MacroSelfFoldedLocalTriangulationVertices{
	\coordinate (Ps) at (0,2.2);  \coordinate (xs) at (0,0); \coordinate (ys) at (0,4);
	\draw [fill=olive] (xs) circle (.4ex); \draw (0,0) node[below=2pt, olive] {$v$};
	\draw [fill=olive] (ys) circle (.4ex); \draw (0,4) node[above=2pt, olive] {$y$};
	\draw [fill=violet] (Ps) circle (.6ex);  \draw[violet] (0,2.2) node[above=2pt] {\tiny $P$}; %Puncture	
}
 
\newcommand \MacroSelfFoldedLocalTriangulationAndMore{
	\MacroSelfFoldedLocalTriangulation
	\MacroSelfFoldedLocalTriangulationVertices
}
\newcommand \SelfFoldedLocalTriangulationGammaCrossesSelfFolded[1] {
\begin{tikzpicture}[scale={#1}] 
      	% gamma is peripheral crossing self-folded triangle
	\MacroSelfFoldedLocalTriangulationAndMore
	\draw [line width = 3pt, red, ->](-1.4,1.7) -- (1.4,1.7) node[pos = 0, above=1pt]{\Large{$\ga$}};
	\draw [fill=blue] (-0.8,1.7) circle (.6ex) node[below=7pt, left=2pt, blue] {$p_k$}; %pk, pk1, pk2
	\draw [fill=blue] (0,1.7) circle (.6ex) node[below=2pt, blue] {$p_{k+1}$}; %pk, pk1, pk2
	\draw [fill=blue] (0.8,1.7) circle (.6ex) node[below=8pt, right=2pt, blue] {$p_{k+2}$}; %pk, pk1, pk2
\end{tikzpicture}
}
\newcommand \SelfFoldedLocalTriangulationGammaCrossesSelfFoldedRadius[1] {
\begin{tikzpicture}[scale={#1}] 
	%\begin{scope}[xshift=6.5cm] % for when gamma is radius
	\MacroSelfFoldedLocalTriangulationAndMore
	\draw [line width = 3pt, red, ->](0,2.4) -- (-0.5,3.3) node[pos = 0.8, left=4pt]{\Large{$\ga$}};
	\draw [fill=blue] (-0.35,2.9) circle (.6ex) node[right=10pt, above=2pt, blue] {$p_1$}; %p1

	
\end{tikzpicture}
}
\newcommand \SelfFoldedLocalTriangulationGammaDoesntCrossSelfFolded[1] {
\begin{tikzpicture}[scale={#1}, orientedTriangulationStyle ]	
	\MacroSelfFoldedLocalTriangulation
	\draw [line width = 3pt, red, ->](0,0) .. controls (-4,4) and (3,4.5) .. (3,1.1) 
	node[pos = 0.8, right=2pt] {\Large{$\ga$}}; % red gamma
	\MacroSelfFoldedLocalTriangulationVertices
\end{tikzpicture}
}

\newcommand\TikzGammaK[1]{
\begin{tikzpicture}[scale={#1}] 	
	\MacroChubbySelfFoldedTriangle
			
	\draw [red, line width = 3pt,
	decoration={ markings, mark=at position .6 with {\arrow[line width=3pt]{to};}}, postaction={decorate}] (-1.6,1.3) -- (0,1.3) 
	node[pos=0.6, above]{\Large{$\ga_k$}};
	\draw [fill=blue] (-1.6,1.3) circle (.6ex) node[left, blue] {$p_k$}; 
	\draw [fill=blue] (0,1.3) circle (.6ex) node[right, blue] {$p_{k+1}$}; 	
\end{tikzpicture}
}

\newcommand\MacroEllPlusEllPlus{
	\draw[->,blue,style = dashed, line width = 2pt, rounded corners = 15pt,
	decoration={ markings, mark=at position .3 with {\arrow[line width=2pt]{to};},
	mark=at position .8 with {\arrow[line width=2pt]{to};}}, postaction={decorate}]
	(-1.8,1.2)  -- (-2,2) -- %(-1.7,3) -- 
	(-1.5,3.5) -- 
	%%.. controls (-2,1.5) and (-1.8,3) .. 
	(0,4.2) .. 	controls (3,2.8)  and (2.5,1) .. 
	(0,-0.4)  .. controls (-2,2) and (-1.6,3) .. 
	(0,3.8) .. 	controls (2,2.8)  and (1.5,1) .. 
	(0.2,0.2) -- (0.2,1.3) ;
}

\newcommand\TikzEllPlusEllPlus[1]{ % TODO not used I think
\begin{tikzpicture}[scale={#1}] 	
	\MacroChubbySelfFoldedTriangle
	\MacroEllPlusEllPlus
\end{tikzpicture}
}

\newcommand\TikzGammaKConcatenate[1]{
\begin{tikzpicture}[scale={#1}] 	
	\draw [fill=violet] (0,2.7) circle (0.6ex) node[right=1pt, violet] {\tiny $P$}; 
	 \draw [red, line width = 3pt, 
	decoration={ markings, mark=at position .5 with {\arrow[line width=3pt]{to};}}, postaction={decorate}]  (0.2,1.2) -- (-1.8,1.2)
	node[pos=0.3, above]{\Large{$\ga_k^-$}};
	\MacroEllPlusEllPlus
\end{tikzpicture}
}
\newcommand\MacroChubbySelfFoldedTriangle{
    \begin{scope}[gray!90]
	\draw[thick]
	(0,0) -- (0,2.5) node[black,pos=0.7, right]{$r$};
	\draw[fill=violet] (0,2.5) circle (0.6ex);
	\draw[fill=olive] (0,0) circle (0.4ex);
	\draw[style = thick, rounded corners = 20pt]
	(0,0) .. controls (-2.5,1) and (-2,3) .. (0,4) .. 
	controls (2,3)  and (2.5,1) .. (0,0) node[black,pos=0.5, right] {$\ell$};
    \end{scope}
}
\newcommand\MacroEllMinusRRLong{
	\draw[<-,blue,style = dashed, line width = 2pt, rounded corners = 15pt,
	decoration={ markings, mark=at position .5 with {\arrow[line width=2pt]{to reversed};}}, postaction={decorate}]
	(0.2,0.8) --  (-0.1,2.4) -- 
	(-0.4,0.3)  .. controls (-1.8,1.3) and (-2,2.8) .. (0,3.9) ..  % left half of ell
	controls (2,2.8)  and (1.5,1) .. (1,0.8) ;% right half of ell
}
\newcommand\MacroEllPlusEllMinusR{
		\draw[->,blue,style = dashed, line width = 2pt, 
		rounded corners = 15pt,
	decoration={ markings, mark=at position .3 with {\arrow[line width=2pt]{to};},
	mark=at position .85 with {\arrow[line width=2pt]{to};}}, 
	postaction={decorate}]
	(-1.8,1.5)  -- (-2,2) -- 
	(-1.5,3.5) -- 	(0,4.1) .. 	controls (2.5,2.8)  and (2,1.5) .. (0.6,0.4) 
	 ..  controls  (1.5,1) and  (2,2.8)  .. 
	(0,3.8) .. controls (-1.6,3) and (-2.3,2)  .. 
	(0,0) -- (0,2.2) ;
}
\newcommand\MacroRLMinLPlus{
		\draw[->,blue,style = dashed, line width = 2pt, rounded corners = 15pt,
	decoration={ markings, mark=at position .06 with {\arrow[line width=2pt]{to};},
	mark=at position .15 with {\arrow[line width=2pt]{to};}}, 
	postaction={decorate}]
	(-0,2) -- (-0,0) .. controls (1.5,1) and (2,2.8)  .. (0,3.8) 
	 .. 
	 controls (-1.6,3) and (-2.3,2)  .. 
	  (-1,0.6)  ..
	controls (-2.2,2)  and  (-1.7,3.3)  .. (0,4.1) .. controls (1.7,3.3) and (2.2,2) .. (1.6,1.3);
}



\newcommand\MacroEllPlusRR{
	\draw[->,blue,style = dashed, line width = 2pt, rounded corners = 15pt,
	decoration={ markings, mark=at position .2 with {\arrow[line width=2pt]{to};}}, postaction={decorate}]
	(-1.4,1.2)  .. controls (-1.7,2) and (-1.6,3) .. (0,3.8) ..  % left half of ell
	controls (2,2.8)  and (1.5,1) .. (0.3,0.2) % right half of ell
	-- (0.1,2.5) -- (-0.3,1);
}
\newcommand\MacroRREllMinus{
	\draw[<-,blue,style = dashed, line width = 2pt, rounded corners = 15pt,
	decoration={ markings, mark=at position .5 with {\arrow[line width=2pt]{to reversed};}}, postaction={decorate}]
	(-1.4,1.2)  .. controls (-1.7,2) and (-1.6,3) .. (0,3.8) ..  % left half of ell
	controls (2,2.8)  and (1.5,1) .. (0.2,0.2) % right half of ell
	-- (0.1,2.5) -- (-0.3,1);
}
\newcommand\TikzEllPlusRR[1]{ % todo not used
\begin{tikzpicture}[scale={#1}] 	
	\MacroChubbySelfFoldedTriangle
	\MacroEllPlusRR
\end{tikzpicture}
}
\newcommand\TikzEllPlusRRConcatenate[1]{
\begin{tikzpicture}[scale={#1}] 	
	\MacroChubbySelfFoldedTriangle
	 \draw [red, line width = 3pt, 
	decoration={ markings, mark=at position .5 with {\arrow[line width=3pt]{to};}}, postaction={decorate}]  (-0.3,1.1) -- (-1.4,1.1)
	node[pos=0.5, above]{\Large{$\ga_k^-$}};
	\MacroEllPlusRR
\end{tikzpicture}
}

\newcommand\MacroEllMinusRR{
	\draw[->,blue,style = dashed, line width = 2pt, rounded corners = 10pt,
	decoration={ markings, mark=at position .2 with {\arrow[line width=2pt]{to};}}, postaction={decorate}]
	(-1.6,1)  --  (-1.3,0.4) -- (0.2,0)  
	-- (0.2,2.5) -- (-0.3,1);
}
\newcommand\TikzEllMinusRRConcatenate[1]{
\begin{tikzpicture}[scale={#1}] 	
	\MacroChubbySelfFoldedTriangle
	 \draw [red, line width = 3pt, 
	decoration={ markings, mark=at position .5 with {\arrow[line width=3pt]{to};}}, postaction={decorate}]  (-0.3,1.1) -- (-1.6,1.1)
	node[pos=0.5, above]{\Large{$\ga_k^-$}};
	\MacroEllMinusRR
\end{tikzpicture}
}


\newcommand\MacroEllPlusRRnonBT{
	\draw[->,blue,style = dashed, line width = 2pt, rounded corners = 15pt,
	decoration={ markings, mark=at position .2 with {\arrow[line width=2pt]{to};}}, postaction={decorate}]	
	(-1.4,1.2)  -- (-1.8,1.8) --  % ell
	(-1.5,3.2) --  % ell
	(0,3.8) .. controls (2.6,2.8)  and (1.5,1) .. % right part of ell
	(0.4,0.2) .. controls (0.3,1) and (0.5,2.8) .. % non backtrack rr
	(-0.4,3.3) .. controls (-0.3, 2) and (-0.2,1.5) .. % non backtrack rr
	(-0.3,1); % non backtrack rr
}

\newcommand\MacroRREllPlusnonBT{
	\draw[->,blue,style = dashed, line width = 2pt, rounded corners = 15pt,
	decoration={ markings, mark=at position .3 with {\arrow[line width=2pt]{to};}},
	decoration={ markings, mark=at position .6 with {\arrow[line width=2pt]{to};}}, postaction={decorate}]	
	(0.3,0.6) .. controls (0.3,1) and (0.5,2.8) .. % non backtrack rr
	(-0.4,3.3) .. controls (-0.3, 2) and (-0.2,1.5) .. % non backtrack rr
	(-0.2,0) % non backtrack rr
	--
	(-1.4,0.7)  -- (-1.8,1.8) --  % ell
	(-1.5,3.2) --  % ell
	(0,3.8) .. controls (2.6,2.8)  and (1.5,1) .. % right part of ell
	(0.8,0.4);
}

\newcommand\TikzEllPlusRRnonBTConcatenate[1]{
\begin{tikzpicture}[scale={#1}] 	
	\MacroChubbySelfFoldedTriangle
	%\draw [fill=violet] (0,2.7) circle (0.6ex) node[below=1pt, violet] {\tiny $P$}; 
	 \draw [red, line width = 3pt, 
	decoration={ markings, mark=at position .5 with {\arrow[line width=3pt]{to};}}, postaction={decorate}]  (-0.3,1.1) -- (-1.4,1.1)
	node[pos=0.5, above]{\Large{$\ga_k^-$}};
	\MacroEllPlusRRnonBT
\end{tikzpicture}
}
\newcommand\TikzEllPlusRRnonBT[1]{
\begin{tikzpicture}[scale={#1}] 	
	\MacroChubbySelfFoldedTriangle
	%\draw [fill=violet] (0,2.7) circle (0.6ex) node[below=1pt, violet] {\tiny $P$}; 
	\MacroEllPlusRRnonBT
\end{tikzpicture}
}
%%%%%%%%%
%%%%%%%%%
\newcommand \tilTriangleSelfFoldedK[4] {
\begin{scope} [scale={#1}, orientedTriangulationStyle ]
	\draw[style = thick, postaction={decorate}] (0,0) -- (-4,4) node[pos = 0.7, left=2pt]{#2};
	\draw[style = thick, postaction={decorate}] (-4,4) -- (0,4) node[pos=0.5, above=6pt]
	{#3};
	\draw[style = thick, postaction={decorate}] (0,0) -- (0,4) node[pos=0.3,right=2pt]
	{#4}; % common edge of both triangles
	\draw[fill=violet] (0,4) circle (.6ex);
\end{scope}
} 
\newcommand \tilTriangleSelfFoldedKplusOne[3] { 
\begin{scope} [scale={#1}, orientedTriangulationStyle ]
	\draw[style = thick, postaction={decorate}] (4,4) -- (0,0) node[pos = 0.3, right=2pt]{#2};
	\draw[style = thick, postaction={decorate}] (4,4) -- (0,4) node[pos=0.5, above=6pt]
	{#3};
\end{scope}
}

\newcommand \tilDoubleSelfFoldedTriangleComplicatedLabels[1]{  
\begin{scope}[gray]
\tilTriangleSelfFoldedK{#1}{$\tiltaui{k}=\ell$}{$\tiltg{k}=\dot{r}$}{$\tiltaui{k+1}= r$}
\tilTriangleSelfFoldedKplusOne{#1}{$\tiltaui{k+2}=\ell$}{$\tiltg{k+1}= \ddot{r}$}
\end{scope}
}
\newcommand \tilSDoubleTriangleForPeripheralGamma[1] {
	\tilDoubleSelfFoldedTriangleComplicatedLabels{#1}	
	\begin{scope}[scale={#1}]
	\coordinate (P) at (0,4);
	\draw [fill=olive] (0,0) circle (.6ex); \draw (0,0) node[below=3pt,olive] {$\tilv^1$};
	\draw [fill=olive] (-4,4) circle (.6ex); \draw (-4,4) node[left=2pt,olive] {$\tilv^2$};
	\draw [fill=olive] (4,4) circle (.6ex); \draw (4,4) node[right=2pt,olive] {$\tilv^3$};
	\draw [fill=violet] (P) circle (.8ex); %\draw (0,4) node[above=3pt, violet] {$\tilP$}; % Puncture
	\draw [fill=blue] (-2,2) circle (.6ex) node[left=4pt, blue] {$\tilp_k$}; %pk, 
	\draw [fill=blue] (0,2) circle (.6ex) node[left=10pt, below=3pt, blue] {$\tilp_{k+1}$};%pk1, pk2
	\draw [fill=blue] (2,2) circle (.6ex) node[right=10pt, blue] {$\tilp_{k+2}$};%pk2
	\end{scope}
}
\newcommand \tilTriangleSelfFoldedOne[4] {
\begin{scope} [scale={#1}, orientedTriangulationStyle ]
	\draw[postaction={decorate}] (0,4) -- (0,0) node[pos=0.4, left=3pt]
	{#2}; % vertical lift of r
	\draw[postaction={decorate}] (4,0) -- (0,0) node[pos=0.4, below=3pt]
	{#3}; % horizontal lift of r
	\draw[postaction={decorate}] (0,4) -- (4,0) node[pos=0.2, right=6pt]
	{#4}; % diagonal, lift of ell
	\draw[fill=violet] (0,0) circle (.6ex);
\end{scope}
}
\newcommand \tilTriangleSelfFoldedTwo[3] {
 \begin{scope}  [scale={#1}, orientedTriangulationStyle ]
  	\draw[postaction={decorate}] (0,4) -- (4,4) node[pos=0.6, above]
	{#2}; % lift of a (horizontal)
	\draw[postaction={decorate}] (4,0) -- (4,4) node[pos=0.7, right=2pt]
	{#3}; % lift of b (vertical)
\end{scope}
}

\newcommand \tilSTriangleOneTwoForRadiusGamma[1] {
	\begin{scope}[gray,scale={#1}]
	
	\coordinate (P2) at (0,4); \coordinate (P1) at (4,0);
	 \tilTriangleSelfFoldedOne{1}{$\ddot{r}$}{$r$}{$\tiltaui{1} = \tilell$}
	 \tilTriangleSelfFoldedTwo{1}{$\a$}{\tiny $\b$} 	
	  
	  % vertices	
	\draw [fill=violet] (0,0) circle (.9ex); \draw (0,0) node[below=3pt,violet] {$\tils = \tilP$};
	\draw [fill=olive] (4,4) circle (.6ex); \draw (4,4) node[above=3pt, right=2pt,olive] {$\tily$};
	\draw [fill=olive] (P2) circle (.6ex); \draw (P2) node[above=3pt,violet] {$\tilv^2$};%v lift
	\draw [fill=olive] (P1) circle (.6ex); \draw (P1) node[below=3pt,violet] {$\tilv^2$};%v
	\draw [fill=blue] (2,2) circle (.6ex) node[right=7pt, blue] {$\tilp_{1}$};%p1
	
	\end{scope}
}
%%%%%%%%%%%%%  
\newcommand \MacroTtwoTriangulationCoordinates{
 \coordinate (t) at (0,0); 
 \coordinate(s) at (4,0);
  \coordinate(vL) at (0,4);  \coordinate(vR) at (4,4);
   \coordinate(x) at (0,4);  \coordinate(z) at (4,4);
 }

%\newcommand \MacroTtwoTriangulationVertices{
% \draw [fill=olive] (0,0) circle (.6ex);
%\draw [fill=olive] (0,4) circle (.6ex);
% \draw (-0.3,4.2) node[olive] {$v$};
%\draw [fill=olive] (4,0) circle (.6ex);
%\draw [fill=olive] (4,4) circle (.6ex);
%\draw [fill=violet] (1.2,2.2) circle (.6ex); %puncture
%% \draw (1.4,2) node[violet] {\tiny $P$};
%  }


 \newcommand \MacroTtwoIdealTriangulationWithNumberLabels {
 \MacroTtwoTriangulationCoordinates
 \newcommand\tempfontsize{\small}
\begin{scope}[NonOrientedTriangulationStyle]
\draw[postaction={decorate}]  (s) -- (vR) node[pos=0.5,left] {\tempfontsize $b_1$};
\draw[postaction={decorate}]  (vR) -- (vL) node[pos=0.5,below]{\tempfontsize $b_2$};
\draw[postaction={decorate}]  (vL) -- (t) node[pos=0.5,right] {\tempfontsize $b_3$};
\draw[postaction={decorate}]  (t) --(s) node[pos=0.5,above] {\tempfontsize $b_4$};
\draw [postaction={decorate}] (t) ..  controls(3.5,0.4) and (3.5,2) .. (vR) node[left, pos=0.3] {\tempfontsize $1$};
\draw [postaction={decorate}] (t) ..  controls (3.7,1) and (3.8,3.9)   .. (vL) node[right, pos=0.5] {\tempfontsize $2$};%;
\draw [NooseArrowAtPosStyleSix] (vL) ..  controls (3.9,1.4)  and (0.7,0.1)  .. (vL) node[pos=0.7, right,above] {\tempfontsize $\ell$}; %loop
\draw[postaction={decorate}] (vL) -- (1.2,2.2) node[left, pos=0.7] {\tempfontsize $r$};
\end{scope}
 } 
  \newcommand\TikzTtwoIdealTriangulationWithNumberLabels[1]{ % USED
 	\begin{tikzpicture} [scale = {#1}, rotate=180] %% T2 triangulation tagged (IDEAL)
	\begin{scope}
	\MacroTtwoIdealTriangulationWithNumberLabels
	\draw [<-,decorate, line width=2 pt, dashed, red!60] (0,0) ..  controls (0,5) and (4,4)   .. (4,0)
	node[sloped, rotate=90, pos=0.7, below=6pt] {\small{$\ga$}}; % red gamma
	%\MacroTtwoTriangulationVertices
	\draw [fill=violet] (1.2,2.2) circle (.3ex); %puncture
	\draw [fill=olive] (t) circle (.2ex); 
	\draw [fill=olive] (vL) circle (.2ex); 
	\draw [fill=olive] (s) circle (.2ex); 
	\draw [fill=olive] (vR) circle (.2ex); 
	\end{scope}
\end{tikzpicture}
}  
\newcommand  \TikzTtwoTagged[1]{
 \begin{tikzpicture} [scale =#1]%% T2 triangulation tagged (IDEAL)
 \begin{scope}[rotate=180]
\MacroTtwoTriangulationCoordinates
\draw[style = thick]  (0,0) rectangle (4,4);
\draw [style = thick] (0,0) ..  controls(3.5,0.4) and (3.5,2) .. (4,4);
\draw [style = thick] (0,0) ..  controls (3.7,1) and (3.8,3.9)   .. (0,4);
\draw [style = very thick, gray] (0,4) ..  controls (0.7,0.1) and (3.9,1.4)   .. (0,4) node[pos=0.75, left] {$\ell$}; %loop
\draw[style = thick] (0,4) -- (1.2,2.2);
\MacroTtwoTriangulationVertices
\end{scope}
\end{tikzpicture}
\begin{tikzpicture}[scale=#1] %% T2 triangulation tagged
\begin{scope}[rotate=180]
\MacroTtwoTriangulationCoordinates
\draw[style = thick]  (0,0) rectangle (4,4); 
\draw [style = thick] (0,0) ..  controls(3.5,0.4) and (3.5,2) .. (4,4); % the right most peripheral arc
\draw [style = thick] (0,0) ..  controls (3.7,1) and (3.8,3.9)   .. (0,4); % the middle arc 
\draw[style = thick] (0,4) .. controls (0.8,1) and (1,2) .. (1.2,2.2);  % the radius (tagged plain) arc
\draw[style = very thick, gray] (0,4) -- (1.2,2.2) node[pos=0.5, left=1pt] {$\iota(\ell)$};  
\path[fill, gray] (1,2.5) -- (1.1,2.8) -- (1.3,2.5) -- cycle; % the notching (right)
\path[fill, gray] (1,2.5) -- (0.9,2.2) -- (0.7,2.5) -- cycle; % the notching (left)
\MacroTtwoTriangulationVertices
\end{scope}
\end{tikzpicture}
 }
%%%%%%%
%%%%%% T1 triangulation %%%%
\newcommand \TikzTone{ %Draw the triangulation where the arcs are labeled 1,2,3,4
\draw[style = thick]  (0,0) rectangle (4,4);
\draw [style = thick] (0,0) ..  controls(3.5,0.4) and (3.5,2) .. node[pos =0.4,right=2pt] {$1$}  (4,4);
\draw [style = thick] (0,0) ..  controls (3.7,1) and (3.8,3.9)   .. node[pos =0.4,left=2pt] {$2$}  (0,4);
\draw[style = thick] (0,4) -- node[pos =0.3,right=1pt] {$3$}  (1.2,2.2);
\draw[style = thick] (1.2,2.2) -- node[pos =0.4,right=1pt] {$4$}  (0,0);
\draw [fill=violet] (1.2,2.2) circle (.7ex); %puncture
\draw [fill=olive] (0,0) circle (.6ex);
\draw [fill=olive] (0,4) circle (.6ex);
\draw [fill=olive] (4,0) circle (.6ex);
\draw [fill=olive] (4,4) circle (.6ex);
\draw (4,2) node [right = 5pt] {\Large{$b_1$}};
\draw  (2,4) node [above = 5pt] {\Large{$b_2$}};
\draw  (0,2) node [left = 5pt] {\Large{$b_3$}};
\draw  (2,0) node [below = 5pt] {\Large{$b_4$}};
 }
\newcommand  \TikzToneWithGamma[1]{
 \begin{tikzpicture} [scale=#1]
\TikzTone %Draw the triangulation where the arcs are labeled 1,2,3,4
\draw [red] (4,0) node [right = 5pt] {\Large{$s$}};
\draw [red] (0,0) node [left = 5pt] {\Large{$t$}};
\draw [<-,decorate, line width=3 pt, dashed, red] (0,0) ..  controls (0,5) and (4,4)   .. (4,0)
node[pos=0.7, above=2pt] {\Large{$\ga$}}; % red gamma
\end{tikzpicture}
  }
\newcommand \TikzToneBlackTriangles[1]{
\begin{tikzpicture} [scale=#1]
\MacroTikzToneTgk %Draw just the triangulation where the arcs are labeled taui or tgk
\draw[gray] (3.3,0.5) node {\Large{$\blacktri{0}$}};
\draw[gray]  (3.2,3.2) node {\Large{$\blacktri{1}$}};
\draw[gray]  (2,2.5) node {\Large{$\blacktri{2}$}};
\draw[gray]  (0.5,2) node {\Large{$\blacktri{3}$}};
\end{tikzpicture}
}
\newcommand \MacroTikzToneTgk{
\begin{scope}[gray]
\draw [style = thick]  (0,0) rectangle (4,4);
\draw [style = thick] (0,0) ..  controls(3.5,0.4) and (3.5,2) .. node[black,pos =0.4,right=2pt] {$\taui{1}$}  (4,4);
\draw [style = thick] (0,0) ..  controls (3.7,1) and (3.8,3.9)   .. node[black,pos =0.4,left=2pt] {$\taui{2}$}  (0,4);
\draw [style = thick] (0,4) -- node[black, pos =0.4,right=2pt] {$\taui{3}$}  (1.2,2.2);
\draw [style = thick] (1.2,2.2) -- node[black, pos =0.4,right=2pt] {$\tg{2}$} 
node [pos=0.6, right, black] {$=\tg{d}$}  (0,0);
\draw [fill=violet] (1.2,2.2) circle (.7ex); %puncture
\draw [fill=olive] (0,0) circle (.6ex);
\draw [fill=olive] (0,4) circle (.6ex);
\draw [fill=olive] (4,0) circle (.6ex);
\draw [fill=olive] (4,4) circle (.6ex);
\draw (4,2) node [right = 2pt, black]  {$\tg{0}$};
%\draw (4,1.5) node[right = 5pt, blue] {$\tg{0}$};
\draw (1.5,4) node [above = 2pt, black] {$\tg{1}$};
\draw  (0,2) node [left = 2pt,black]  {$\tg{d+1}$};
%\draw (0,1.5) node[left = 5pt, blue] {$\tg{d+1}$};
\draw  (2,0) node [below = 2pt,black] {$\tg{-1}$};
%\draw  (3,0) node [below = 5pt, blue] {$\tg{-1}$};
%\draw  (2.8,4) node [blue, above=2pt] {$\tg{1}$};
\end{scope}
}

%%%%%%%%
\newcommand \MacroSelfFolded[2] {
\begin{scope}[scale={#1}, {#2},gray]
	\draw[postaction={decorate}] (0,0) -- (0,2) node[pos=0.7, right=4pt]{r};
	\draw[fill=violet] (0,2) circle (0.3ex);
	\draw[fill=olive] (0,0) circle (0.2ex);
	\draw[postaction={decorate}] (0,0) .. controls (-3,4) and (3,4) .. (0,0) node[pos = 0.3, left]{$\ell$};
\end{scope}
}

%%%%

\newcommand\DiskWithPunctureTauPeripheral[1]{
\begin{tikzpicture}[scale=#1]
\begin{scope}[xscale=-1]
	\path (0:0) coordinate (origin);
	\path (180:2) coordinate (4);
	\path (255:2) coordinate (v1);
	\path (300:2) coordinate (v1');
	\path (7*50:2) coordinate (7);
	
	\draw [fill=violet] (origin) circle (.6ex); \draw (origin) node[above=2pt, violet] {\tiny $P$};
	\draw[thick] (origin) circle (2);
	\draw[thick] (v1) -- (7) node[pos=0.4,below]{$\al$}; %v1 to vertex (alpha)
	\draw[thick, dotted] (v1) -- (4) node[pos=0.4,right]{$\lambda$}; 

	\draw [->, line width=3 pt, dashed, red] (v1) ..  controls (-2.5,2.5) and (1.5,2.5)   .. (v1')
	node[pos=0.6, left=2pt] {\Large{$\si$}}; % red sigma between (5)=(p1) and (6)=p2
	
	\draw [fill=olive] (v1) circle (.4ex); \draw (v1) node[below=3pt] {$v_1$};
	\draw [fill=olive] (v1') circle (.4ex); \draw (v1') node[below=3pt] {$v_1'$};	
	\draw [fill=olive] (7) circle (.4ex) node[left=3pt] {$v_2$}; 
	\draw[thick, rounded corners=20pt] (4) -- (270:0.8) -- (7) node[pos=0.2, above] {$\taui{1}$}; % tau is peripheral	
	\draw [fill=olive] (4) circle (.4ex) node[right=3pt] {$v_0$}; 
	\draw (290:1.1) node[right=3pt] {$\Large{\blacktri{0}}$}; 
\end{scope}
\end{tikzpicture}
  }
    
  
\newcommand\TikzTauPeripheralSelfFolded[1]{
\begin{tikzpicture}[scale=#1]
\begin{scope}[xscale=-1]
	\path (0:0) coordinate (origin);
	\path (180:2) coordinate (4);
	\path (255:2) coordinate (v1);
	\path (300:2) coordinate (v1');
	\path (7*50:2) coordinate (7);
	\path (80:2) coordinate (80);
	
	\draw [fill=violet] (origin) circle (.6ex); \draw (origin) node[above=2pt, violet] {\tiny $P$};
	\draw[thick] (origin) circle (2);
	
	\draw[thick, rounded corners=18pt] (v1) --  (320:1.6) -- (30:1.7) 
	node[pos=0.5,left]{$\al$} -- (80) ; %v1 to vertex (alpha)
	
	\draw[thick, dotted, rounded corners=18pt] (v1) -- (210:1.8) -- (160:1.8) -- (100:1.8) 
	node[pos=0.5,right]{$\lambda$} -- (80); 

	\draw [->, line width=3 pt, dashed, red] (v1) ..  controls (-2.5,2.2) and (1.5,2.2)   .. (v1')
	node[pos=0.65, above] {\Large{$\si$}}; % red sigma between (5)=(p1) and (6)=p2
	
	\draw [fill=olive] (v1) circle (.4ex); \draw (v1) node[below=3pt] {$v_1$};
	\draw [fill=olive] (v1') circle (.4ex); \draw (v1') node[below=3pt] {$v_1'$};	
	
	\draw[thick, rounded corners=15pt] (80) --  (175:0.8) -- (270:0.8) node[pos=0.6, left] {$\taui{1}$} -- (0:1) -- (80) ; % tau is ell-loop	
	
	\draw [fill=olive] (80) circle (.4ex) node[above=3pt] {$v_0=v_2$}; 

	
	\draw (290:1.1) node[right=3pt] {$\Large{\blacktri{0}}$}; 
\end{scope}
\end{tikzpicture}
  }  
  
  
\newcommand\MacroPuncturedDiskMarkedPoints[1]{
	\draw[fill=olive] (0:2) circle ({#1});
	\draw[fill=olive] (20:2) circle ({#1}); 
	\draw[fill=olive] (50:2) circle ({#1}); 
	\draw[fill=olive] (90:2) circle ({#1}); 
	\draw[fill=olive] (120:2) circle ({#1}); 
	\draw[fill=olive] (150:2) circle ({#1}); 
	\draw[fill=olive] (160:2) circle ({#1});  
	\draw[fill=olive] (210:2) circle ({#1}); 
	\draw[fill=olive] (240:2) circle ({#1});
	\draw[fill=olive] (270:2) circle ({#1}); 
	\draw[fill=olive] (300:2) circle ({#1}); 
	\draw[fill=olive] (330:2) circle ({#1});
}
\newcommand\DiskLambda[1]{
\begin{tikzpicture}[scale=#1]
	\draw[fill=violet] (0:0) circle (\TikzPunctureActualSize); 
	\draw (0:0) node[left=4pt,below=3pt,violet] {\tiny $\puncture$};	
	\draw[fill,gray!20]  (20:2) -- (30:2) -- (40:2) -- (50:2) -- (70:2) -- (90:2) -- (110:2) -- (130:2) -- (150:2) -- (160:2) --  cycle;
	\draw[thick] (0:0) circle (2);
	\draw[line width=2pt,dashed, cyan] (20:2) -- (160:2) node[pos=0.2,below,cyan] {$\Large{\lambda}$};
	\draw (90:1.2) node[cyan] {$\Large{\disk{\lambda}}$};
	\MacroPuncturedDiskMarkedPoints{.4ex}
\end{tikzpicture}
}

\newcommand\OncePuncturedDiskLambda[1]{
\begin{tikzpicture}[scale=#1]
	\coordinate (vleft) at (150:2); \coordinate (vright) at (30:2);
	\draw[fill=gray!20] (0:0) circle (2);
	\draw[fill,white]  (vright)  -- (40:2) -- (50:2) -- (70:2) -- (90:2) -- (110:2) -- (130:2) -- (150:2) -- (vleft) --  cycle;
	\draw[black,thick] (0:0) circle (2);
	\draw[line width=2pt,dashed, cyan] (vright) -- (vleft) node[pos=0.4,above,cyan] {$\Large{\lambda}$};
	\draw (-90:1) node[cyan] {$\Large{\outsidedisk{\lambda}}$};
	\MacroPuncturedDiskMarkedPoints{.4ex}
	\draw[fill=violet] (0:0) circle (\TikzPunctureActualSize) (0:0) node[left=4pt,above,violet] {\tiny $\puncture$};	
	
	\node at (0:3) {};\node at (180:3) {};
\end{tikzpicture}
}

\newcommand\MacroPuncturedDisk{
	\draw[fill=violet] (0:0) circle (.6ex); 
	\draw (0:0) node[left=4pt,below=3pt,violet] {\tiny $P$};
	\draw[thick] (0:0) circle (2);	
}
\newcommand\MacroPuncturedDiskNoP{
	\draw[fill=violet] (0:0) circle (.6ex); 
	\draw (0:0); 
	\draw[thick] (0:0) circle (2);	
}

\newcommand\ChoiceOfSigmaLambdaKs[1]{
\begin{tikzpicture}[scale=#1]
	\MacroPuncturedDiskNoP
	\draw[line width=2pt,dashed, red] (20:2) -- (160:2) node[pos=0.3, below,red] {$\Large{\lambda_r}$};
	\draw[line width=2pt,dashed, cyan] (50:2) -- (150:2); 
	\draw[line width=2pt,dashed, cyan, rounded corners] (90:2) -- (100:1.6) -- (150:2); 
	\draw[line width=2pt,dashed, red, rounded corners=20pt] (20:2) -- (0:1) -- (330:2) node[pos=0.8, left,red] {$\Large{\lambda_1}$};
	\draw[line width=2pt,dashed, red, rounded corners=20pt] (210:2) -- (240:0.7) node[pos=0.4, above,red] {$\Large{\lambda_2}$} -- (300:2);
	\draw[line width=2pt,dashed, cyan, rounded corners=20pt] (210:2) -- (240:1) -- (270:2);
	\MacroPuncturedDiskMarkedPoints{.4ex}
	\node at (0:3) {};\node at (180:3) {};
\end{tikzpicture}
}

\newcommand\ChoiceOfSigmaLambdaR[1]{
\begin{tikzpicture}[scale=#1]
	\MacroPuncturedDiskNoP
	\draw[line width=2pt,dashed, red] (20:2) -- (160:2) node[pos=0.45, above,red] {$\Large{\si:=\lambda_r}$};
	\draw[line width=2pt,dashed, cyan] (50:2) -- (150:2); 
	\draw[line width=2pt,dashed, cyan, rounded corners] (90:2) -- (100:1.6) -- (150:2); 
	\draw[line width=2pt, rounded corners=15pt] (120:2) -- (220:1) node[pos=0.9, left] {$\Large{\tau}$} -- (320:1) -- (50:2); 
	\MacroPuncturedDiskMarkedPoints{.4ex}
	\node at (0:3) {};\node at (180:3) {};
\end{tikzpicture}
}

\newcommand\ChoiceOfSigmaGoodA[1]{
\begin{tikzpicture}[scale=#1]
	\MacroPuncturedDisk
	\draw[line width=2pt,dashed, red] (20:2) -- (160:2) node[pos=0.2, below,red] {$\Large{\si}$};
	\draw[line width=2pt,dashed, cyan] (50:2) -- (150:2) node[pos=0.2, below,cyan] {$\Large{\be}$};
	\MacroPuncturedDiskMarkedPoints{.4ex}
	\node at (0:3) {}; \node at (180:3) {};
\end{tikzpicture}
}
\newcommand\ChoiceOfSigmaGoodB[1]{
\begin{tikzpicture}[scale=#1]
	\MacroPuncturedDisk
	\draw[line width=2pt,dashed, red] (20:2) -- (160:2) node[pos=0.2, below,red] {$\Large{\si}$};
	\draw[line width=2pt,dashed, cyan] (330:2) -- (210:2) node[pos=0.2, below,cyan] {$\Large{\be}$};
	\MacroPuncturedDiskMarkedPoints{.4ex}
	\node at (0:3) {}; \node at (180:3) {};
\end{tikzpicture}
}
%%%%
%%%%
\newcommand\MacroFourCover {
\begin{scope}[gray]
	\path (0:0) coordinate (origin);
	\path (0:2) coordinate (v0);
	\path (40:2) coordinate (v40);
	\path (70:2) coordinate (v70); 
	\path (90:2) coordinate (v90);
	\path (130:2) coordinate (v130);
	\path (160:2) coordinate (v160); 
	\path (180:2) coordinate (v180);
	\path (270:2) coordinate (v270);
	
	\draw [fill=violet] (origin) circle (0.6ex); \draw (origin) node[left=4pt,below=3pt, violet] {\tiny $P$};
	\draw[thick] (origin) circle (2);
	
	\draw [fill=olive] (v0) circle (.4ex);
	\draw [fill=olive] (v40) circle (.4ex); 
	\draw [fill=olive] (v90) circle (.4ex);
	\draw [fill=olive] (v180) circle (.4ex); 
	\draw [fill=olive] (v130) circle (.4ex); 
	\draw [fill=olive] (v270) circle (.4ex); 
\end{scope}
}
\newcommand\FourCoverWheelSigmaPeripheral[1]{
\begin{tikzpicture}[scale=#1,gray]
	\MacroFourCover
			
	\draw[thick,gray] (v0) .. controls (1.5,0.3)  and (1.4,0.8) .. (v40); %node[pos=0.5,right]{$\b$};
	\draw[thick,gray] (v40)  .. controls (0.6,1.3)  and (0.4,1.4) .. (v90); % node[pos=0.5,right]{$\a$};
	
	\begin{scope}[rotate=90]
		\path (0:2) coordinate (u0);
		\path (90:2) coordinate (u90);
		\path (40:2) coordinate (u40);
		\draw[thick] (u0) .. controls (1.5,0.3)  and (1.4,0.8) .. (u40);
		\draw[thick] (u40)  .. controls (0.6,1.3)  and (0.4,1.4) .. (u90);
	\end{scope}
	
	\draw[thick] (origin) -- (v0) node[pos=0.6,below]{$1$};
	\draw[thick] (origin) -- (v180) node[pos=0.6,below]{$1$};
	\draw[thick] (origin) -- (v90) node[pos=0.4, left]{$1$};
	\draw[thick] (origin) -- (270:2) node[pos=0.6, left]{$1$};
	
	\draw[thick] (origin) -- (v40) node[pos=0.6,below]{$2$};
	\draw[thick] (origin) -- (v130) node[pos=0.6,below]{$2$};
	
	\draw [fill=olive] (150:2) circle (.4ex); 
	\draw [fill=olive] (v70) circle (.4ex); 

	\draw [line width = 1.5pt, dashed, red, rounded corners = 20pt] (v70) -- (92:0.8) -- (150:2);
	\draw (150:2) node[left=3pt, red] {\Large{$\tilsi$}}; 
	
	\draw [line width = 1.5pt, dashed, cyan, rounded corners = 20pt] (v70) -- (90:1) -- (u40);
	\draw (130:2) node[left, above=3pt, cyan] {\Large{$\tilbe$}}; 

\end{tikzpicture}
}  
\newcommand\FourCoverWheel[1]{
\begin{tikzpicture}[scale=#1,gray]
	\MacroFourCover
		
	\draw[thick] (v0) .. controls (1.5,0.3)  and (1.4,0.8) .. (v40);% node[pos=0.5,right]{$\b$};
	\draw[thick] (v40)  .. controls (0.6,1.3)  and (0.4,1.4) .. (v90);% node[pos=0.5,right]{$\a$};
	
	\begin{scope}[rotate=90]
		\path (0:2) coordinate (u0);
		\path (90:2) coordinate (u90);
		\path (40:2) coordinate (u40);
		\draw[thick] (u0) .. controls (1.5,0.3)  and (1.4,0.8) .. (u40);
		\draw[thick] (u40)  .. controls (0.6,1.3)  and (0.4,1.4) .. (u90);
	\end{scope}
	
	\draw[thick] (origin) -- (v0) node[pos=0.6,below]{$1$};
	\draw[thick] (origin) -- (v180) node[pos=0.6,below]{$1$};
	\draw[thick] (origin) -- (v90) node[pos=0.4, left]{$1$};
	\draw[thick] (origin) -- (270:2) node[pos=0.6, left]{$1$};
	
	\draw[thick] (origin) -- (v40) node[pos=0.6,below]{$2$};
	\draw[thick] (origin) -- (v130) node[pos=0.6,below]{$2$};
	
	\draw [fill=olive] (v70) circle (.4ex); 
	\draw[line width=2 pt, dashed, red] (origin) -- (v70); \draw (v70) node[right, above=3pt, red] {\Large{$\tilsi$}}; 
	
	\draw [fill=olive] (v160) circle (.4ex); 
	\draw[line width=2 pt, dashed, red] (origin) -- (v160); \draw (v160) node[left=5pt, above=3pt, red] {\Large{$\tilsi$'}}; 
	
	\draw [line width = 1.5pt, dashed, cyan, rounded corners = 20pt] (v70) -- (90:1) -- (u40);
	\draw (130:2) node[left, above=3pt, cyan] {\Large{$\tilbe$}}; 
\end{tikzpicture}
  }
\newcommand\FourCoverSelfFolded[1]{
\begin{tikzpicture}[scale=#1,gray]
	\MacroFourCover
		
	\draw[thick] (v0) .. controls (1.5,0.3)  and (1.4,0.8) .. (v40) node[pos=0.5,right]{\tiny $\b$};
	\draw[thick] (v40)  .. controls (0.6,1.3)  and (0.4,1.4) .. (v90) node[pos=0.5,right]{$\a$};
	
	\begin{scope}[rotate=90]
		\path (0:2) coordinate (u0);
		\path (90:2) coordinate (u90);
		\path (40:2) coordinate (u40);
		\draw[thick] (u0) .. controls (1.5,0.3)  and (1.4,0.8) .. (u40) node[pos=0.5,left]{\tiny $\b$};
		\draw[thick] (u40)  .. controls (0.6,1.3)  and (0.4,1.4) .. (u90) node[pos=0.3]{$\a$};
	\end{scope}
	
	\draw[thick] (origin) -- (v0) node[pos=0.4,below]{$r$};
	\draw[thick] (origin) -- (v180) node[pos=0.4,below]{$r$};
	\draw[thick] (origin) -- (v90) node[pos=0.4, left]{$r$};
	\draw[thick] (origin) -- (270:2) node[pos=0.6, left]{$r$};
	
	\draw[thick] (v0) .. controls (0.5,0.4)  and (0.1,1.2) .. (v90) node[pos=0.3,right]{$\ell$};
	\draw[thick] (v180)  .. controls (-0.5,0.4)  and (-0.1,1.2) .. (v90) node[pos=0.3,right]{$\ell$};
	
	\draw [fill=olive] (v70) circle (.4ex); 
	\draw[line width=2 pt, dashed, red] (origin) -- (v70); \draw (v70) node[right, above=3pt, red] {\Large{$\tilsi$}}; 
	
	\draw [fill=olive] (v160) circle (.4ex); 
	\draw[line width=2 pt, dashed, red] (origin) -- (v160); \draw (v160) node[left=5pt, above=3pt, red] {\Large{$\tilsi$'}}; 
	
	\draw [line width = 1.5pt, dashed, cyan, rounded corners = 20pt] (v70) -- (90:1) -- (u40);
	\draw (130:2) node[left, above=3pt, cyan] {\Large{$\tilbe$}}; 
	
	\draw (v0) node[right,olive] {$v$}; %{$x$}; 
	\draw (v90) node[above=2pt,olive] {$v$}; %{$x$}; 
	\draw (v180) node[left,olive] {$v$}; %{$x$}; 
	\draw (270:2) node[below=2pt,olive] {$v$}; %{$x$}; 
	
	\draw (v40) node[right, olive] {$y$}; \draw (v130) node[left, olive] {$y$};
\end{tikzpicture}
  }
  %%%
  %%%


\newcommand\MacroMiniLemmaPeripheralPolygon[2]{
\begin{scope}[scale={#1}, {#2},gray]
\coordinate(theta1A) at (-4,-2); 
\coordinate(theta1B) at (4,-2); 
\coordinate(theta2A) at (-4,-1); \coordinate(theta3A) at (-4,0);  \coordinate(theta4A) at (-4,1); 
\coordinate(theta2B) at (4,-1); \coordinate(theta3B) at (4,0);  \coordinate(theta4B) at (4,1); 
\coordinate(thetahA) at (-4,2);
\coordinate(thetahB) at (4,2);


\draw (theta1A) -- (theta1B) node[pos=0.2, above,black] {$\taui{d}$}; %{$\theta_1$};
\draw (thetahA) -- (thetahB); %node[pos=0.7, above] {$\theta_h$};
\draw (theta1A) -- (thetahA) -- (thetahB) -- (theta1B) -- (theta1A);

\draw[style=dashed] (theta2A) -- (theta1B); 
\draw[style=dashed] (theta3A) -- (theta1B);
\draw[style=dashed] (theta3A) -- (theta3B);
\draw[style=dashed] (theta4A) -- (theta3B);
\draw[style=dashed] (theta4A) -- (thetahB);


\draw [fill=olive] (theta1A) circle (.6ex) node[left] {$t$}; %{$\theta_1^A$};
\draw [fill=olive] (theta1B) circle (.6ex); %node[right=2pt] {$\theta_1^B$};

\draw [fill=olive] (theta2A) circle (.6ex); %node[left=2pt] {$\theta_2^A$};
\draw [fill=olive] (theta4A) circle (.6ex); %node[left=2pt, left=2pt] {$\theta_{h-1}^A$};
\draw [fill=olive] (theta3B) circle (.6ex); %node[right=2pt] {$\theta_{h-1}^B$};

\end{scope}
}

\newcommand \MacroMiniLemmaWheelPolygon[2]{
\begin{scope}[scale={#1}, {#2},gray]
\coordinate(rho1A) at (-4,-2); \coordinate(rho2A) at (-4,0); \coordinate(rho3A) at (-3,2);  
\coordinate(rho4A) at (0,2.5); \coordinate(rho5A) at (3,2); 
\coordinate(P) at (0,0); \coordinate(rhof1A) at (4,0); \coordinate(rhofA) at (4,-2);
\draw (P) -- (rho1A) node[pos=0.5, above,black] {$\rho_1$};
\draw (P) -- (rho2A) node[pos=0.6, above,black] {$\rho_2$};
\draw (P) -- (rho3A) node[pos=0.4, above,black] {$\rho_3$};
\draw[style=dashed] (P) -- (rho4A); \draw[style=dashed] (P) -- (rho5A);
\draw (P) -- (rhof1A) node[pos=0.6, above,black] {$\rho_{f-1}$};
\draw (P) -- (rhofA) node[pos=0.6, above,black] {$\rho_{f}$};
\draw (rho1A) -- (rho2A) -- (rho3A) -- (rho4A) -- (rho5A) -- (rhof1A) -- (rhofA);

\draw[fill=violet] (P) circle(1 ex);
\draw [fill=olive] (rho2A) circle (.6ex);
\draw [fill=olive] (rho3A) circle (.6ex);
\draw [fill=olive] (rhof1A) circle (.6ex);
\end{scope}
}

\newcommand\MacroTechnicalLemma{
\MacroMiniLemmaWheelPolygon{1}{NonOrientedTriangulationStyle}
	\draw [fill=olive] (rho1A) circle (.4ex);% node[left=2pt] {$\rho_1^A=\theta_h^A$};
	\draw [fill=olive] (rhofA) circle (.4ex);% node[right=2pt] {$\rho_{f}^A=\theta_h^B$};
	\draw (1,-2) node[above,black] {$\taui{1}$};
	\begin{scope}[yshift=-4cm]
		\MacroMiniLemmaPeripheralPolygon{1}{NonOrientedTriangulationStyle}
	\end{scope}
}

\newcommand\MacroTechnicalLemmaSelfFolded{
       %  \coordinate(rho1A) at (-4,-2); 
        % \coordinate(P) at (0,0); 
         \begin{scope}[gray]
         \coordinate(rho1A) at (-4,-2); \coordinate(rho2A) at (-4,0); \coordinate(rho3A) at (-3,2);  
	\coordinate(rho4A) at (0,2.5); \coordinate(rho5A) at (3,2); 
	\coordinate(P) at (0,0); \coordinate(rhof1A) at (4,0); \coordinate(rhofA) at (4,-2);
         
	\draw (P) -- (rho1A) node[pos=0.5, above,black] {$\rho_1=r$};
	\draw[NonOrientedTriangulationStyle, rounded corners = 10pt] (rho1A) -- (rho2A) -- (rho3A) -- (rho4A) -- (rho5A) -- (rhof1A) -- (rhofA);
	\draw[fill=violet] (P) circle(1 ex); %node[below, violet] {\tiny $P$};
	\draw (2,-2) node[above,black] {$\taui{2}$};
         
	% draw loop around rho 1
	\coordinate (startL1) at (rho1A);
	\draw[NonOrientedTriangulationStyle] (rho1A) 
	.. controls ($(P)+(3,6)$) and ($(P)+(3,-2)$) .. (rho1A) node[pos=0.5, above=3pt,black] {$\ell=\taui{1}$};
	\draw [fill=olive] (rho1A) circle (.6ex);
	\draw [fill=olive] (rhofA) circle (.6ex);
	\begin{scope}[yshift=-4cm]
		\MacroMiniLemmaPeripheralPolygon{1}{NonOrientedTriangulationStyle}
	\end{scope}
	\end{scope}
}

\newcommand\TikzMinilemmaRadiusGammaPolygonRhoOneLoop[1]{
\begin{tikzpicture}[scale=#1]
	\MacroTechnicalLemma
	% draw loop around rho 1
	\coordinate (startL1) at (rho1A);
	\draw[ inducedmapTPathonCnStyle, %GammaonCnStyle,
	postaction={decorate} ] (rho1A) 
	.. controls ($(P)+(3,6)$) and ($(P)+(3,-2)$) .. (rho1A) node[pos=0.5, above=5pt] {$\ell_1$};
\end{tikzpicture}
}

\newcommand\TikzTechnicalLemmaUnnotchedRadius[1]{
\begin{tikzpicture}[scale=#1]
	\MacroTechnicalLemma
	\draw[line width=2 pt, dashed, red,PathArrowAtPosStyleSeven] (P) -- (theta1A) node[pos=0.5,right=2pt, red] {\Large{$\si$}}; 
\end{tikzpicture}
}

\newcommand\TikzTechnicalLemmaUnnotchedRadiusSelfFolded[1]{
\begin{tikzpicture}[scale=#1]
	\MacroTechnicalLemmaSelfFolded
	\draw[line width=2 pt, dashed, red,PathArrowAtPosStyleSeven] (P) -- (theta1A) node[pos=0.5,right=2pt, red] {\Large{$\si$}}; 
\end{tikzpicture}
}

\newcommand\TikzTechnicalLemmaNotchedRadiusSelfFolded[1]{
\begin{tikzpicture}[scale=#1]
	\MacroTechnicalLemmaSelfFolded
	\draw[line width=2 pt, dashed, red] (P) -- (theta1A) node[pos=0.6,right=2pt, red] {\Large{$\si$}}; 
	\begin{scope}[yshift=-0.5cm,xshift=-0.2cm]
	\draw (-0.8,-0.4) node[red] {\Large{$\blacktriangleleft$}};
	\draw (-0.3,-0.9) node[red] {\Large{$\blacktriangleright$}};
	\end{scope}
\end{tikzpicture}
}

\newcommand\TikzTechnicalLemmaNotchedRadius[1]{
\begin{tikzpicture}[scale=#1]
	\MacroTechnicalLemma
	\draw[line width=2 pt, dashed, red] (P) -- (theta1A) node[pos=0.5,right=2pt, red] {\Large{$\si$}}; 
	%\draw (P) node[red, left=9pt,below=8pt] {\Huge{$\filledbowtie$}};
	\draw (-0.8,-0.4) node[red] {\Large{$\blacktriangleleft$}};
	\draw (-0.3,-0.9) node[red] {\Large{$\blacktriangleright$}};
\end{tikzpicture}
}

\newcommand\TikzTechnicalLemmaRhoOne[1]{
\begin{tikzpicture}[scale=#1]
	\MacroTechnicalLemma
	\draw[line width=2 pt, dashed, red] (P) -- (rho1A) node[pos=0.75,above=5pt, red] {\Large{$\si=$}}; 
	\draw (-1,0) node[red] {\Large{$\blacktriangleleft$}};
	\draw (-0.4,-0.6) node[red] {\Large{$\blacktriangleright$}};
\end{tikzpicture}
}

%%%%%% MAKE TRIANGULATED POLYGON %%%%%%%%%%%
%%%%%%%%%%%%%%%%%%%%%%%% and snake graph

\newcommand \tilTtwoIdealTriangulationCoordinates {
	\coordinate (P) at (0,4);
	\coordinate (rightd) at (0.2,0); \coordinate (leftd) at (-0.2,0); 
	\coordinate (downd) at (0,-0.2); \coordinate (upd) at (0,0.2); 
	\coordinate (x1) at (0,0); \coordinate (x2) at (-4,4); \coordinate (x3) at (4,4); 
	\coordinate (t1) at (-4,0); \coordinate (s) at (-4,-4); \coordinate (z) at (0,-4);  \coordinate (t2) at (4,0);
}  

\newcommand \tilTtwoIdealTriangulationVertices {
\tilTtwoIdealTriangulationCoordinates
\draw [fill=violet] (P) circle (1ex); % the six vertices to make the top rectangular
\draw [fill=olive] (x2) circle (.6ex); % node[above=5pt,olive] {$\tilx^2$};
\draw [fill=olive] (x3) circle (.6ex); % node[above=5pt,olive] {$\tilx^3$};
\draw [fill=olive] (t1) circle (.6ex); % node[left=5pt,olive] {$\tilt^1$};
\draw [fill=olive] (t2) circle (.6ex); % node[below=5pt,olive] {$\tilt^2$};
\draw [fill=olive] (x1) circle (.6ex); % node[right=18pt, above=1pt,olive] {$\tilx^1$};

\draw [fill=olive] (s) circle (.6ex); % node[left=5pt,olive] {$\tils$}; % the two bottom vertices 
\draw [fill=olive] (z) circle (.6ex); % node[right=5pt,olive] {$\tilz$};
}

\newcommand\TtwoTriangulatedPolygonPlainer[1]{
\begin{tikzpicture}[scale={#1},NonOrientedTriangulationStyle]
\begin{scope}[rotate=-90]
	\tilTtwoIdealTriangulationCoordinates

	\draw[style = thick, NooseArrowAtPosStyleSix] 
	(x1) -- (-4,4) node[pos = 0.6, left=2pt]{$\ell$};
	\draw[style = thick, postaction={decorate}] 
	(-4,4) -- (0,4) node[pos=0.5, right]
	{$\dot{r}$};  %{$\tiltg{3}=\dot{r}$};
	\draw[style = thick, postaction={decorate}] 
	(0,0) -- (0,4) node[pos=0.4, above]
	{$r$}; %{$\tiltaui{4}= r$}; 
	\draw[style = thick, NooseArrowAtPosStyleFive] 
	(4,4) -- (0,0) node[pos = 0.3, right=2pt]{$\ell$};
	\draw[style = thick, postaction={decorate}] 
	(4,4) -- (0,4) node[pos=0.5, right]
	{$\ddot{r}$}; %{$\tiltg{4}= \ddot{r}$};

	% The rest of the top rectangle
	\draw [postaction={decorate}] (x2) -- (t1) node[pos=0.3, above] {$b_3$};
	\draw [postaction={decorate}] (t1) -- (x1) node[pos=0.3, left] {$2$};
	\draw [postaction={decorate}] (x1) -- (t2) node[pos=0.6, left] {$b_3$}; %{$\tiltg{6}=b_3$};
	\draw [postaction={decorate}] (t2) -- (x3) node[pos=0.5, below] {$2$}; %{$\tiltg{5} = 2$};
	
	\draw [postaction={decorate}] (t1) -- (s) node[pos=0.5, above] {$b_4$}; %{$\tiltg{-1}=b_4$};
	\draw [postaction={decorate}] (s) -- (z) node[pos=0.3, left] {$b_1$};%{$\tiltg{0}=b_1$};
	\draw [postaction={decorate}] (z) -- (x1) node[pos=0.5, below] {$b_2$}; % {$\tiltg{1}=b_2$};
	\draw [postaction={decorate}] (t1) -- (z) node[pos=0.3, left] {$1$};
	
	\begin{scope}[gray!65]
	\draw (-3,-3) node {\large$\tilblacktri{0}$};
	\draw (-1,-1) node {\large$\tilblacktri{1}$};
	\draw (-3,1) node {\large$\tilblacktri{2}$};
	\draw (-1,3) node {\large$\tilblacktri{3}$};
	\draw (1,3) node {\large$\tilblacktri{4}$};
	\draw (3,1) node {\large$\tilblacktri{5}$};
	\end{scope}
	
	\draw
	[red!60, line width = 2.5pt, ->, style=dashed, rounded corners = 10pt,
	decoration={ markings, mark=at position .6 with {\arrow[red!80, line width=2.5pt]{to};}}]
	(s) -- (-2,-2) %-- (-2,0)
	 -- (-2,2) 
	-- (-0,2) -- (2,2) node[pos=0.3,left] {\Large$\tilga$}  -- ($(t2)+(-0.1,0.1)$);

	\tilTtwoIdealTriangulationVertices
\end{scope}
	\end{tikzpicture}
}


\newcommand \TtwoTriangulatedPolygon[1]{
\begin{tikzpicture}[scale={#1},NonOrientedTriangulationStyle]
\begin{scope}[rotate=-90]
	\tilTtwoIdealTriangulationCoordinates

	\draw[style = thick, NooseArrowAtPosStyleSix] 
	(x1) -- (-4,4) node[pos = 0.6, left=2pt]{$\ell$};
	\draw[style = thick, postaction={decorate}] 
	(-4,4) -- (0,4) node[pos=0.5, right]
	{$\tiltg{3}=\dot{r}$};
	\draw[style = thick, postaction={decorate}] 
	(0,0) -- (0,4) node[pos=0.4, above]
	{$\tiltaui{4}= r$}; 
	\draw[style = thick, NooseArrowAtPosStyleFive] 
	(4,4) -- (0,0) node[pos = 0.3, right=2pt]{$\ell$};
	\draw[style = thick, postaction={decorate}] 
	(4,4) -- (0,4) node[pos=0.5, right]
	{$\tiltg{4}= \ddot{r}$};

	% The rest of the top rectangle
	\draw [postaction={decorate}] (x2) -- (t1) node[pos=0.3, above] {$b_3$};
	\draw [postaction={decorate}] (t1) -- (x1) node[pos=0.3, left] {$2$};
	\draw [postaction={decorate}] (x1) -- (t2) node[pos=0.6, left] {$\tiltg{6}=b_3$};
	\draw [postaction={decorate}] (t2) -- (x3) node[pos=0.5, below] {$\tiltg{5} = 2$};
	
	\draw [postaction={decorate}] (t1) -- (s) node[pos=0.5, above] {$\tiltg{-1}=b_4$};
	\draw [postaction={decorate}] (s) -- (z) node[pos=0.3, left] {$\tiltg{0}=b_1$};
	\draw [postaction={decorate}] (z) -- (x1) node[pos=0.5, below] {$\tiltg{1}=b_2$};
	\draw [postaction={decorate}] (t1) -- (z) node[pos=0.3, left] {$1$};
	
	\begin{scope}[gray!65]
	%\draw (-3,-3) node {\large$\tilblacktri{0}$};
	%\draw (-1,-1) node {\large$\tilblacktri{1}$};
	\draw (-3,1) node {\large$\tilblacktri{2}$};
	%\draw (-1,3) node {\large$\tilblacktri{3}$};
	%\draw (1,3) node {\large$\tilblacktri{4}$};
	%\draw (3,1) node {\large$\tilblacktri{5}$};
	\end{scope}
	
	\draw
	[red!60, line width = 2.5pt, ->, style=dashed, rounded corners = 10pt,
	decoration={ markings, mark=at position .6 with {\arrow[red!80, line width=2.5pt]{to};}}]
	(s) -- (-2,-2) %-- (-2,0)
	 -- (-2,2) 
	-- (-0,2) -- (2,2) node[pos=0.3,left] {\Large$\tilga$}  -- ($(t2)+(-0.1,0.1)$);

	\tilTtwoIdealTriangulationVertices
	\end{scope}
	\node at (10,0) {};
	\node at (-10,0) {};
	\node[gray!65] at (-3.2,2.2) {\large$\tilblacktri{0}$};
	\node[gray!65] at (3.1,1.6) {\large$\tilblacktri{3}$};
	\node[gray!65] at (-1.4,1) {\large$\tilblacktri{1}$};
	\node[gray!65] at (3.1,-1.2) {\large$\tilblacktri{4}$};
	\node[gray!65] at (2.1,-3.2) {\large$\tilblacktri{5}$};
	\end{tikzpicture}
}
\newcommand \tilTtwoSnakeGraphCoordinates {
	\coordinate (P) at (8,0);
	\coordinate (rightd) at (0.2,0); \coordinate (leftd) at (-0.2,0); 
	\coordinate (downd) at (0,-0.2); \coordinate (upd) at (0,0.2); 
	\coordinate (x1a) at (0,0); \coordinate (x2) at (4,0);  
	\coordinate (x1b) at (4,4); \coordinate (x1c) at (8,-4);
	\coordinate (x3a) at (8,4);  \coordinate (x3b) at (12,0); 
	\coordinate (t1a) at (-4,0);  \coordinate (t1b) at (4,-4); 
	\coordinate (s) at (-4,-4); \coordinate (z) at (0,-4);  
	\coordinate (t2) at (12,4);
}  

\newcommand \tilTtwoSnakeGraphVertices {
\tilTtwoSnakeGraphCoordinates
\draw [fill=violet] (P) circle (1ex); % the six vertices to make the top rectangular
\draw [fill=olive] (x2) circle (.6ex); % node[above=5pt, left=2pt,olive] {$\tilx^2$};
\draw [fill=olive] (x3a) circle (.6ex); % node[above=5pt,olive] {$\tilx^3_a$};
\draw [fill=olive] (x3b) circle (.6ex); % node[below=5pt,olive] {$\tilx^3_b$};
\draw [fill=olive] (t1a) circle (.6ex); % node[above=5pt,olive] {$\tilt^1_a$};
\draw [fill=olive] (t1b) circle (.6ex); % node[below=5pt,olive] {$\tilt^1_b$};
\draw [fill=olive] (t2) circle (.6ex); % node[right=5pt,olive] {$\tilt^2$};
\draw [fill=olive] (x1a) circle (.6ex); % node[right=2pt, above=1pt,olive] {$\tilx^1_a$};
\draw [fill=olive] (x1b) circle (.6ex); % node[left=2pt, above=1pt,olive] {$\tilx^1_b$};
\draw [fill=olive] (x1c) circle (.6ex); % node[right=2pt, right=1pt,olive] {$\tilx^1_c$};
\draw [fill=olive] (s) circle (.6ex); % node[left=5pt,olive] {$\tils$}; % the two bottom vertices 
\draw [fill=olive] (z) circle (.6ex); %node[below=5pt,olive] {$\tilz$};
}

\tikzstyle{orientedAtEllTriangulationStyle}=[style=thick,
decoration={ markings, mark=at position .6 with {\arrow{to};}} ]
\newcommand\TtwoSnakeGraphPlainer[1]{
\begin{tikzpicture}[scale={#1},orientedAtEllTriangulationStyle,gray!80]
	\tilTtwoSnakeGraphCoordinates
	% fifth tile
	\draw [postaction={decorate}] (x3a) -- (x3b) node[black,pos=0.4, below] {$\ell$}; % diagonal
	\draw (t2) -- (x3a) node[black,pos=0.4, above] {$2$}; % node[pos=0.4, above] {$2^b$};
	\draw (x3b) -- (t2) node[black,pos=0.5, right] {$b_3$}; % node[pos=0.4, right] {$b_3^b$};
	\draw (x3b) -- (P) node[black,pos=0.5, below] {$r$}; % node[pos=0.3, below] {$r^b$};
	

	% the fourth tile (negative)
	\draw (x1b) -- (P) node[black,pos=0.4, below] {$r$}; % diagonal
	\draw [postaction={decorate}] (x1b) -- (x2) node[black,pos=0.5, left] {$\ell$}; % node[pos=0.3, left] {$\ell^b$};
	\draw [postaction={decorate}] (x3a) -- (x1b) node[black,pos=0.4, above] {$\ell$}; % node[pos=0.3, above] {$\ell^b$};
	\draw (x3a) -- (P) node[black,pos=0.5] {\large $\ddot{r}$}; % not a border of snake graph
	
	% the third tile (positive)
	\draw [postaction={decorate}] (x1c) -- (x2) node[black,pos=0.5, below] {$\ell$}; % diagonal
	\draw (t1b) -- (x1c) node[black,pos=0.3, below] {$2$}; % node[pos=0.3, below] {$2^b$}; 
	\draw (x1c) -- (P) node[black,pos=0.5, right] {$r$}; % node[pos=0.3, right] {$r^b$};
	\draw (x2) -- (P) node[black,pos=0.5] {\large $\dot{r}$}; % not a border of snake graph

	% The second tile (negative)
	\draw (t1b) -- (x1a) node[black,pos=0.5] {$2$}; % diagonal
	\draw (x2) -- (t1b) node[black,pos=0.5] {$b_3$}; % not a border of snake graph 
	\draw (t1b) -- (z) node[black,pos=0.5, below] {$1$}; % node[pos=0.3, below] {$1^b$};
	\draw [postaction={decorate}] (x1a) -- (x2) node[black,pos=0.5, above] {$\ell$}; % boundary
	
	% first tile (positive)
	\draw (t1a) -- (z) node[black,pos=0.5, below] {$1$}; % diagonal
	\draw (t1a) -- (s) node[black,pos=0.4, left] {$b_4$}; %{$\tiltg{-1}=b_4$}; % node[pos=0.4, left] {$\tiltg{-1}^b=b_4^b$};
	\draw (s) -- (z) node[black,pos=0.5, below] {$b_1$};%{$\tiltg{0}=b_1$}; % node[pos=0.4, below=5pt] {$\tiltg{0}^b=b_1^b$};
	\draw (z) -- (x1a) node[black,pos=0.6] {$b_2$};%{$\tiltg{1}=b_2$}; % not a border of snake graph
	\draw (t1a) -- (x1a) node[black,pos=0.5, above] {$2$}; % node[pos=0.3, above] {$2^b$};
	
	\begin{scope}[gray]
	\draw (-3,-3) node {\Large$\tilblacktri{0}$};
	\draw (-1.4,-0.8) node {\Large$\tilblacktri{1}$};
	\draw (1.4,-3) node {$\tilblacktri{1}^{(-)}$};
	\draw (2.5,-1) node {$\tilblacktri{2}^{(-)}$};
	\draw (5,-3) node {\Large$\tilblacktri{2}$};
	\draw (6.5,-1) node {\Large$\tilblacktri{3}$};
	\draw (5.5,1) node {$\tilblacktri{3}^{(-)}$}; 
	\draw (7,3) node {$\tilblacktri{4}^{(-)}$}; %good
	\draw (9,1) node {\Large$\tilblacktri{4}$}; %good
	\draw (11,3) node {\Large$\tilblacktri{5}$}; 
	\end{scope}

	\tilTtwoSnakeGraphVertices
	%\node at (-6,0) {};
	\end{tikzpicture}
}

\newcommand\MacroTtwoPerfectMatchingsWithDiagonals{

	\tilTtwoSnakeGraphCoordinates
	\MacroTtwoPerfectMatchings
	% fifth tile
	\begin{scope}[line width=2pt, dashed]
	\draw (x3a) -- (x3b) node[pos=0.4, right] {$\ell$}; % diagonal
	
	% the fourth tile (negative)
	\draw [postaction={decorate}] (x1b) -- (P) node[pos=0.5, below] {$r$};

	% the third tile (positive)
	\draw [postaction={decorate}] (x1c) -- (x2) node[pos=0.6, right] {$\ell$};
	

	% The second tile (negative)
	\draw [postaction={decorate}] (t1b) -- (x1a) node[pos=0.5, right] {$2$};
	
	% first tile (positive)
	\draw [postaction={decorate}] (t1a) -- (z) node[pos=0.4, right] {$1$};
	\end{scope}
}



\newcommand\MacroTtwoPerfectMatchings{

	\tilTtwoSnakeGraphCoordinates
	% fifth tile
	%\draw (x3a) -- (x3b) node[pos=0.4, below] {$\ell$}; % diagonal
	\draw (t2) -- (x3a) ;%node[pos=0.4, above]; %{$2$};
	\draw (x3b) -- (t2) ;%node[pos=0.4, right];% {$b_3$};
	\draw (x3b) -- (P) ;%node[pos=0.3, below];% {$r$};
	
	% the fourth tile (negative)
	%\draw [postaction={decorate}] (x1b) -- (P) node[pos=0.3, below] {$r$};
	\draw [fill=gray, fill opacity=0.3] (x3a)  -- (x1b) -- (x2) -- (P) -- cycle;
	\draw (x1b) -- (x2);% node[pos=0.3, left];% {$\ell$};
	\draw (x3a) -- (x1b);% node[pos=0.3, above];% {$\ell$};
	\draw  (x3a) -- (P);% node[pos=0.3, right];% {$\ddot{r}$};
	
	% the third tile (positive)
	%\draw [postaction={decorate}] (x1c) -- (x2) node[pos=0.3, below] {$\ell$};
	\draw (t1b) -- (x1c);% node[pos=0.3, below]; %{$2$};
	\draw (x1c) -- (P) ;%node[pos=0.3, right];% {$r$};
	\draw (x2) -- (P);% node[pos=0.3, below];% {$\dot{r}$};

	% The second tile (negative)
	%\draw [postaction={decorate}] (t1b) -- (x1a) node[pos=0.3, below] {$2$};
	\draw(x2) -- (t1b) ;%node[pos=0.5, left];% {$b_3$};
	\draw (t1b) -- (z) node[white,pos=0.1, below] {$1$};%node[pos=0.3, below];% {$1$};
	\draw (x1a) -- (x2) ;%node[pos=0.6, above]% {$\ell$};
	
	% first tile (positive)
	%\draw [postaction={decorate}] (t1a) -- (z) node[pos=0.4, below] {$1$};
	\draw  (t1a) -- (s);% node[pos=0.4, left] ;%{$\tiltg{-1}=b_4$};
	\draw (s) -- (z);% node[pos=0.4, below=5pt];% {$\tiltg{0}=b_1$};
	\draw (z) -- (x1a);% node[pos=0.6];% {$\tiltg{1}=b_2$};
	\draw (t1a) -- (x1a); %node[pos=0.3, above]; {$2$};
	
	%\tilTtwoSnakeGraphVertices
	\node at (t1b) {};
	\node at (x1b) {};% node[pos=0.3, above];% {$\ell$};
}


\newcommand\TtwoPerfectMatchingsOne[1]{
\begin{tikzpicture}[scale={#1}]
	\MacroTtwoPerfectMatchings
	\begin{scope}[line width=4pt]
	\draw  (t1a) -- (s) node[pos=0.4, left] {$b_4$}; 
	\draw (z) -- (x1a) node[pos=0.6, left=2pt] {$b_2$};
	\draw(x2) -- (t1b) node[pos=0.5, left] {$b_3$};
	\draw (x1c) -- (P) node[pos=0.3, right] {$r$};
	\draw (x3a) -- (x1b) node[pos=0.3, below] {$\ell$};
	\draw (x3b) -- (t2) node[pos=0.4, right] {$b_3$};
	%\draw (-4,4) node {($1$)};
	\end{scope}
\end{tikzpicture}
}


\newcommand\TtwoPerfectMatchingsThree[1]{
\begin{tikzpicture}[scale={#1}]
	\MacroTtwoPerfectMatchings
	\begin{scope}[line width=4pt]
	 \draw  (t1a) -- (x1a) node[pos=0.3, above] {$2$};
	\draw (s) -- (z) node[pos=0.4, below] {$b_1$};
	\draw (t1b) -- (x1c) node[pos=0.3, below] {$2$};
	\draw (x1b) -- (x2) node[pos=0.3, left] {$\ell$};
	\draw (t2) -- (x3a) node[pos=0.4, below] {$2$};
	\draw (x3b) -- (P) node[pos=0.3, below] {$r$};
	%\draw (-4,4) node {($3$)};
	\end{scope}
\end{tikzpicture}
}

\newcommand\TtwoPerfectMatchingsFour[1]{
\begin{tikzpicture}[scale={#1}]
	\MacroTtwoPerfectMatchings
	\begin{scope}[line width=4pt]
	 \draw (t1a) -- (x1a) node[pos=0.3, above] {$2$};
	\draw (s) -- (z) node[pos=0.4, above=2pt] {$b_1$};
	\draw(x2) -- (t1b) node[pos=0.5, left] {$b_3$};
	\draw (x1c) -- (P) node[pos=0.3, right] {$r$};
	\draw (x3a) -- (x1b) node[pos=0.3, below] {$\ell$};
	\draw (x3b) -- (t2) node[pos=0.4, right] {$b_3$};
	%\draw (-4,4) node {($4$)};
	\end{scope}
\end{tikzpicture}
}

\newcommand\TtwoPerfectMatchingsFive[1]{
\begin{tikzpicture}[scale={#1}]
	\MacroTtwoPerfectMatchings
	\begin{scope}[line width=4pt]
	\draw  (t1a) -- (s) node[pos=0.4, left=2pt] {$b_4$}; 
	\draw (z) -- (x1a) node[pos=0.6, left=2pt] {$b_2$};
	\draw (t1b) -- (x1c) node[pos=0.3, below] {$2$};
	\draw (x1b) -- (x2) node[pos=0.3, left] {$\ell$};
	\draw (t2) -- (x3a) node[pos=0.4, above] {$2$};
	\draw (x3b) -- (P) node[pos=0.3, below] {$r$};
	%\draw (-4,4) node {($5$)};
	\end{scope}
\end{tikzpicture}
}

\newcommand\TtwoPerfectMatchingsTwo[1]{
\begin{tikzpicture}[scale={#1}]
	\MacroTtwoPerfectMatchings
	\begin{scope}[line width=4pt]
	\draw  (t1a) -- (s) node[pos=0.4, left=2pt] {$b_4$}; 
	\draw (t1b) -- (z) node[pos=0.3, below] {$1$};
	\draw (x1a) -- (x2) node[pos=0.6, above] {$\ell$};
	\draw (x1c) -- (P) node[pos=0.3, right] {$r$};
	\draw (x3a) -- (x1b) node[pos=0.3, above] {$\ell$};
	\draw (x3b) -- (t2) node[pos=0.4, right=2pt] {$b_3$};
	%\draw (-4,4) node {($2$)};
	\end{scope}
\end{tikzpicture}
}

\newcommand\TtwoPerfectMatchingsSix[1]{
\begin{tikzpicture}[scale={#1}]
	\MacroTtwoPerfectMatchings
	\begin{scope}[line width=4pt]
	\draw  (t1a) -- (s) node[pos=0.4, left=2pt] {$b_4$}; 
	\draw (z) -- (x1a) node[pos=0.6, left=2pt] {$b_2$};
	\draw (t1b) -- (x1c) node[pos=0.3, above] {$2$};
	\draw (x1b) -- (x2) node[pos=0.3, left=2pt] {$\ell$};
	\draw  (x3a) -- (P) node[pos=0.3, right=2pt] {$\ddot{r}$};
	\draw (x3b) -- (t2) node[pos=0.4, right=2pt] {$b_3$};
	%\draw (-4,4) node {($6$)};
	\end{scope}
\end{tikzpicture}
}

\newcommand\TtwoPerfectMatchingsSeven[1]{
\begin{tikzpicture}[scale={#1}]
	\MacroTtwoPerfectMatchings
	\begin{scope}[line width=4pt]
	\draw  (t1a) -- (s) node[pos=0.4, left=2pt] {$b_4$}; 
	\draw (z) -- (x1a) node[pos=0.6, left=2pt] {$b_2$};
	\draw (t1b) -- (x1c) node[pos=0.3, below] {$2$};
	\draw (x2) -- (P) node[pos=0.3, below] {$\dot{r}$};
	\draw (x3a) -- (x1b) node[pos=0.3, above] {$\ell$};
	\draw (x3b) -- (t2) node[pos=0.4, right=2pt] {$b_3$};
	%\draw (-4,4) node {($7$)};
	\end{scope}
\end{tikzpicture}
}

\newcommand\MacroTtwoPerfectMatchingsEight[1]{
\begin{scope}[scale={#1}]
	\MacroTtwoPerfectMatchings
	\begin{scope}[line width=4pt]
	\draw (s) -- (z) node[pos=0.4, below] {$b_1$};
	\draw [postaction={decorate}] (t1a) -- (x1a) node[pos=0.3, above] {$2$};
	\draw (t1b) -- (x1c) node[pos=0.3, below] {$2$};
	\draw (x1b) -- (x2) node[pos=0.3, left=2pt] {$\ell$};
	\draw  (x3a) -- (P) node[pos=0.6, right] {$\ddot{r}$};
	\draw (x3b) -- (t2) node[pos=0.4, right] {$b_3$};
	%\draw (-4,4) node {($8$)};
	\end{scope}
\end{scope}
}

\newcommand\TtwoPerfectMatchingsWithDiagonalsEight[1]{
\begin{tikzpicture}[scale={#1}]
\MacroTtwoPerfectMatchingsEight{1}
\MacroTtwoPerfectMatchingsWithDiagonals
\node at (-8,0) {};\node at (6,0) {};
\end{tikzpicture}
}


\newcommand\TtwoPerfectMatchingsEight[1]{
\begin{tikzpicture}[scale={#1}]
\MacroTtwoPerfectMatchingsEight{1}
\end{tikzpicture}
}

\newcommand\TtwoPerfectMatchingsNine[1]{
\begin{tikzpicture}[scale={#1}]
	\MacroTtwoPerfectMatchings
	\begin{scope}[line width=4pt]
	\draw (s) -- (z) node[pos=0.4, below] {$b_1$};
	\draw [postaction={decorate}] (t1a) -- (x1a) node[pos=0.3, above] {$2$};
	\draw (t1b) -- (x1c) node[pos=0.5, below] {$2$};
	\draw (x2) -- (P) node[pos=0.3, below]{$\dot{r}$};
	\draw (x3a) -- (x1b) node[pos=0.3, below]{$\ell$};
	\draw (x3b) -- (t2) node[pos=0.4, right=2pt] {$b_3$};
	%\draw (-4,4) node {($9$)};
	\end{scope}
\end{tikzpicture}
}
%%%%%%%%%%%%%%%%%%%%%%%%%%%%
%%%%%%%%%% LIFTED T-PATHS %%%%%%%%%
\newcommand \tilTriangleDiagonalL {
	\begin{scope}[orientedTriangulationStyle]
	\draw[style = thick, postaction={decorate}] (0,4) -- (4,0) node[pos=0.2, right=15pt]
	%{$\tiltaui{1}$ = 
	{$\tilell$}; % diagonal, lift of ell
	\end{scope}
}



%%%%%%%%%%%%%%
\newcommand \inducedmapLAorLBCoordinates {
	\coordinate (P2) at (0,4); \coordinate (P1) at (4,0);
	\coordinate (rightd) at (0.2,0); \coordinate (leftd) at (-0.4,0); 
	\coordinate (downd) at (0,-0.3); \coordinate (upd) at (0,0.3); 
	\coordinate (x) at (0,0); 
	\coordinate (y) at (4,4);
}  
\newcommand \inducedmapLAorLBleft[1] {
\begin{tikzpicture}[scale={#1}] 	
	\begin{scope}[xshift=5cm]
	 \tilTriangleSelfFoldedTwo{1}{$\a$}{\tiny $\b$} 	
	\tilTriangleDiagonalL
	
	\inducedmapLAorLBCoordinates
	\draw (x) node[right=30pt, above=3pt] {$\tilS$};
		
	 \begin{scope}[inducedmapTPathStyle] 	
		\draw ($(P2)+3*(rightd)+(downd)$) 
		-- ($(P1)+(leftd)+2*(upd)$) 
		node[pos=0.2, left=3pt]{$\tilom_j$} 
		-- ($(y)+(leftd)+2*(downd)$)
		node[pos=0.6, right=15pt]{$\tilom_{j+1}$};
	\end{scope}
	\end{scope}
	
	\begin{scope}[xshift=5cm, yshift=-5cm]
		  \tilTriangleSelfFoldedTwo{1}{$\a$}{\tiny $\b$} 	
		\tilTriangleDiagonalL
		
		\inducedmapLAorLBCoordinates
		\draw (x) node[right=30pt, above=3pt] {$\tilS$};
		
		\begin{scope}[inducedmapTPathStyle]
			\draw ($(P1)+(upd)$) 
			-- ($(P2)+3*(rightd)+(downd)$) 
			node[pos=0.2, left=8pt]{$\tilom_j$} 
			-- ($(y)+2*(leftd)+(downd)$)
			node[pos=1, above=11pt]{$\tilom_{j+1}$};
		\end{scope}
	\end{scope}
\end{tikzpicture}
} 

%%%%
%%%%
\newcommand \inducedmapLAorLBright[1] {
\begin{tikzpicture}[scale={#1}] 	

	\begin{scope}[xshift=15cm]
		\MacroSelfFoldedLocalTriangulation{1}
		\inducedmapLAorLBCoordinatesOnS
		\draw (x) node[left=30pt, above=3pt] {$\Cn$};
		
		\begin{scope}[inducedmapTPathonCnStyle]
			\draw[postaction={decorate}] ($(x)+4*(upd)+3*(leftd)$) ..
			 controls (-2.5,3.5) and (3,4) .. ($(x)+(rightd)+(upd)$) 
			node[pos = 0.7, left=1.5pt]{$w_j$};
			\draw[ postaction={decorate}] 
			($(x)+(rightd)+(upd)$) 
			 .. controls (3,2) and (2,3.5) .. ($(y)+(downd)$)
		node[pos = 0.5, right=3pt]{$w_{j+1}$};
		\end{scope}
	\end{scope}
	
	\begin{scope}[yshift=-5cm, xshift=15cm]
		\MacroSelfFoldedLocalTriangulation{1}
		\inducedmapLAorLBCoordinatesOnS
		\draw (x) node[left=30pt, above=3pt] {$\Cn$};
		
		\begin{scope}[inducedmapTPathonCnStyle]
			\draw[postaction={decorate}] 
			($(x)+2*(rightd)+2*(upd)$) ..
			controls (3,4) and (-3,4) ..
			($(x)+(upd)+(leftd)$) 
			node[pos = 0.8, right=1.5pt]{$w_j$};
			
			\draw[postaction={decorate}] ($(x)+(upd)+(leftd)$) 
			.. controls (-3,2) and (-2,3.5) .. ($(y)+(downd)$) 
			node[pos = 0.5, left=3pt]{$w_{j+1}$};
		\end{scope}
	\end{scope}
\end{tikzpicture}
} 
%%%%%%%%%
\newcommand\MacroRLTileMatchingA[2]{
\begin{scope}[scale={#1}]
	\MacroRLTileWithoutDiagonal{1} % matching a
	\draw[line width = {#2}] (v1)-- (v2);
	%\draw ($(v2)$) node[left=5pt, above=10pt] {$(a)$};
	\draw (v3) node[left=10pt, below=5pt] {$+$};
\end{scope}
}

\newcommand\MacroRLTileMatchingC[2]{ % A is possible for this matching
\begin{scope}[scale={#1}]
	\MacroRLTileWithoutDiagonal{1} % matching a
	\draw[line width = {#2}] (v3)-- (v4);
	%\draw ($(v2)$) node[left=5pt, above=10pt] {$(c)$};
	\draw (v3) node[left=10pt, below=5pt] {$+$};
\end{scope}
}

\newcommand\MacroRLTileMatchingB[2]{
\begin{scope}[scale={#1}]
	\MacroRLTileWithoutDiagonal{1} % matching a
	\draw[line width = {#2}] %(v2)-- (v3) 
	(v1) -- (v4);
	%\draw ($(v2)$) node[left=5pt, above=10pt] {$(b)$};
	\draw (v3) node[left=10pt, below=5pt] {$+$};
\end{scope}
}
\newcommand \RLmatching[1] {
\begin{tikzpicture}[scale={#1}] 			
	\MacroRLTileMatchingA{0.55}{5pt}	
	\begin{scope}[yshift=-6cm]
		\MacroRLTileMatchingB{0.55}{5pt}
	\end{scope}
\end{tikzpicture}
} 
\newcommand \inducedmapRLCoordinates {
	\coordinate (P2) at (0,4); \coordinate (P1) at (4,0);
	\coordinate (rightd) at (0.2,0); \coordinate (leftd) at (-0.4,0); 
	\coordinate (downd) at (0,-0.3); \coordinate (upd) at (0,0.3); 
	\coordinate (x) at (0,0); %\coordinate (y) at (4,4);
}  
\newcommand \RLtilS[1] {
\begin{tikzpicture}[scale={#1}] 	
	\begin{scope}
		 \tilTriangleSelfFoldedOne{1}{$\ddot{r}$}{$r$}{$\tiltaui{1} = \ell$}
		 \begin{scope}[inducedmapTPathStyle]
	 		\inducedmapRLCoordinates
			\draw ($(x)+(rightd)+(upd)$) -- ($(P2)+(rightd)+(downd)$) 
			node[pos=0.3, right=3pt]{$\tilom_1$} 
			-- ($(P1)+(leftd)+(-2,2)$)
			node[pos=0.6, right=10pt]{$\tilom_2$};
		\end{scope}
	\draw (x) node[left=10pt, below=3pt] {$\tilS$};
	\end{scope}
	\begin{scope}[yshift=-6cm]
		%\tilTriangleSelfFoldedOne{#1} % path from vertical r (ddot r) then ell
		  \tilTriangleSelfFoldedOne{1}{$\ddot{r}$}{$r$}{$\tiltaui{1} = \ell$}
		
		\begin{scope}[inducedmapTPathStyle]
			\inducedmapRLCoordinates
			\draw ($(x)+(upd)+(rightd)$) -- ($(P1)+(upd)+(leftd)$) 
			node[pos=0.3, above=3pt]{$\tilom_1$} 
			-- ($(P2)+(leftd)+(2,-2)$)
			node[pos=0.6, right=7pt]{$\tilom_2$};
		\end{scope}
		\draw (x) node[left=10pt, below=3pt] {$\tilS$};		
	\end{scope}
\end{tikzpicture}
} 
\newcommand \RLactual[1] {
\begin{tikzpicture}[scale={#1}] 	
	\begin{scope}
		\inducedmapLRRLCoordinatesOnS
		\MacroSelfFolded{2}{orientedTriangulationStyle}
		\draw (x) node[left=30pt, above=2pt] {$\Cn$};
		\begin{scope}[inducedmapTPathonCnStyle]
			\draw [postaction={decorate}]  ($(P)+(leftd)$) -- ($(x)+3*(upd)+(leftd)$) 
			node[pos=0.5, right=4pt]{$w_1$};
		
			\draw [postaction={decorate}]  ($(x)+3*(upd)+ (leftd)$) .. controls (-5,8) and (4,6) .. ($(x)+(1,2)$) 
			node[pos = 0.2, left=3pt]{$w_2$};
		\end{scope}
	\end{scope}
	\begin{scope}[yshift=-6.3cm]
		\inducedmapLRRLCoordinatesOnS
		\MacroSelfFolded{2}{orientedTriangulationStyle}
		\draw (x) node[left=30pt, above=2pt] {$\Cn$};
		\begin{scope}[inducedmapTPathonCnStyle]
			\draw [postaction={decorate}]  ($(P)+(leftd)$) -- ($(x)+3*(upd)+(leftd)$) 
			node[pos=0.5, right=4pt]{$w_1$};
			\draw [postaction={decorate}]   ($(x)+3*(upd)+ (leftd)$)  .. 
			controls  (5,8) and  (-4,6) ..  ($(x)+(-1,2)$) 
			node[pos = 0.2, left=3pt]{$w_2$};
		\end{scope}
	\end{scope}
\end{tikzpicture}
} 


  
\newcommand \inducedmapLRRLCoordinates {
	\coordinate (P) at (0,4);
	\coordinate (rightd) at (0.2,0); \coordinate (leftd) at (-0.2,0); 
	\coordinate (downd) at (0,-0.2); \coordinate (upd) at (0,0.2); 
	\coordinate (x1) at (0,0); \coordinate (x2) at (-4,4); \coordinate (x3) at (4,4);
}  
  
\newcommand \inducedmapLRRLCoordinatesOnS {
	\coordinate (P) at (0,4); \coordinate (x) at (0,0);
	\coordinate (rightd) at (0.2,0); \coordinate (leftd) at (-0.2,0); 
	\coordinate (downd) at (0,-0.2); \coordinate (upd) at (0,0.2); 
}  

\newcommand \inducedmapLAorLBCoordinatesOnS {
	\coordinate (y) at (0,4); \coordinate (x) at (0,0);
	\coordinate (rightd) at (0.2,0); \coordinate (leftd) at (-0.2,0); 
	\coordinate (downd) at (0,-0.2); \coordinate (upd) at (0,0.2); 
}  
 %%% 
\newcommand\LRRLmatching[1]{
\begin{tikzpicture}[scale={#1}] 	
	\begin{scope}[yshift=1.5cm]	
		\MacroTripleTilesMatchingD{0.55}{5pt}	
	\end{scope}	
	\begin{scope}[yshift=-5.5cm]	
		\MacroTripleTilesMatchingC{0.55}{5pt}	
	\end{scope}
\end{tikzpicture}
}
\newcommand\LRRLtilS[1]{
\begin{tikzpicture}[scale={#1}] 	
\begin{scope}
		\tilDoubleSelfFoldedTriangleComplicatedLabels{1}
		\inducedmapLRRLCoordinates
		\draw (x1) node[left=50pt, above=3pt] {$\tilS$};
		\draw (x1) node[right=6pt] {Lemma \ref{lemma:four_paths} (\ref{item:four_paths_lrrll_nonbacktrack})}; %(4)
	
		 \begin{scope}[inducedmapTPathStyle] 	
			\draw  ($(x1)+(rightd)+(upd)+(-2,2)$) -- ($(x2)+2*(rightd)+(downd)$) 
			node[pos=0.1, left=10pt]{$\tilom_{2k}$} 
			-- ($(P)+(leftd)+(downd)$) 
			%node[pos=0.5, below]{$\tilom_{2k+1}$} 
			-- ($(x1)+(leftd)+(upd)$)
			%node[pos=0.3, right=10pt]{$\tilom_{2k+2}$}			
			-- ($(x3)+(leftd)$)
			%node[pos=0.54, right=12pt]{$\tilom_{2k+3}$}
			-- ($(x1)+(0.8,0.8)+5*(rightd) + 2*(upd)$) 
			node[pos=0.83, right=4pt]{$\tilom_{2k+4}$} ;
			
		\end{scope}
	\end{scope}
	
	\begin{scope}[yshift=-7cm]
		\tilDoubleSelfFoldedTriangleComplicatedLabels{1}
		\inducedmapLRRLCoordinates
		\draw (x1) node[left=50pt, above=3pt] {$\tilS$};
		\draw (x1) node[right=6pt] {Lemma \ref{lemma:four_paths} (\ref{item:four_paths_llrrl_nonbacktrack})}; % (2)
		
		\begin{scope}[inducedmapTPathStyle]
			\draw 
			 ($(x1)+9*(leftd)+6*(upd)$) 
			--
			 ($(x1)+(rightd)+(upd)+(-4,4)$)
			 node[pos=0.3,left]{$\tilom_{2k}$} 
			  -- ($(x1)+(leftd)+2*(upd)$) 
			%node[pos=0.2, left=10pt]{$\tilom_{2k+1}$} 
			-- ($(P)+(leftd)+(downd)$) 
			%node[pos=0.7, left]{$\tilom_{2k+2}$} 
			-- ($(x3)+2*(leftd)+(downd)$)
			%node[pos=0.5, below]{$\tilom_{2k+3}$}
			-- ($(x3)+(leftd)+(-2.2,-2)$)
			node[pos=0.85, right=15pt]{$\tilom_{2k+4}$};
		\end{scope}
	\end{scope}
\end{tikzpicture}
}
\newcommand\LRRLactual[1]{
\begin{tikzpicture}[scale={#1}] 	
\begin{scope}
		\inducedmapLRRLCoordinatesOnS
		\MacroSelfFolded{2}{orientedTriangulationStyle}
		\draw (x) node[left=40pt, above=3pt] {$\Cn$};
		
		\begin{scope}[inducedmapTPathonCnStyle]
			\draw [postaction={decorate}]  ($(x)+15*(upd)+ 8*(leftd)$)
			 .. controls (-3,8.5) and (4,6) .. ($(x)+6*(upd)+1*(rightd)$)
			node[pos = 0.3, left=3pt]{$w_{2k}$};

			\draw [postaction={decorate},rounded corners=10pt] 
			($(x)+6*(upd)+1*(rightd)$)
			-- ($(P)+(0.5,1)$) 
			-- ($(P)+(-0.5,1)$) 
			-- ($(P)+(-0.5,-2)$) 
			--(x);% ($(x)+(leftd)+3*(downd)$);

			\draw [postaction={decorate}, ->]  (x) %($(x)+(leftd)+3*(downd)$)
			.. controls (3,3) and (3,6) .. 
			($(P)+(-1.5,3)$) node[pos=0.3, right=3pt] {$w_{2k+3}$} %;	
			--  (2,7.3)  --  (3.1,4) node[pos=0.1,right]{$w_{2k+4}$};		
		\end{scope}
	\end{scope}
	\begin{scope}[yshift=-7cm]
		\inducedmapLRRLCoordinatesOnS
		\draw (x) node[left=40pt, above=3pt] {$\Cn$};
		\MacroSelfFolded{2}{orientedTriangulationStyle}
		\begin{scope}[inducedmapTPathonCnStyle]
			
			\draw [postaction={decorate}] 
			%($(x)+5*(upd)+5*(leftd)$) ..
			(-2,1) ..
			controls (-2.6,2) and (-3,3)
			.. ($(P)+(-2.1,1)$) 
			node[pos=0.3, left]{$w_{2k}$};
			
			\draw [postaction={decorate}] 
			($(P)+(-2.1,1)$) %--  ($(x)+(-1.5,1)$) -- 
			.. controls (-2.5,3) and (-2,1) .. ($(x)+(0.2,0)$) %($(x)+5*(upd)+1*(rightd)$)
			node[pos = 0.4, left=8pt]{$w_{2k+1}$};

			\draw [postaction={decorate},->]  
			 ($(P)+(-0.5,-2)$)
			.. controls (-4,8) and (4,6) .. ($(x)+(1.2,2)$) 
			node[pos = 0.9, right=4pt]{$w_{2k+4}$};	
		\end{scope}
		
			\draw[blue, very thick, %line width = 2pt, 
			style=dashed, rounded corners = 5pt,
			decoration={ markings, mark=at position .8 with {\arrow[blue, line width=2.5pt]{to};}}, 
			postaction={decorate}]
			($(x)+(0.2,0)$) %($(x)+5*(upd)+1*(rightd)$)
			-- ($(P)+(0.4,1)$) 
			-- ($(P)+(-0.4,1)$)
			-- ($(P)+(-0.5,-2)$);
	\end{scope}	
\end{tikzpicture}
}


%%%%%%%%
%begin new%%
\newcommand\TikzPMLRR[1]{
\begin{tikzpicture}[scale={#1}] 	
\MacroTripleTilesOrientationDiagonalsLR{1}
\draw [tilTwoSnakeGraphMatchingStyle] (x1c) -- (P) node[pos=0.3, right] {$r$};
\end{tikzpicture}
}
\newcommand\TikzPMLLR[1]{ 
\begin{tikzpicture}[scale={#1}] 	
\MacroTripleTilesOrientationDiagonalsLR{1}
\draw [tilTwoSnakeGraphMatchingStyle] (x1b) -- (x2) node[pos=0.3, left] {$\ell$};
\end{tikzpicture}
}
\newcommand\TikzPMLRdotR[1]{
\begin{tikzpicture}[scale={#1}] 	
\MacroTripleTilesOrientationDiagonalsLR{1}
\draw [tilTwoSnakeGraphMatchingStyle] (x2) -- (P) node[pos=0.7, below] {$\dot{r}$};
\end{tikzpicture}
}

%%%
\newcommand\TikzPMRLL[1]{
\begin{tikzpicture}[scale={#1}] 	
\MacroTripleTilesOrientationDiagonalsRL{1}
\draw [tilTwoSnakeGraphMatchingStyle] (x3a) -- (x1b) node[pos=0.3, above] {$\ell$}; 
\end{tikzpicture}
}
\newcommand\TikzPMRRL[1]{ 
\begin{tikzpicture}[scale={#1}] 	
\MacroTripleTilesOrientationDiagonalsRL{1}
\draw [tilTwoSnakeGraphMatchingStyle] (x3b) -- (P) node[pos=0.3, below] {$r$};
\end{tikzpicture}
}
\newcommand\TikzPMRRddotL[1]{
\begin{tikzpicture}[scale={#1}] 	
\MacroTripleTilesOrientationDiagonalsRL{1}
\draw [tilTwoSnakeGraphMatchingStyle] (x3a) -- (P) node[pos=0.3, left] {$\ddot{r}$};
\end{tikzpicture}
}

%% end new %%


\newcommand\BacktrackLRRLmatching[1]{
\begin{tikzpicture}[scale={#1}] 	
	\begin{scope}[yshift=1.5cm] 
		\MacroTripleTilesMatchingB{0.55}{5pt}	 %	Lemma 3.9(3)
	\end{scope}
	\begin{scope}[yshift=-5cm]	
		\MacroTripleTilesMatchingA{0.55}{5pt}	%	Lemma 3.9(1)
	\end{scope}
\end{tikzpicture}
}
\newcommand\BacktrackLRRLtilS[1]{
\begin{tikzpicture}[scale={#1}] 	
	\begin{scope}
		\tilDoubleSelfFoldedTriangleComplicatedLabels{1}
		\inducedmapLRRLCoordinates
		\draw (x1) node[left=50pt, above=3pt] {$\tilS$};
		\draw (x1) node[right=6pt] {Lemma \ref{lemma:four_paths} (\ref{item:four_paths_lrrll_backtrack})}; %  (3)
	
		 \begin{scope}[inducedmapTPathStyle] 	%	Lemma 3.9(3)
			\draw  ($(x2)+2*(rightd)+(downd)$) --
			 ($(x1)+(leftd)+2*(upd)$) 
			node[pos=0.1, left=5pt]{$\tilom_{2k}$} 
			-- ($(P)$) 
			node[pos=0.6, left]{$\tilom_{2k+1}$} 
			-- ($(x1)+2*(rightd)+2*(upd)$)
			node[pos=0.3, right=2pt]{$\tilom_{2k+2}$}
			-- ($(x3)+(leftd)$)
			node[pos=0.5, right=10pt]{$\tilom_{2k+3}$}
			-- ($(x1)+(0.8,0.8)+5*(rightd) + 2*(upd)$) 
			node[pos=0.8, right]{$\tilom_{2k+4}$} ;
		\end{scope}	
	\end{scope}
	\begin{scope}[yshift=-7cm] %	Lemma 3.9(1)
		\tilDoubleSelfFoldedTriangleComplicatedLabels{1}
		\inducedmapLRRLCoordinates
		\draw (x1) node[left=50pt, above=3pt] {$\tilS$};
		\draw (x1) node[right=6pt] {Lemma \ref{lemma:four_paths} (\ref{item:four_paths_llrrl_backtrack})}; %  (1)
	
		 \begin{scope}[inducedmapTPathStyle] 	
			\draw
			($(x1)+9*(leftd)+6*(upd)$) 
			-- ($(x1)+(rightd)+(upd)+(-4,4)$)
			 node[pos=0.3,left]{$\tilom_{2k}$} 
			 --
			 ($(x1)+(leftd)+2*(upd)$) 
			%node[pos=0.2, left=5pt]{$\tilom_{2k+1}$} 
			-- ($(P)$) 
			%node[pos=0.6, left]{$\tilom_{2k+2}$} 
			-- ($(x1)+2*(rightd)+2*(upd)$)
			%node[pos=0.3, right=2pt]{$\tilom_{2k+3}$}
			-- ($(x3)+(leftd)+(-1.2,-1)$)
			node[pos=0.7, right=5pt]{$\tilom_{2k+4}$};
		\end{scope}
	\end{scope}
\end{tikzpicture}
}

\newcommand\BacktrackLRRLactual[1]{
\begin{tikzpicture}[scale={#1}] 

	\begin{scope} %	Lemma 3.9(3)
		\inducedmapLRRLCoordinatesOnS
		\MacroSelfFolded{2}{orientedTriangulationStyle}
		\draw (x) node[left=30pt, above=2pt] {$\Cn$};
		\begin{scope}[inducedmapTPathonCnStyle]
		\draw [postaction={decorate}]  ($(P)+(1,1.5)$) .. controls (1,6) and (-4,7)  ..  ($(x)+3*(upd)+ (leftd)$)
		node[pos = 0.7, left]{$w_{2k}$};
		
		\draw [postaction={decorate}, ->]  ($(x)+4*(upd)+(leftd)$)%  -- ($(P)+2*(downd)+(leftd)$)
		%node[pos=0.5, right=4pt]{$w_1$};
		.. controls  (-0.5,4) and (0.5,4) .. ($(x)+(upd)+(rightd)$)
		.. controls (2,2) and (2, 3) .. ($(P)+10*(rightd)+(0,2)$) node[pos=0.6, right=14pt] {$w_{2k+3}$}
		-- ($(P)+10*(rightd)+(1,-1)$) --   ($(P)+10*(rightd)+(0.5,-3)$) node[pos=0.3,right]{$w_{2k+4}$};
		\end{scope}
	\end{scope}
	
	\begin{scope}[yshift=-6.5cm] %	Lemma 3.9(1)
		\inducedmapLRRLCoordinatesOnS
		\MacroSelfFolded{2}{orientedTriangulationStyle}
		\draw (x) node[left=30pt, above=2pt] {$\Cn$};
		\begin{scope}[inducedmapTPathonCnStyle]
		
		\draw [postaction={decorate}]  ($(P)+(-2,-2.5)$) ..
		 controls  (-4,8)  and (1,6) .. ($(P)+(1,1.5)$) node[pos=0.15,left]{$w_{2k}$};
		
		\draw [postaction={decorate}]  ($(P)+(1,1.5)$) .. controls (1,6) and (-4,7)  ..  ($(x)+3*(upd)+ (leftd)$)
		node[pos = 0.7, left=14pt]{$w_{2k+1}$};
		
		\draw [postaction={decorate}, ->]  ($(x)+4*(upd)+(leftd)$)%  -- ($(P)+2*(downd)+(leftd)$)
		%node[pos=0.5, right=4pt]{$w_1$};
		.. controls  (-0.5,4) and (0.5,4) .. ($(x)+(upd)+(rightd)$)
		.. controls (2,2) and (2, 3) .. ($(P)+10*(rightd)$) node[pos=0.5, right=4pt] {$w_{2k+4}$};
		\end{scope}
	\end{scope}
\end{tikzpicture}
}


\tikzstyle{tilTwoSnakeGraphMatchingStyle}=[line width=3pt, blue]
\newcommand\MacroTripleTilesOrientationDiagonalsLR[1]{ % current

	\tilTtwoSnakeGraphCoordinates
	\MacroTtwoPerfectMatchings
	% fifth tile
	\begin{scope}[tilTwoSnakeGraphMatchingStyle, dashed]
	%\draw (x3a) -- (x3b) node[pos=0.4, right] {$\ell$}; % diagonal
	
	% the fourth tile (negative)
	\draw [postaction={decorate}] (x1b) -- (P) node[pos=0.5, above] {$r$};

	% the third tile (positive)
	\draw [postaction={decorate}] (x1c) -- (x2) node[pos=0.6, below] {$\ell$};
	
	\end{scope}
}

\newcommand\MacroTripleTilesOrientationDiagonalsRL[1]{ 

	\tilTtwoSnakeGraphCoordinates
	\MacroTtwoPerfectMatchings
	% fifth tile
	\begin{scope}[tilTwoSnakeGraphMatchingStyle, dashed]
	\draw (x3a) -- (x3b) node[pos=0.4, right] {$\ell$}; % diagonal
	
	% the fourth tile (negative)
	\draw [postaction={decorate}] (x1b) -- (P) node[pos=0.5, below] {$r$};

	% the third tile (positive)
	%\draw [postaction={decorate}] (x1c) -- (x2) node[pos=0.6, right] {$\ell$};
	
	\end{scope}
}





\newcommand\MacroTripleTilesOrientationWithoutDiagonals[1]{
	\begin{scope}[style=thick]
	\MacroTripleTilesWithoutDiagonals{#1}
	\draw (P) node[left=10pt, below=1pt] {$-$};
	\draw (x3) node[left=10pt, below=2pt] {$+$};
	\draw (yr) node[left=10pt, below=1pt] {$-$};
	\end{scope}
}

\newcommand\MacroTripleTilesMatchingA[2]{
\begin{scope}[scale={#1}]
	\MacroTripleTilesOrientationWithoutDiagonals{1} % matching a
	\draw[line width = {#2}] (P)-- (x5) (x2) -- (x1);
	%\draw ($(P)-(x1)$) node {$(a)$};
\end{scope}
}

\newcommand\MacroTripleTilesMatchingB[2]{
\begin{scope}[scale={#1}]
	\MacroTripleTilesOrientationWithoutDiagonals{1}
	\draw[line width = {#2}] (x1) -- (x3)  (P) -- (x4);
	%\draw ($(P)-(x1)$) node {$(b)$};
\end{scope}
}

\newcommand\MacroTripleTilesMatchingC[2]{
\begin{scope}[scale={#1}]
	\MacroTripleTilesOrientationWithoutDiagonals{1}
	\draw[line width = {#2}] (x3) -- (P) (x2) -- (x1);
	%\draw ($(P)-(x1)$) node {$(c)$};
\end{scope}
}

\newcommand\MacroTripleTilesMatchingD[2]{
\begin{scope}[scale={#1}]
\MacroTripleTilesOrientationWithoutDiagonals{1}
	\draw[line width = {#2}] (P) -- (x2) (x1) -- (x3);
\end{scope}
}


\newcommand\TikzTripleTilesFourPerfectMatchings[2] {
\begin{scope}[scale={#1}]
	\MacroTripleTilesMatchingA{1}{#2}
	
	\begin{scope}[xshift=9cm] % matching b
	\MacroTripleTilesMatchingB{1}{#2}
	 \end{scope}
	 
	 \begin{scope}[yshift=-6cm] % matching c
	 \MacroTripleTilesMatchingC{1}{#2}
	 \end{scope}
	 
	 \begin{scope}[yshift=-6cm, xshift=9cm] % matching d
	 \MacroTripleTilesMatchingD{1}{#2}
	 \end{scope} 
\end{scope}
}

\newcommand \MacroTripleTilesCoordinates {
		\coordinate (x1) at (3,-2); \coordinate (x2) at (0,-2); \coordinate (x3) at (3,0);
		\coordinate (x4) at (-3,0); \coordinate (x5) at (0,2);
		\coordinate (yl) at (-3,-2); \coordinate (yr) at (3,2);
		\coordinate (P) at (0,0);
}

\newcommand \MacroTripleTilesWithoutDiagonals[1] {
	\begin{scope}[scale=#1]
		\MacroTripleTilesCoordinates
		\draw (P)-- (x5) node[pos=0.5, left=1pt] {$r$}; 
		\draw (x5) -- (yr) node[pos=0.5, above=1pt] {\tiny $\b$};
		\draw (yr) -- (x3) node[pos=0.5, right=1pt] {$\a$}; 
		\draw (x3) -- (P) node[pos=0.5, above=1pt] {$\ddot{r}$};
		
		\draw (P) -- (x2) node[pos=0.5, left=1pt] {$\dot{r}$};
		 \draw (x2) -- (x1) node[pos=0.5, below=1pt] {$\ell$}; 
		 \draw (x1) -- (x3) node[pos=0.5, right=1pt] {$\ell$};
		 
		 \draw (P) -- (x4) node[pos=0.5, above=1pt] {$r$};
		 \draw (x4) -- (yl) node[pos=0.5, left=1pt] {$\a$};
		 \draw (yl) -- (x2) node[pos=0.5, below=1pt] {\tiny $\b$};		
	\end{scope}
}

\newcommand \MacroRLTileWithoutDiagonalCoordinates {
		\coordinate (v1) at (0,0); \coordinate (v2) at (0,2); 
		\coordinate (v3) at (3,2); \coordinate (v4) at (3,0); 
		%\coordinate (v5) at (6,2); \coordinate (v6) at (6,0); 
}

\newcommand \MacroRLTileWithoutDiagonal[1] {
	\begin{scope}[scale=#1]
		\MacroRLTileWithoutDiagonalCoordinates
		\draw (v1)-- (v2) node[pos=0.5, left=1pt] {$\ddot{r}$}; 
		\draw (v2) -- (v3) node[pos=0.5, above=1pt] {$\a$};
		\draw (v3) -- (v4) node[pos=0.5, right=1pt] {\tiny $\b$}; 
		\draw (v4) -- (v1) node[pos=0.5, above=1pt] {$r$};	
	\end{scope}
}

\newcommand\MacroSubsequenceDoublecross{ % the triangle
	\draw (0,0) -- (-4,0)   node[pos=0.05, below] {$w_{2k+1}$} 
	-- (0,5) node[pos=0.4, left] {$w_{2k+2}$}
	-- (0,0) node[pos=0.4, right] {$w_{2k}$};
	\draw[cyan, line width=3pt] (-3,3) -- (1,4) node[pos=0.2, above]{\Large{$\be$}};
}

\newcommand\MacroSubsequencePuncture{
	\draw [fill=violet] (-2,-1) circle (.6ex) node[violet, left=5pt, above=1pt] {\tiny $P$};
}
\newcommand\TikzSubsequenceDoublecrossContradiction[1]{
\begin{tikzpicture}[scale={#1}, orientedTriangulationStyle]
	\MacroSubsequenceDoublecross
	\MacroSubsequencePuncture
	\draw[line width=3pt,red, dashed, rounded corners=20pt]  (1,1) -- (-1,1) node[pos=0.2,left, below] {\Large{$\si$}}
	-- (-1, -1.5) -- (-3,-1.5) -- (-3,1) -- (-4,1);
\end{tikzpicture}
}

\newcommand\TikzSubsequenceDoublecrossPossible[1]{
\begin{tikzpicture}[scale={#1}, orientedTriangulationStyle]
	\MacroSubsequenceDoublecross
	\MacroSubsequencePuncture
	\draw[line width=3pt, red, dashed, rounded corners=20pt]  (1,1) -- (-1,1) node[pos=0.2,left, below] {\Large{$\si$}}
	-- (-1, -1.5) -- (-3,-1.5)  -- (-2,2) -- (1,2);
	
	\begin{scope}[xshift=200pt]
	\MacroSubsequenceDoublecross
	\draw [fill=violet] (1.5,1.7) circle (.6ex) node[violet, left=5pt, above] {\tiny $P$};
	 \draw[line width=3pt, red, dashed, rounded corners=20pt] (-1, -1.5) -- (-1,1) 
	 -- (2,0.7) -- (2,2.5) -- (-2,2) -- (-2,-1.5) node[pos=0.8,left] {\Large{$\si$}};
	\end{scope}
	
\end{tikzpicture}
}



%%%%%%% start of new since sept 2014
%%%%%
%%%%% 4 types of triangles and the ways to cross them

%%% ordinary triangle %%%%

\newcommand\MacroTriangle{
\draw[thick]
(0,0) -- (4,0) -- (2,4) -- (0,0);
\draw [fill=olive] (0,0) circle (.6ex);
\draw [fill=olive] (4,0) circle (.6ex);
\draw [fill=olive] (2,4) circle (.6ex);
}
\newcommand\TikzTriangle[1]{
\begin{tikzpicture}[scale={#1} ]
\MacroTriangle
\node at (4.5,0) {}; \node at (-0.5,0) {};
\end{tikzpicture}
}

\newcommand\TikzTriangleCrossesTwoEdges[1]{
\begin{tikzpicture}[scale={#1} ]
\MacroTriangle
\draw (2,-0.5) node[] {$\tg{k}$};
\draw (4,2.8) node[] {$k+1$};
\draw (0.6,2.8) node[] {$k$};
\draw[TPathStyle] 
(0.5,2) -- (3.4,2) node[pos=0.5, below]{$\ga$};
\end{tikzpicture}
}

\newcommand\TikzTriangleCrossesOneEdge[1]{
\begin{tikzpicture}[scale={#1} ]
\MacroTriangle
\draw[TPathStyle,->] 
(0,0) -- (3.7,2.1) node[pos=0.5, above]{$\ga$};
\draw (3.1,3) node[] {$1$};
\draw (2,-0.5) node[] {$\tg{0}$};
\draw (0.2,2.8) node[] {$\tg{-1}$};

\begin{scope}[yshift=-6cm]
\MacroTriangle
\draw[TPathStyle, <-] 
(0.2,0.1) -- (3.6,2) node[pos=0.5, above]{$\ga$};
\draw (3.1,3) node[] {$d$};
\draw (2,-0.5) node[] {$\tg{d}$};
\draw (0.2,3) node[] {$\tg{d+1}$};
\end{scope}

\end{tikzpicture}
}

%%% triangle with two vertices and three sides %%%%

\newcommand\MacroTwoVertexTriangle{ % the outer side
\draw[style = thick]%, rounded corners = 20pt]
	(0,0) .. controls (-1.7,1) and (-1.6,3) .. (0,3.5) .. 
	controls (1.6,3)  and (1.7,1) .. (0,0); 
	\draw [fill=olive] (0,0) circle (.6ex);
	\draw [fill=olive] (0,3.5) circle (.6ex);

\draw[style = thick, rounded corners = 20pt] % inner loop
	(0,0) .. controls (-0.8,1) and (-0.5,1.7) .. (0,2.5) .. 
	controls (0.5,1.7)  and (0.8,1) .. (0,0); 
}
\newcommand\TikzTwoVertexTriangle[1]{
\begin{tikzpicture}[scale={#1} ]
\MacroTwoVertexTriangle
\end{tikzpicture}
}


\newcommand\TikzTwoVertexTriangleCrossesTwoEdgesNoLoop[1]{
\begin{tikzpicture}[scale={#1} ]
\MacroTwoVertexTriangle
\draw[TPathStyle] 
(-1.4,2.5) -- (1.4,2.5) node[pos=0.5, above]{$\ga$};
\draw (0.1,1.5) node[] {$\tg{k}$};
\draw (-1.4,2.3) node[] {$k$};
\draw (1.4,2.3) node[] {$k+1$};
\end{tikzpicture}
}

\newcommand\TikzTwoVertexTriangleCrossesTwoEdgesKOneisLoop[1]{
\begin{tikzpicture}[scale={#1} ]
\MacroTwoVertexTriangle
\draw[TPathStyle] 
(-1.3,1.2) -- (-0.2,1.2) node[pos=0.4, above]{$\ga$} 
node[black,pos=0,above=24pt]{$k$}
node[black,pos=1,above=18pt]{$k+1$};
\end{tikzpicture}
}


\newcommand\TikzTwoVertexTriangleCrossesOneEdgeLoop[1]{
\begin{tikzpicture}[scale={#1} ]
\MacroTwoVertexTriangle
\draw[TPathStyle,->] (0,3.5) -- (0.1,1.6) node[pos=0.5, left]{$\ga$};
\draw (-0.3,1.3) node[] {$1$};
\draw (-1.7,2.3) node[] {$\tg{0}$};
\draw (1.8,2.3) node[] {$\tg{-1}$};
\begin{scope}[yshift=-4cm]
\MacroTwoVertexTriangle
\draw[TPathStyle,<-] (0,3.4) -- (0.2,1.7) node[pos=0.5, left]{$\ga$};
\draw (-0.3,1.3) node[] {$d$};
\draw (-1.7,2.3) node[] {$\tg{d}$};
\draw (1.8,2.3) node[] {$\tg{d+1}$};
\end{scope}
\end{tikzpicture}
}

\newcommand\TikzTwoVertexTriangleCrossesOneEdgeNotLoop[1]{
\begin{tikzpicture}[scale={#1} ]
\MacroTwoVertexTriangle
\draw[->,TPathStyle,rounded corners=15pt] 
(0,0) -- (-1,1) -- (-0.7,2.5) -- (1.3,3) node[pos=0.4, above]{$\ga$};
\draw (0.1,1.5) node[] {$\tg{0}$};
\draw (-1.8,2.3) node[] {$\tg{-1}$};
\draw (1.4,2.3) node[] {$1$};
\begin{scope}[yshift=-4cm]
\MacroTwoVertexTriangle
\draw[PathArrowAtPosStyleReversedFive,TPathStyle,rounded corners=15pt] 
(0,0) -- (-1,1) -- (-0.7,2.5) -- (1.2,3) node[pos=0.4, above]{$\ga$};
\draw (0,1.6) node[] {$\tg{d}$};
\draw (-1.8,2.8) node[] {$\tg{d+1}$};
\draw (1.5,2.3) node[] {$d$};
\end{scope}
\end{tikzpicture}
}

%%% triangle with one vertex and three sides %%%%

\newcommand\MacroVeryThinNoose{
\draw[style = thick, rounded corners = 10pt] % inner loop
	(0,0) .. controls (-0.6,1) and (-0.4,1.7) .. (0,2.2) .. 
	controls (0.4,1.7)  and (0.6,1) .. (0,0); 
}

\newcommand\MacroVeryThinNooseHidden{
\draw[style = thick, rounded corners = 5pt] % inner loop
	(0,0) .. controls (-0.6,1) and (-0.4,1.7) .. (0,2) .. 
	controls (0.4,1.7)  and (0.6,1) .. (0,0); 
}

\newcommand\MacroOneVertexTriangle{
\draw[style = thick, rounded corners = 30pt]
	(0,0) .. controls (-2.7,1) and (-2.4,2.5) .. (0,3.5) .. 
	controls (2.4,2.5)  and (2.7,1) .. (0,0); 
	
	\draw [fill=olive] (0,0) circle (.6ex);
	
\begin{scope}[rotate=30]
\MacroVeryThinNoose
\end{scope}

\begin{scope}[rotate=-35]
\MacroVeryThinNoose
\end{scope}
}

\newcommand\MacroOneVertexTriangleHidden{
    \begin{scope}[rotate=30]
    \MacroVeryThinNooseHidden \end{scope}

    \begin{scope}[rotate=-35]
    \MacroVeryThinNooseHidden \end{scope}

    \begin{scope}[rotate=-100]
    \MacroVeryThinNooseHidden \end{scope}
    
    \draw [fill=olive] (0,0) circle (.6ex);
}

\newcommand\TikzOneVertexTriangle[1]{
\begin{tikzpicture}[scale={#1} ]
\MacroOneVertexTriangle
\end{tikzpicture}
}

\newcommand\TikzOneVertexTriangleCrossesTwoEdges[1]{
\begin{tikzpicture}[scale={#1} ]
\MacroOneVertexTriangle
\draw[TPathStyle] 
(-0.6,1.5) --(0.8,1.5) node[pos=0.5, above]{$\ga$};

\begin{scope}[yshift=-3.5cm]
\MacroOneVertexTriangle
\draw[TPathStyle,rounded corners=15pt] 
 (-2,1) --(-0.7,1) node[pos=0.4, above]{$\ga$};
\end{scope}

\begin{scope}[yshift=-6cm] 
\MacroOneVertexTriangleHidden
\draw[TPathStyle] (-0.6,1) --(0.8,1) node[pos=0.5, above]{$\ga$};    
\end{scope}
\end{tikzpicture}
}

\newcommand\TikzOneVertexTriangleCrossesOneEdge[1]{
\begin{tikzpicture}[scale={#1} ]
\MacroOneVertexTriangle
\draw[TPathStyle] 
(0,0) -- (0.2,3.5) node[pos=0.5, left]{$\ga$};

\begin{scope}[yshift=-3.5cm]
\MacroOneVertexTriangle
\draw[TPathStyle,rounded corners=15pt] 
(0,0) -- (-1.6,1) -- (-1.6,2.3) --  (0,2.1) --(0.4,1) node[pos=0.4, above]{$\ga$};
\end{scope}

\begin{scope}[yshift=-6cm] 
\MacroOneVertexTriangleHidden
\draw[TPathStyle,rounded corners=15pt] 
(0,0) -- (-1.6,1) -- (-1.6,2.3) --  (0,2.1) --(0.4,1) node[pos=0.4, above]{$\ga$};
\end{scope}

\end{tikzpicture}
}

%%% self-folded triangle %%%%%

%\newcommand\MacroTikzArrowAtPos[1]{
%decoration={ markings, mark=at position {#1} with {\arrow[black, line width=2pt]{stealth};}}
%}



\newcommand\MacroSelfFoldedTriangleNoLabel{
	\draw[thick]
	(0,0) -- (0,2);
	
	\draw [fill=violet] (0,2) circle (1.2*\TikzPunctureActualSize);
	
	\draw[style = thick, rounded corners = 20pt,NooseArrowAtPosStyleSix]	
	(0,0) .. controls (-1.8,1) and (-1.5,2.5) .. (0,3.5) .. 
	controls (1.5,2.5)  and (1.8,1) .. (0,0); %node[pos=0.5, right] {$\ell$};
	\draw [fill=olive] (0,0) circle (.6ex);
}
\newcommand\TikzSelfFoldedTriangle[1]{
\begin{tikzpicture}[scale={#1} ]
\MacroSelfFoldedTriangleNoLabel
\draw (-1.6,2) node {$\ell$};
\draw (0.3,1) node {$r$};
\end{tikzpicture}
}



%%% self-folded triangle where both radius and noose are crossed
\newcommand\TikzSelfFoldedTriangleCrossesRadiusNoose[1]{
\begin{tikzpicture}[scale={#1} ]
\MacroSelfFoldedTriangleNoLabel
\draw (-1.3,2.8) node {$k$};
\draw (0.2,1.4) node {$k+1$};
\draw (1.3,3.2) node {$k+2$};
\draw[TPathStyle,->] 
(-1.5,1) -- (1.6,1) %(0.4,1) 
node[pos=0.05, below]{$\ga$};
% \begin{scope}[yshift=-4cm]
% \draw (1.3,3.2) node {$k+1$};
% \draw (-0.2,1.5) node {$k$};
% \MacroSelfFoldedTriangleNoLabel
% \draw[TPathStyle,->] 
% (-0.2,1) -- (1.6,1) node[pos=0.4, above]{$\ga$};
% \end{scope}
\end{tikzpicture}
}

%%% self-folded triangle when only the noose is crossed
\newcommand\TikzSelfFoldedTriangleCrossesNoose[1]{
\begin{tikzpicture}[scale={#1} ]
\MacroSelfFoldedTriangleNoLabel
\draw[->,TPathStyle] 
(0,2) -- (0.5,3.6) node[pos=0.5, left]{$\ga$};
\draw (-0.1,1.2) node {$\tg{-1}\,\,\, \tg{0}$};
\draw (1.5,2.3) node {$1$};

\begin{scope}[black,yshift=-4cm]
\MacroSelfFoldedTriangleNoLabel
\draw[<-,TPathStyle] 
(0,2.2) -- (0.5,3.5) node[pos=0.5, left]{$\ga$};
\draw (-0.1,1.2) node {$\tg{d+1}\,\,\, \tg{d}$};
\draw (1.5,2.3) node {$d$};
\end{scope}

\end{tikzpicture}
}

%%%%%%
%%%%%%
% end of new since sept 2014

%%% current begins
\newcommand\TikzPunctureActualSize{0.1}
\newcommand\TikzPointActualSize{0.07}

\newcommand\MacroPunctureClosestToKOrKOneTwoVertexTriangle{
	\begin{scope}[rotate=-90]\MacroTwoVertexTriangle\end{scope}
	\draw (1.8,-1.5) node[black]{$\tg{k}$};
	\draw (-0.3,0) node[olive] {$v$};
	\draw[fill=violet] (1.5,0) circle(\TikzPunctureActualSize) node[violet, above] {$\puncture$};
	\draw[CyanGammaStyle] (0.3,-1.2) -- (1.5,1.7) node[left,pos=0.9]{$\ga$};
}
\newcommand\MacroPunctureClosestToKOrKOneTwoVertexTriangleWithSigma{
	\draw[RedSigmaStyle, rounded corners=10pt] (1,-1.6) --
	(1.1,0) -- (1.5,1) -- (2.3,0.8) -- (2.5,-1.6) node[red,pos=0.6,right]{$\si$};
}

\newcommand\TikzPunctureClosestToKTwoVertexTriangle[1]{
\begin{tikzpicture}[scale={#1}]
\MacroPunctureClosestToKOrKOneTwoVertexTriangle
\draw (2.8,1.3) node[black] {$k+1$};
\end{tikzpicture}
}
\newcommand\TikzPunctureClosestToKTwoVertexTriangleWithSigma[1]{
\begin{tikzpicture}[scale={#1}]
\MacroPunctureClosestToKOrKOneTwoVertexTriangle
\MacroPunctureClosestToKOrKOneTwoVertexTriangleWithSigma
\draw (2.8,1.3) node[black] {$k+1$};
\end{tikzpicture}
}
\newcommand\TikzPunctureClosestToKOneTwoVertexTriangle[1]{
\begin{tikzpicture}[scale={#1}]
\begin{scope}[xscale=-1]\MacroPunctureClosestToKOrKOneTwoVertexTriangle\end{scope}
\draw (-2.7,1.3) node[black] {$k$};
%	\begin{scope}[rotate=90]\MacroTwoVertexTriangle\end{scope}
%	\draw (-1.5,-1) node[black]{$\tg{k}$};
%	\draw (0.2,0) node[olive] {$v$};
%	\draw[fill=violet] (-1.5,0) circle(\TikzPunctureActualSize) node[violet, above] {$\puncture$};
\end{tikzpicture}
}
\newcommand\TikzPunctureClosestToKOneTwoVertexTriangleWithSigma[1]{
\begin{tikzpicture}[scale={#1}]
	\begin{scope}[xscale=-1]
	\MacroPunctureClosestToKOrKOneTwoVertexTriangle
	\MacroPunctureClosestToKOrKOneTwoVertexTriangleWithSigma
	\end{scope}
	\draw (-2.7,1.3) node[black] {$k$};
\end{tikzpicture}
}


\newcommand\MacroPunctureClosestToTGKTwoVertexTriangle{
	\begin{scope}\MacroTwoVertexTriangle\end{scope}
	\draw (-0.6,2) node[black]{$\tg{k}$};
	\draw (0,3.8) node[olive] {$v$};
	\draw[fill=violet] (0,1.5) circle(\TikzPunctureActualSize) node[violet, below=2pt] {$\puncture$};
	\draw[CyanGammaStyle] (-1.4,2.9) -- (1.4,2.9) node[pos=0.9, above]{$\ga$};
}
\newcommand\TikzPunctureClosestToTGKTwoVertexTriangle[1]{
\begin{tikzpicture}[scale={#1}]
	\MacroPunctureClosestToTGKTwoVertexTriangle
\end{tikzpicture}
}
\newcommand\MacroDrawGammaSigmaCrossingTwiceOpeningLeft{
	\draw[RedSigmaStyle, rounded corners=10pt] (-1.5,2.3) -- (0.6,2.5) -- (1,1.8) -- (0.7,1) -- (-1.5,1);
}
\newcommand\TikzPunctureClosestToTGKTwoVertexTriangleWithSigma[1]{
\begin{tikzpicture}[scale={#1}]
	\MacroPunctureClosestToTGKTwoVertexTriangle
	\MacroDrawGammaSigmaCrossingTwiceOpeningLeft
\end{tikzpicture}
}
\newcommand\TikzPunctureClosestToTGKTwoVertexTriangleWithSigmaOpeningRight[1]{
\begin{tikzpicture}[scale={#1}]
	\MacroPunctureClosestToTGKTwoVertexTriangle
	\begin{scope}[xscale=-1]\MacroDrawGammaSigmaCrossingTwiceOpeningLeft\end{scope}
\end{tikzpicture}
}

\newcommand\MacroTriangleLabelsKAndKOne{
\draw (-0.55*\TikzBase,0.7*\TikzHeight) node[black] {$k$};
\draw (0.8*\TikzBase,0.7*\TikzHeight) node[black] {$k+1$};
\draw(0,0.15*\TikzHeight) node[black] {$\tg{k}$};
}

\newcommand\MacroTriangleWithGamma{
\newcommand\TikzHeight{1.8}
\newcommand\TikzBase{1.4}
\draw[thick] (-\TikzBase,0) -- (\TikzBase,0) -- (0,\TikzHeight)  -- (-\TikzBase,0);
\draw [fill=olive] (-\TikzBase,0) circle (.6ex);
\draw [fill=olive] (\TikzBase,0)  circle (.6ex);
\draw [fill=olive] (0,\TikzHeight)  circle (.6ex) node[olive,above=2pt] {$v$};
\draw[CyanGammaStyle] (-0.8*\TikzBase,0.8*\TikzHeight) -- (0.8*\TikzBase,0.8*\TikzHeight) node[cyan, pos=0.2, above] {$\ga$};
}
\newcommand\TikzPunctureClosestToK[1]{
\begin{tikzpicture}[scale={#1}]
\MacroTriangleWithGamma
\draw[fill=violet] (-0.9*\TikzBase,0.4*\TikzHeight) circle (\TikzPunctureActualSize);
\MacroTriangleLabelsKAndKOne
\end{tikzpicture}
}
\newcommand\TikzPunctureClosestToKOne[1]{
\begin{tikzpicture}[scale={#1}]
\MacroTriangleWithGamma
\draw[fill=violet] (0.9*\TikzBase,0.4*\TikzHeight) circle (\TikzPunctureActualSize);
\MacroTriangleLabelsKAndKOne
\end{tikzpicture}
}
\newcommand\TikzPunctureClosestToTGK[1]{
\begin{tikzpicture}[scale={#1}]
\MacroTriangleWithGamma
\draw[fill=violet] (0,-0.2*\TikzBase) circle (\TikzPunctureActualSize);
\MacroTriangleLabelsKAndKOne
\end{tikzpicture}
}

\newcommand\MacroOrdinaryTriangleSigmaClosingLeft{
\coordinate (P) at (-0.9*\TikzBase,0.4*\TikzHeight) ;
\draw[fill=violet] (P) circle (\TikzPunctureActualSize);
\draw[RedSigmaStyle, rounded corners=10pt] (0,-0.5) -- (-1.2*\TikzBase,0.4*\TikzHeight) 
-- (-0.8*\TikzBase,0.7*\TikzHeight)  -- (0.5*\TikzBase,-0.5) ;
\draw[red] (0*\TikzBase,0.3*\TikzHeight) node {$\si$};
}

\newcommand\TikzPunctureClosestToKWithSigma[1]{
\begin{tikzpicture}[scale={#1}]
\MacroTriangleWithGamma
\MacroOrdinaryTriangleSigmaClosingLeft
\end{tikzpicture}
}
\newcommand\TikzPunctureClosestToKOneWithSigma[1]{
\begin{tikzpicture}[scale={#1}]
\MacroTriangleWithGamma
\begin{scope}[xscale=-1]\MacroOrdinaryTriangleSigmaClosingLeft\end{scope}
\end{tikzpicture}
}
\newcommand\MacroOrdinaryTriangleSigmaOpeningRight{
\draw[fill=violet] (0,-0.2*\TikzBase) circle (\TikzPunctureActualSize);
\draw[RedSigmaStyle, rounded corners=10pt]  (0.9*\TikzBase,0.7*\TikzHeight) --(0.5*\TikzBase,0.6*\TikzHeight)  --   
 (-0.4*\TikzBase,-0.3*\TikzHeight) --  (0.3*\TikzBase,-0.3*\TikzHeight) -- 
 (0.7*\TikzBase,0.3*\TikzHeight) -- (1*\TikzBase,0.4*\TikzHeight);
 \node[red] at (0,0.45*\TikzBase) {$\si$};
 }

\newcommand\TikzPunctureClosestToTGKWithSigma[1]{
\begin{tikzpicture}[scale={#1}]
\MacroTriangleWithGamma
\MacroOrdinaryTriangleSigmaOpeningRight
\end{tikzpicture}
}
\newcommand\TikzPunctureClosestToTGKWithSigmaOpeningLeft[1]{
\begin{tikzpicture}[scale={#1}]
\MacroTriangleWithGamma
\begin{scope}[xscale=-1]\MacroOrdinaryTriangleSigmaOpeningRight\end{scope}
\end{tikzpicture}
}
%%%
%%%
\newcommand\MacroNooseTriangulation[1]{
\newcommand\Tikzn{2}
\begin{scope}[#1]
\coordinate (P) at (0,0);
%\node at (1.3*\Tikzn,0) {$b_1$}; \node at (-1.3*\Tikzn,0) {$b_3$};
%\node at (0,1.3*\Tikzn) {$b_2$}; \node at (0,-1.3*\Tikzn) {$b_4$};
\draw (\Tikzn,\Tikzn) -- (\Tikzn,-\Tikzn) -- (-\Tikzn,-\Tikzn) -- (-\Tikzn,\Tikzn)  -- cycle;
\foreach \pos in {(\Tikzn,\Tikzn),(\Tikzn,-\Tikzn),(-\Tikzn,\Tikzn),(-\Tikzn,-\Tikzn)}
		{\draw (P) -- \pos;
		\draw[fill=olive] \pos circle (1.3*\TikzPointActualSize);}
\draw[fill=violet] (P) circle (1.3*\TikzPunctureActualSize);
\end{scope}
}

\newcommand\TikzNooseGammaIdealTriangulation[1]{
\begin{tikzpicture}[scale={#1}]
\MacroNooseTriangulation{black}
\draw[RedSigmaStyle, NooseArrowAtPosStyleTwo,rounded corners=8pt] (\Tikzn,-\Tikzn) -- (0.6*\Tikzn,0) -- (0.1*\Tikzn,0.7*\Tikzn)
-- (-0.5*\Tikzn,0.5*\Tikzn)
-- (-0.6*\Tikzn,-0.3*\Tikzn) -- (0,-0.6*\Tikzn) -- (\Tikzn,-\Tikzn);
\node at (0.4*\Tikzn,0.4*\Tikzn) [right]{\TikzPosterTpathSixNineFontSize $1$};
\node at (-0.4*\Tikzn,0.4*\Tikzn) [above]{\TikzPosterTpathSixNineFontSize $2$};
\node at (-0.6*\Tikzn,-0.6*\Tikzn) [right]{\TikzPosterTpathSixNineFontSize $3$};
\node at (0.4*\Tikzn,-0.4*\Tikzn) [left]{\TikzPosterTpathSixNineFontSize $\tau$};
\node at (1.3*\Tikzn,0) {\TikzPosterTpathSixNineFontSize $b_1$}; \node at (-1.3*\Tikzn,0) {\TikzPosterTpathSixNineFontSize $b_3$};
\node at (0,1.3*\Tikzn) {\TikzPosterTpathSixNineFontSize $b_2$}; \node at (0,-1.3*\Tikzn) {\TikzPosterTpathSixNineFontSize $b_4$};
\node at (-0.7*\Tikzn,0) [red] {\TikzPosterTpathSixNineFontSize $\ga$};
\end{tikzpicture}
}
\newcommand\MacroDrawBlueDot[2]{
\node[style=thick,fill=blue,blue,circle,minimum width=2.5pt,inner sep=0pt,draw] at (#1,#2) {};
}
\newcommand\MacroNooseTPathOne{
\draw [TPathStyle, PathArrowAtPosStyleFourSmall] (\Tikzn,-\Tikzn) -- (\Tikzn,\Tikzn); 
\draw [TPathStyle,rounded corners=20pt,PathArrowAtPosStyleFourSmall] (0.9*\Tikzn,0.9*\Tikzn) -- (P) -- (-0.9*\Tikzn,0.9*\Tikzn);
\MacroDrawBlueDot{0}{0.3*\Tikzn}
\draw [TPathStyle,rounded corners=20pt] (-1*\Tikzn,0.9*\Tikzn) -- (P) -- (-\Tikzn,-\Tikzn);
\MacroDrawBlueDot{-0.3*\Tikzn}{0}
\draw [TPathStyle,rounded corners=20pt,PathArrowAtPosStyleSevenSmall] (-0.9*\Tikzn,-0.9*\Tikzn) -- (P) -- (\Tikzn,-\Tikzn) ;
\MacroDrawBlueDot{0}{-0.3*\Tikzn}
}
\newcommand\MacroNooseTpathDrawLabels{
\node at (0.5*\Tikzn,0.5*\Tikzn) [right]{\TikzPosterTpathSixNineFontSize $1$};
\node at (-0.5*\Tikzn,0.5*\Tikzn) [left]{\TikzPosterTpathSixNineFontSize $2$};
\node at (-0.7*\Tikzn,-0.7*\Tikzn) [right]{\TikzPosterTpathSixNineFontSize $3$};
\node at (0.7*\Tikzn,-0.7*\Tikzn) [left]{\TikzPosterTpathSixNineFontSize $\tau$};
\begin{scope}[transparent]
\node at (1.3*\Tikzn,0) {\TikzPosterTpathSixNineFontSize $b_1$}; 
\node at (0,1.3*\Tikzn) {\TikzPosterTpathSixNineFontSize $b_2$};
\node at (-1.3*\Tikzn,0) {\TikzPosterTpathSixNineFontSize $b_3$};
\node at (0,-1.3*\Tikzn) {\TikzPosterTpathSixNineFontSize $b_4$};
\end{scope}
}

\newcommand\TikzNooseTpathOneBad[1]{
\begin{tikzpicture}[scale={#1}]
\MacroNooseTriangulation{gray}
\MacroNooseTpathDrawLabels
%\MacroNooseTPathOne
\coordinate (P1) at (0,0.3*\Tikzn);
\draw [TPathStyle, PathArrowAtPosStyleFourSmall] (\Tikzn,-\Tikzn) -- (\Tikzn,\Tikzn); 
\draw [TPathStyle,rounded corners=5pt,PathArrowAtPosStyleSixSmall] (0.99*\Tikzn,0.99*\Tikzn) -- (P1);
\draw [TPathStyle,rounded corners=5pt,PathArrowAtPosStyleFourSmall] (P1) --  (-0.2*\Tikzn,0.8*\Tikzn) -- (-0.99*\Tikzn,0.99*\Tikzn);
\MacroDrawBlueDot{0}{0.3*\Tikzn}
\draw [TPathStyle,rounded corners=20pt,PathArrowAtPosStyleSixSmall] (-0.9*\Tikzn,0.9*\Tikzn) -- (P1) ;
\draw [TPathStyle,rounded corners=20pt,PathArrowAtPosStyleFourSmall] 
(P1) -- (-\Tikzn,-\Tikzn);
\draw [TPathStyle,rounded corners=20pt,PathArrowAtPosStyleSevenSmall] (-0.9*\Tikzn,-0.9*\Tikzn) -- (P) -- (\Tikzn,-\Tikzn) ;
\MacroDrawBlueDot{0}{-0.3*\Tikzn}

\node at (1.3*\Tikzn,0) {\TikzPosterTpathSixNineFontSize $b_1$}; 
\node at (1.8*\Tikzn,0) {}; \node at (-1.8*\Tikzn,0) {}; 
\end{tikzpicture}
}

\newcommand\TikzPosterNooseTpathOneFourCaptionLocation{(0,-3.9)}

\newcommand\TikzNooseTpathOne[2]{
\begin{tikzpicture}[scale={#1}]
\MacroNooseTriangulation{gray}

\MacroNooseTpathDrawLabels
\MacroNooseTPathOne
\node at (1.3*\Tikzn,0) {\TikzPosterTpathSixNineFontSize $b_1$}; 
\node at \TikzPosterNooseTpathOneFourCaptionLocation {\TikzPosterTpathSixNineCaptionSize #2};
\end{tikzpicture}
}
\newcommand\TikzNooseTpathTwo[2]{
\begin{tikzpicture}[scale={#1}]
\MacroNooseTriangulation{gray}
%\draw [TPathStyle,PathArrowAtPosStyleFour] (\Tikzn,\Tikzn) -- (-\Tikzn,\Tikzn); 
\begin{scope}[rotate=90]
\MacroNooseTPathOne
\end{scope}
\MacroNooseTpathDrawLabels
\node at (0,1.3*\Tikzn) {\TikzPosterTpathSixNineFontSize $b_2$};
%\MacroDrawBlueDot{0.2*\Tikzn}{0}
%\MacroDrawBlueDot{-0.2*\Tikzn}{0}
%\MacroDrawBlueDot{0}{-0.2*\Tikzn}
\node at \TikzPosterNooseTpathOneFourCaptionLocation {\TikzPosterTpathSixNineCaptionSize #2};
\end{tikzpicture}
}
\newcommand\TikzNooseTpathThree[2]{
\begin{tikzpicture}[scale={#1}]
\MacroNooseTriangulation{gray}
%\draw [TPathStyle, PathArrowAtPosStyleFour] (-\Tikzn,\Tikzn) -- (-\Tikzn,-\Tikzn); 
\begin{scope}[rotate=180]
\MacroNooseTPathOne
\end{scope}
\MacroNooseTpathDrawLabels
\node at (-1.3*\Tikzn,0) {\TikzPosterTpathSixNineFontSize $b_3$};
%\MacroDrawBlueDot{0.2*\Tikzn}{0}
%\MacroDrawBlueDot{0}{0.2*\Tikzn}
%\MacroDrawBlueDot{0}{-0.2*\Tikzn}
\node at \TikzPosterNooseTpathOneFourCaptionLocation {\TikzPosterTpathSixNineCaptionSize #2};
\end{tikzpicture}
}
\newcommand\TikzNooseTpathFour[2]{
\begin{tikzpicture}[scale={#1}]
\MacroNooseTriangulation{gray}
%\draw [TPathStyle, PathArrowAtPosStyleFour] (-\Tikzn,-\Tikzn) -- (\Tikzn,-\Tikzn); 
\begin{scope}[rotate=-90]
\MacroNooseTPathOne
\end{scope}
\MacroNooseTpathDrawLabels
\node at (0,-1.3*\Tikzn) {\TikzPosterTpathSixNineFontSize $b_4$};
\node at \TikzPosterNooseTpathOneFourCaptionLocation {\TikzPosterTpathSixNineCaptionSize #2};
\end{tikzpicture}
}


%%%
%%%
\newcommand\MacroNotNooseDrawArcThree[1]{
\draw [rounded corners=15pt,#1] (-\Tikzn,\Tikzn) -- (-0.5*\Tikzn, -0.5*\Tikzn) -- (\Tikzn,-\Tikzn);
}
\newcommand\MacroNotNooseTriangulation[1]{
\newcommand\Tikzn{2}
\begin{scope}[#1]
\coordinate (P) at (0,0);
%\node at (1.3*\Tikzn,0) {$b_1$}; \node at (-1.3*\Tikzn,0) {$b_3$};
%\node at (0,1.3*\Tikzn) {$b_2$}; \node at (0,-1.3*\Tikzn) {$b_4$};
\draw (\Tikzn,\Tikzn) -- (\Tikzn,-\Tikzn) -- (-\Tikzn,-\Tikzn) -- (-\Tikzn,\Tikzn)  -- cycle;

\MacroNotNooseDrawArcThree{}
%\draw [rounded corners=15pt] (-\Tikzn,\Tikzn) -- (-0.5*\Tikzn, -0.5*\Tikzn) -- (\Tikzn,-\Tikzn);

\foreach \pos in {(\Tikzn,\Tikzn),(\Tikzn,-\Tikzn),
(-\Tikzn,\Tikzn)%,(-\Tikzn,-\Tikzn)
}
		{\draw (P) -- \pos;
		\draw[fill=olive] \pos circle (1.3*\TikzPointActualSize);}
\draw[fill=violet] (P) circle (1.3*\TikzPunctureActualSize);
\draw[fill=olive] (-\Tikzn,-\Tikzn) circle (1.3*\TikzPointActualSize);
\end{scope}
}

\newcommand\TikzNotNooseGammaIdealTriangulation[1]{
\begin{tikzpicture}[scale={#1}]
\MacroNotNooseTriangulation{black}
\draw[RedSigmaStyle, NooseArrowAtPosStyleTwo,rounded corners=8pt] (\Tikzn,-\Tikzn) -- (0.6*\Tikzn,0) -- (0.1*\Tikzn,0.7*\Tikzn)
-- (-0.5*\Tikzn,0.5*\Tikzn)
%-- (-0.6*\Tikzn,-0.3*\Tikzn) -- (0,-0.6*\Tikzn) 
-- (-\Tikzn,-\Tikzn);
\node at (0.4*\Tikzn,0.4*\Tikzn) [right]{\TikzPosterTpathSixNineFontSize $1$};
\node at (-0.4*\Tikzn,0.5*\Tikzn) [above]{\TikzPosterTpathSixNineFontSize $2$};
\node at (-0.5*\Tikzn,-0.5*\Tikzn) {\TikzPosterTpathSixNineFontSize $3$};
\node at (0.5*\Tikzn,-0.5*\Tikzn) [left]{\TikzPosterTpathSixNineFontSize $0$};
\node at (1.3*\Tikzn,0) {\TikzPosterTpathSixNineFontSize $b_1$}; \node at (-1.3*\Tikzn,0) {\TikzPosterTpathSixNineFontSize $b_3$};
\node at (0,1.3*\Tikzn) {\TikzPosterTpathSixNineFontSize $b_2$}; \node at (0,-1.3*\Tikzn) {\TikzPosterTpathSixNineFontSize $b_4$};
\node[rotate=50] at (0.2*\Tikzn,0.74*\Tikzn) [red] {\TikzPosterTpathSixNineFontSize $\ga$};
\end{tikzpicture}
}
%\newcommand\MacroDrawBlueDot[2]{
%\node[style=thick,fill=blue,blue,circle,minimum width=2.5pt,inner sep=0pt,draw] at (#1,#2) {};
%}
%\newcommand\MacroNotNooseTPathOne{
%\draw [TPathStyle, PathArrowAtPosStyleFourSmall] (\Tikzn,-\Tikzn) -- (\Tikzn,\Tikzn); 
%\draw [TPathStyle,rounded corners=20pt,PathArrowAtPosStyleFourSmall] (0.9*\Tikzn,0.9*\Tikzn) -- (P) -- (-0.9*\Tikzn,0.9*\Tikzn);
%\MacroDrawBlueDot{0}{0.3*\Tikzn}
%\draw [TPathStyle,rounded corners=20pt] (-1*\Tikzn,0.9*\Tikzn) -- (P) -- (-\Tikzn,-\Tikzn);
%\MacroDrawBlueDot{-0.3*\Tikzn}{0}
%\draw [TPathStyle,rounded corners=20pt,PathArrowAtPosStyleSevenSmall] (-0.9*\Tikzn,-0.9*\Tikzn) -- (P) -- (\Tikzn,-\Tikzn) ;
%\MacroDrawBlueDot{0}{-0.3*\Tikzn}
%}
\newcommand\MacroNotNooseTpathDrawLabels{
\node at (0.4*\Tikzn,0.4*\Tikzn) [right]{\TikzPosterTpathSixNineFontSize$1$};
\node at (-0.4*\Tikzn,0.5*\Tikzn) [above]{\TikzPosterTpathSixNineFontSize$2$};
\node at (-0.5*\Tikzn,-0.5*\Tikzn) {\TikzPosterTpathSixNineFontSize$3$};
\node at (0.5*\Tikzn,-0.5*\Tikzn) [right]{\TikzPosterTpathSixNineFontSize$0$};
\node at (1.3*\Tikzn,0) {}; \node at (-1.3*\Tikzn,0) {};
%\node at (0,1.3*\Tikzn) {$b_2$}; \node at (0,-1.3*\Tikzn) {$b_4$};
%
%\node at (0.5*\Tikzn,0.5*\Tikzn) [right]{\tiny$1$};
%\node at (-0.5*\Tikzn,0.5*\Tikzn) [left]{\tiny$2$};
%\node at (-0.7*\Tikzn,-0.7*\Tikzn) [right]{\tiny$3$};
%\node at (0.7*\Tikzn,-0.7*\Tikzn) [left]{\tiny$\tau$};
\begin{scope}[transparent]
\node at (1.3*\Tikzn,0)  {\TikzPosterTpathSixNineFontSize $b_1$}; 
\node at (0,1.3*\Tikzn) {\TikzPosterTpathSixNineFontSize $b_2$};
\node at (-1.3*\Tikzn,0) {\TikzPosterTpathSixNineFontSize $b_3$};
\node at (0,-1.3*\Tikzn) {\TikzPosterTpathSixNineFontSize $b_4$};
\end{scope}
}

\newcommand\TikzNotNooseTpathThreeBad[1]{
\begin{tikzpicture}[scale={#1}]
\MacroNotNooseTriangulation{gray}

\MacroNotNooseTpathDrawLabels

\begin{scope}[TPathStyle]
%\coordinate (P1) at (0.3*\Tikzn,0);
\coordinate (P1) at (-0.2*\Tikzn,-0.2*\Tikzn);

\draw[PathArrowAtPosStyleFourSmall] (\Tikzn,-\Tikzn) -- (P1); 
\draw[PathArrowAtPosStyleFourSmall, rounded corners=15pt] (P1)  -- (0.3*\Tikzn,-0.2*\Tikzn) -- (\Tikzn,0.95*\Tikzn);

\draw[rounded corners=15pt,PathArrowAtPosStyleSixSmall] (\Tikzn,0.95*\Tikzn) -- (-0.4*\Tikzn,0.1*\Tikzn) %--  (0.3*\Tikzn,-0.3*\Tikzn)  
-- (P1) ;

%\MacroDrawBlueDot{0.3*\Tikzn}{0}
\MacroDrawBlueDot{-0.2*\Tikzn}{-0.2*\Tikzn}

\draw[rounded corners=15pt,PathArrowAtPosStyleFourSmall] (P1) % -- (0,-0.6*\Tikzn) 
-- (-\Tikzn,\Tikzn);

\draw [TPathStyle, PathArrowAtPosStyleEnd] (-\Tikzn,\Tikzn) -- (-\Tikzn,-\Tikzn); \node[black] at (-1.3*\Tikzn,0) {\TikzPosterTpathSixNineFontSize $b_3$};

\end{scope}
%
%\draw [TPathStyle, PathArrowAtPosStyleFourSmall] (\Tikzn,-\Tikzn) -- (\Tikzn,\Tikzn); 
%\draw [TPathStyle,rounded corners=5pt,PathArrowAtPosStyleSixSmall] (0.99*\Tikzn,0.99*\Tikzn) -- (P1);
%\draw [TPathStyle,rounded corners=5pt,PathArrowAtPosStyleFourSmall] (P1) --  (-0.2*\Tikzn,0.8*\Tikzn) -- (-0.99*\Tikzn,0.99*\Tikzn);
%\MacroDrawBlueDot{0}{0.3*\Tikzn}
%\draw [TPathStyle,rounded corners=20pt,PathArrowAtPosStyleFourSmall] (-0.9*\Tikzn,0.9*\Tikzn) -- (P1) 
%(P1) -- (-\Tikzn,-\Tikzn);
%%\MacroDrawBlueDot{-0.3*\Tikzn}{0}
%\draw [TPathStyle,rounded corners=20pt,PathArrowAtPosStyleSevenSmall] (-0.9*\Tikzn,-0.9*\Tikzn) -- (P) -- (\Tikzn,-\Tikzn) ;
%\MacroDrawBlueDot{0}{-0.3*\Tikzn}
\node at (-2.3*\Tikzn,0) {}; \node at (2.3*\Tikzn,0) {}; 
\end{tikzpicture}
}

\newcommand\TikzPosterNotNooseTpathOneThreeCaptionLocation{(0,-4.5)}

\newcommand\TikzNotNooseTpathOne[2]{
\begin{tikzpicture}[scale={#1}]
\MacroNotNooseTriangulation{gray}

\MacroNotNooseTpathDrawLabels
%\MacroNotNooseTPathOne
\begin{scope}[TPathStyle]
\draw [PathArrowAtPosStyleFourSmall] (\Tikzn,-\Tikzn) -- (\Tikzn,\Tikzn); \node[black] at (1.3*\Tikzn,0) {\TikzPosterTpathSixNineFontSize $b_1$}; 
\draw [TPathStyle,rounded corners=20pt,PathArrowAtPosStyleFourSmall] (0.9*\Tikzn,0.9*\Tikzn) -- (P) -- (-0.9*\Tikzn,0.9*\Tikzn);
\MacroDrawBlueDot{0}{0.3*\Tikzn}

\draw [PathArrowAtPosStyleEightSmall] (-\Tikzn,\Tikzn) -- 
%(-0.2*\Tikzn,0.2*\Tikzn) --
(-0.2*\Tikzn,-0.1*\Tikzn) ;
\draw [rounded corners=5pt]
(-0.2*\Tikzn,-0.2*\Tikzn) -- 
(-0.5*\Tikzn,-0.1*\Tikzn) --
(-\Tikzn,\Tikzn);

\MacroDrawBlueDot{-0.2*\Tikzn}{-0.2*\Tikzn}

\MacroNotNooseDrawArcThree{PathArrowAtPosStyleSevenSmall}
\draw [PathArrowAtPosStyleEnd]  (\Tikzn,-\Tikzn) -- (-\Tikzn,-\Tikzn); \node[black] at (0,-1.3*\Tikzn) {\TikzPosterTpathSixNineFontSize $b_4$};
\end{scope}
\node at \TikzPosterNotNooseTpathOneThreeCaptionLocation {\TikzPosterTpathSixNineCaptionSize #2};
\end{tikzpicture}
}
\newcommand\TikzNotNooseTpathTwo[2]{
\begin{tikzpicture}[scale={#1}]
\MacroNotNooseTriangulation{gray}
\MacroNotNooseTpathDrawLabels
\begin{scope}[TPathStyle]
\draw [rounded corners=20pt,PathArrowAtPosStyleSevenSmall] (\Tikzn,-\Tikzn) -- (P) -- (\Tikzn,\Tikzn);
\MacroDrawBlueDot{0.3*\Tikzn}{0}
\draw [PathArrowAtPosStyleFourSmall] (\Tikzn,\Tikzn) -- (-\Tikzn,\Tikzn); \node[black] at (0,1.3*\Tikzn) {\TikzPosterTpathSixNineFontSize $b_2$};
\draw [rounded corners=5pt,PathArrowAtPosStyleSevenSmall] (-\Tikzn,\Tikzn) -- 
(-0.2*\Tikzn,0.2*\Tikzn) --
(-0.2*\Tikzn,-0.1*\Tikzn) ;
\draw [rounded corners=5pt,PathArrowAtPosStyleFourSmall]
(-0.2*\Tikzn,-0.1*\Tikzn) -- 
(-0.5*\Tikzn,0) --
(-\Tikzn,\Tikzn);
\MacroDrawBlueDot{-0.2*\Tikzn}{-0.1*\Tikzn}
\MacroNotNooseDrawArcThree{PathArrowAtPosStyleSevenSmall}
\draw [PathArrowAtPosStyleEnd]  (\Tikzn,-\Tikzn) -- (-\Tikzn,-\Tikzn); \node[black] at (0,-1.3*\Tikzn) {\TikzPosterTpathSixNineFontSize $b_4$};
\end{scope}
\node at \TikzPosterNotNooseTpathOneThreeCaptionLocation {\TikzPosterTpathSixNineCaptionSize #2};
\end{tikzpicture}
}
\newcommand\TikzNotNooseTpathThree[2]{
\begin{tikzpicture}[scale={#1}]
\MacroNotNooseTriangulation{gray}
\MacroNotNooseTpathDrawLabels
\begin{scope}[TPathStyle]
\draw [rounded corners=20pt,PathArrowAtPosStyleFourSmall] (\Tikzn,-\Tikzn) -- (P) -- (\Tikzn,0.95*\Tikzn);
\MacroDrawBlueDot{0.3*\Tikzn}{0}
\draw [TPathStyle,rounded corners=20pt,PathArrowAtPosStyleEightSmall] (0.9*\Tikzn,0.99*\Tikzn) -- (P) -- (-0.9*\Tikzn,0.9*\Tikzn);
\MacroDrawBlueDot{0}{0.3*\Tikzn}
\draw [TPathStyle, PathArrowAtPosStyleEnd] (-\Tikzn,\Tikzn) -- (-\Tikzn,-\Tikzn); \node[black] at (-1.3*\Tikzn,0) {\TikzPosterTpathSixNineFontSize $b_3$};
\end{scope}
\node at \TikzPosterNotNooseTpathOneThreeCaptionLocation {\TikzPosterTpathSixNineCaptionSize #2};
\end{tikzpicture}
}

%% current ends

%% starting for poster
%%%%
%%%% T2 triangulation

%%%%%%%%%%%%%  
\newcommand \MacroTtwoTriangulationCoordinatesForDrawingQuiver{
 \coordinate (t) at (0,0); 
 \coordinate(s) at (4,0);
  \coordinate(vL) at (0,4);  \coordinate(vR) at (4,4);
   \coordinate(x) at (0,4);  \coordinate(z) at (4,4);
 }
\newcommand \MacroTtwoTriangulationVerticesForDrawingQuiver{
 \draw [fill=green] (0,0) circle (.3ex);
\draw [fill=green] (0,4) circle (.3ex);
\draw [fill=green] (4,0) circle (.3ex);
\draw [fill=green] (4,4) circle (.3ex);
\draw [fill=violet] (1.2,2.2) circle (.3ex); %puncture
  }
%%%%%%%%%%%

\newcommand  \TikzTtwoTaggedNextToQuiver[1]{
\begin{tikzpicture}[scale=#1] %% T2 triangulation tagged
	\begin{scope}[rotate=180]
\MacroTtwoTriangulationCoordinatesForDrawingQuiver
\newcommand\localfontsize{\small}
\draw[style = thick]  (0,0) rectangle (4,4); 
\draw [style = thick] (0,0) ..  controls(3.5,0.4) and (3.5,2) .. (4,4) node[pos=0.5,above]{\localfontsize $1$};  % the right most peripheral arc
\draw [style = thick] (0,0) ..  controls (3.7,1) and (3.8,3.9)   .. (0,4)  node[pos=0.5,left]{\localfontsize $2$}; % the middle arc 
\draw[style = thick] (0,4) .. controls (0.8,1) and (1,2) .. (1.2,2.2) node[pos=0.3,right]{\localfontsize $r$};  % the bridging (tagged plain) arc
\draw[style = ultra thick, gray] (0,4) -- (1.2,2.2) node[pos=0.3, left] {\localfontsize $r$};  % r bowtie
% the briding (tagged notched) arc
\path[fill, gray] (1,2.5) -- (1.1,2.8) -- (1.3,2.5) -- cycle; % the notching (right)
\path[fill, gray] (1,2.5) -- (0.9,2.2) -- (0.7,2.5) -- cycle; % the notching (left)

\MacroTtwoTriangulationVerticesForDrawingQuiver
	\end{scope}
\end{tikzpicture}
\begin{tikzpicture}[scale=#1] %% T2 triangulation tagged
	\begin{scope}[rotate=180]
\MacroTtwoTriangulationCoordinatesForDrawingQuiver

\draw[style = thick]  (0,0) rectangle (4,4); 
\draw [style = thick] (0,0) ..  controls(3.5,0.4) and (3.5,2) .. (4,4) node[pos=0.5,above]{$1$};  % the right most peripheral arc
\draw [style = thick] (0,0) ..  controls (3.7,1) and (3.8,3.9)   .. (0,4)  node[pos=0.5,left]{$2$}; % the middle arc 
\draw[style = thick] (0,4) .. controls (0.8,1) and (1,2) .. (1.2,2.2) node[pos=0.3,right]{$r$};  % the bridging (tagged plain) arc
\draw[style = ultra thick, gray] (0,4) -- (1.2,2.2) node[pos=0.3, left] {$r$};  % r bowtie
% the briding (tagged notched) arc
\path[fill, gray] (1,2.5) -- (1.1,2.8) -- (1.3,2.5) -- cycle; % the notching (right)
\path[fill, gray] (1,2.5) -- (0.9,2.2) -- (0.7,2.5) -- cycle; % the notching (left)
\begin{scope}[->, style=very thick, dashed, red]
\draw[rounded corners = 10 pt] (2.5,1.5) -- (0.7,0.8) -- (0.8,1.9); % red arrow two to r ( no bow tie )
\draw[rounded corners = 10 pt] (2.5,1.5) -- (1.5,3.5) -- (0.75,3); % two to r (with bow tie)
\draw (2.5,1.5) -- (3.2,1.5); % two to one
\end{scope}
\MacroTtwoTriangulationVerticesForDrawingQuiver
	\end{scope}
\end{tikzpicture}
 }
 
\newcommand \MacroTikzToneWithBowtiesNoLabel{ %Draw the triangulation where the arcs are labeled 1,2,3,4
 
 \begin{scope}[style=thick]
\draw[style = thick]  (0,0) rectangle (4,4);
\draw [style = thick] (0,0) ..  controls(3.5,0.4) and (3.5,2) .. %node[pos =0.4,right=2pt] {$1$} 
 (4,4);
\draw [style = thick] (0,0) ..  controls (3.7,1) and (3.8,3.9)   .. %node[pos =0.4,left=2pt] {$2$}  
(0,4);
\draw[style = thick] (0,4) -- (1.2,2.2);
\draw[style = thick] (1.2,2.2) --  (0,0);
\draw (1,2.5) -- (1.1,2.8) -- (1.3,2.5) -- cycle; % the top notching (right)
\draw (1,2.5) -- (0.9,2.2) -- (0.7,2.5) -- cycle; % the top notching (left)
\draw (1,1.9) -- (1.1,1.6) -- (1.3,1.9) -- cycle; % the bottom notching (right)
\draw (1,1.9) -- (0.9,2.2) -- (0.7,1.9) -- cycle; % the bottom notching (left)
\end{scope}
\draw [fill=violet] (1.2,2.2) circle (.3ex); %puncture
\draw [fill=green] (0,0) circle (.3ex);
\draw [fill=green] (0,4) circle (.3ex);
\draw [fill=green] (4,0) circle (.3ex);
\draw [fill=green] (4,4) circle (.3ex);
}

\newcommand  \TikzToneBowtiesOnly[1]{
\begin{tikzpicture} [scale=#1]
\MacroTikzToneWithBowtiesNoLabel %Draw the triangulation where the arcs are labeled 1,2,3,4
\end{tikzpicture}
}

% Just a once-punctured square
\newcommand \OncePuncturedSquare[1]{
\begin{tikzpicture} [scale =#1]
	\MacroTtwoTriangulationCoordinatesForDrawingQuiver
	\draw[style = thick]  (0,0) rectangle (4,4);
	\MacroTtwoTriangulationVerticesForDrawingQuiver
\end{tikzpicture}
}

\newcommand  \TikzTtwoTaggedTriangulation[1]{ 
\begin{tikzpicture}[scale=#1] %% T2 triangulation tagged
\MacroTtwoTriangulationCoordinatesForDrawingQuiver

\begin{scope}[style=thick]
\draw (0,0) rectangle (4,4); 
\draw (0,0) ..  controls(3.5,0.4) and (3.5,2) .. (4,4); % the right most peripheral arc
\draw (0,0) ..  controls (3.7,1) and (3.8,3.9)   .. (0,4); % the middle arc 
\draw (0,4) .. controls (0.8,1) and (1,2) .. (1.2,2.2);  % the radius (tagged plain) arc
\draw (0,4) -- (1.2,2.2);% node[pos=0.5, right=1pt] {$\iota(\ell)$};  
\draw (1,2.5) -- (1.1,2.8) -- (1.3,2.5) -- cycle; % the notching (right)
\draw (1,2.5) -- (0.9,2.2) -- (0.7,2.5) -- cycle; % the notching (left)
\end{scope}
\MacroTtwoTriangulationVerticesForDrawingQuiver
\end{tikzpicture}
 }

 \newcommand \MacroTikzToneNoLabel{ %Draw the triangulation where the arcs are labeled 1,2,3,4
\draw[style = thick]  (0,0) rectangle (4,4);
\draw [style = thick] (0,0) ..  controls(3.5,0.4) and (3.5,2) .. %node[pos =0.4,right=2pt] {$1$} 
 (4,4);
\draw [style = thick] (0,0) ..  controls (3.7,1) and (3.8,3.9)   .. %node[pos =0.4,left=2pt] {$2$}  
(0,4);
\draw[style = thick] (0,4) -- %node[pos =0.3,right=1pt] {$3$}  
(1.2,2.2);
\draw[style = thick] (1.2,2.2) -- %node[pos =0.4,right=1pt] {$4$}  
(0,0);
\draw [fill=violet] (1.2,2.2) circle (.3ex); %puncture
\draw [fill=green] (0,0) circle (.3ex);
\draw [fill=green] (0,4) circle (.3ex);
\draw [fill=green] (4,0) circle (.3ex);
\draw [fill=green] (4,4) circle (.3ex);
 }
 \newcommand  \TikzToneOnly[1]{
 \begin{tikzpicture} [scale=#1]
\MacroTikzToneNoLabel 
\end{tikzpicture}
  }
\newcommand  \TikzTtwoIdealTriangulation[1]{
 \begin{tikzpicture} [scale =#1]%% T2 triangulation tagged (IDEAL)
\MacroTtwoTriangulationCoordinatesForDrawingQuiver

\draw[style = thick]  (0,0) rectangle (4,4);
\draw [style = thick] (0,0) ..  controls(3.5,0.4) and (3.5,2) .. (4,4);
\draw [style = thick] (0,0) ..  controls (3.7,1) and (3.8,3.9)   .. (0,4);
\draw [style = very thick] (0,4) ..  controls (0.7,0.1) and (3.9,1.4)   .. (0,4);% node[pos=0.87, right] {$\ell$}; %loop
\draw[style = thick] (0,4) -- (1.2,2.2);
\MacroTtwoTriangulationVerticesForDrawingQuiver
\end{tikzpicture}
 }

\newcommand\TikzNoose[1]{
\begin{tikzpicture}[scale={#1}, rotate=90]
\draw[style = very thick, rounded corners = 5pt]	
	(0,0) .. controls (-0.5,1) and (-1,2) .. (0,3) .. 
	controls (1,2)  and (0.5,1) .. (0,0); %node[pos=0.5, right] {$\ell$};
	\draw [fill=olive] (0,0) circle (.6ex);
\draw [fill=violet] (0,2) circle (1.2*\TikzPunctureActualSize);
\draw (0,4) node {$\ell$};
\end{tikzpicture}
}


\newcommand\MacroAffineDSix{
	\begin{scope}%[rotate=180]
%\MacroTtwoTriangulationCoordinatesForDrawingQuiver
\newcommand\TikzH{4}
\newcommand\TikzW{6}
\coordinate (topright) at (\TikzW,\TikzH);
\draw[style = thick]  (0,0) rectangle (topright); 
\draw [style = thick] (0,0) ..  controls(5.5,0.4) and (1,3.9) .. (topright) node[pos=0.5,above]{$1$};  % the right most peripheral arc
\draw [style = thick] (0,0) ..  controls (3,1) and (3,3.9)   .. (0,\TikzH)  node[pos=0.7,right]{$2$}; 
\draw[style = thick] (0,\TikzH) .. controls (0.8,1) and (1,2) .. (1.2,2.2) node[pos=0.3,right]{$r$};  % the bridging (tagged plain) arc
\draw[style = ultra thick, gray] (0,4) -- (1.2,2.2) node[pos=0.3, left] {$r$};  % r bowtie
% the briding (tagged notched) arc
\path[fill, gray] (1,2.5) -- (1.1,2.8) -- (1.3,2.5) -- cycle; % the notching (right)
\path[fill, gray] (1,2.5) -- (0.9,2.2) -- (0.7,2.5) -- cycle; % the notching (left)

\coordinate (rightpuncture) at (\TikzW*0.8,\TikzH/2);
\draw [style = thick] (\TikzW,0) ..  controls (3.3,0.5) and (3.3,3.5)   .. (\TikzW,\TikzH)  node[pos=0.3,left]{$3$}; 
\draw[style = thick] (\TikzW,0) .. controls (\TikzW*0.95,0.2) and (\TikzW*0.95,\TikzH*0.8) .. (rightpuncture) node[pos=0.65,right]{$t$};  % the bridging (tagged plain) arc
\draw[style = ultra thick, gray] (\TikzW,0) -- (rightpuncture) node[pos=0.3, left] {$t$};  % r bowtie

\newcommand\TikzTPW{\TikzW*0.85}
\newcommand\TikzTPH{\TikzH*0.25}
\path[fill, gray] (\TikzTPW,\TikzTPH+0.5) -- (\TikzTPW-0.1,\TikzTPH+0.2) -- (\TikzTPW-0.3,\TikzTPH+0.5) -- cycle; % the notching (left)
\path[fill, gray] (\TikzTPW,\TikzTPH+0.5) -- (\TikzTPW+0.1,\TikzTPH+0.8) -- (\TikzTPW+0.3,\TikzTPH+0.5) -- cycle; % the notching (right)

 \draw [fill=green] (0,0) circle (.6ex);
\draw [fill=green] (0,\TikzH) circle (.6ex);
\draw [fill=green] (\TikzW,0) circle (.6ex);
\draw [fill=green] (topright) circle (.6ex);
\draw [fill=violet] (1.2,2.2) circle (.6ex); %puncture
\draw [fill=violet] (rightpuncture) circle (.6ex); %puncture
\end{scope}
}

\newcommand  \TikzAffineDSixTaggedNextToQuiver[1]{
\begin{tikzpicture}[scale=#1] %% T2 triangulation tagged
\MacroAffineDSix
\end{tikzpicture}
\begin{tikzpicture}[scale=#1] %% T2 triangulation tagged
\MacroAffineDSix
\begin{scope}[dashed,red,->,style=very thick]
\draw[rounded corners = 10 pt] (2,1.5) -- (0.7,0.8) -- (0.8,1.9); % red arrow 2 to r ( no bow tie )
\draw[rounded corners = 10 pt] (2,1.5) -- (1.5,3.5) -- (0.75,3); % 2 to r (with bow tie)
\draw (2,1.5) -- (3,1.5); % two to 1
\draw (4,1.5) -- (3.1,1.5); % 3 to 1
\draw [rounded corners = 15 pt] (4,1.5) -- (5.2,3.5) -- (5.3,2.2); % 3 to t (no bow tie)
\draw [rounded corners = 10 pt] (4,1.5) -- (4.7,0.8) -- (5.4,1); % 3 to t ( with bow tie )
\end{scope}
\end{tikzpicture}
 }
 
 \newcommand\mysmallbowtie{\text{\tiny $\bowtie$}}
 
 \newcommand\MacroTikzDFourClustersTriangle[1]{
  \begin{scope}[thick]
 \draw ($(L)+(#1)$) -- ($(R)+(#1)$) -- ($(top)+(#1)$) -- ($(L)+(#1)$);
\draw [fill=violet] ($(P)+(#1)$) circle (.08); %puncture
 \end{scope}
 } 
 
\newcommand\MacroTikzDNorth{
\MacroTikzDFourClustersTriangle{north}
\draw ($(L)+(north)$) -- node[pos=0.7,red]{\small $x_1$} ($(P)+(north)$) -- node[pos=0.4, blue]{\small $x_2$} ($(R)+(north)$);
\draw[rounded corners=10pt] ($(L)+(north)$) -- ($(0,\TikzH*0.5)+(north)$) node[pos=1,brown]{\small $x_0$} -- ($(R)+(north)$);
 }
 
 \newcommand\MacroTikzDWest{
\MacroTikzDFourClustersTriangle{west}
\draw[magenta] ($(P)+(west)$) -- node[sloped,rotate=90,pos=0.2]{$\mysmallbowtie$}  ($(R)+(west)$);
\draw[rounded corners=10pt, brown] ($(L)+(west)$) -- ($(0,\TikzH*0.5)+(west)$) -- ($(R)+(west)$);
\draw[rounded corners=10pt, blue] ($(R)+(west)$) -- ($(P)+(west)+(0,-0.6)$) -- ($(P)+(west)$);
 }
 \newcommand\MacroTikzDEast{
\MacroTikzDFourClustersTriangle{east}
\draw[green] ($(L)+(east)$) -- node[sloped,rotate=90,pos=0.8]{$\mysmallbowtie$} ($(P)+(east)$) ;
\draw[brown] [rounded corners=10pt] ($(L)+(east)$) -- ($(0,\TikzH*0.5)+(east)$) -- ($(R)+(east)$);
\draw[rounded corners=10pt, red] ($(L)+(east)$) -- ($(P)+(east)+(0,-0.6)$) -- ($(P)+(east)$);
 }
 
 
 \newcommand\MacroTikzDBelowNorth{
 \MacroTikzDFourClustersTriangle{belownorth}
\draw[red]  ($(L)+(belownorth)$)  --  ($(P)+(belownorth)$);
\draw[olive] ($(P)+(belownorth)$) --  ($(top)+(belownorth)$) ; 
\draw[blue] ($(P)+(belownorth)$) -- ($(R)+(belownorth)$);
 }
 
  
  % left diamond begins
  \newcommand\MacroTikzDleftdiamondNorth{
 \MacroTikzDFourClustersTriangle{leftdiamondNorth}
\draw[blue]  ($(R)+(leftdiamondNorth)$)  --  ($(P)+(leftdiamondNorth)$);
\draw[olive] ($(P)+(leftdiamondNorth)$) --  ($(top)+(leftdiamondNorth)$) ;
\draw[rounded corners=15pt, orange] ($(R)+(leftdiamondNorth)$) -- ($(-0.6,-0.6)+(leftdiamondNorth)$) -- ($(top)+(leftdiamondNorth)$);
 }
 \newcommand\MacroTikzDleftdiamondWest{
 \MacroTikzDFourClustersTriangle{leftdiamondWest}
\draw [magenta] ($(R)+(leftdiamondWest)$)  -- node[sloped,rotate=90,pos=0.8]{$\mysmallbowtie$}  ($(P)+(leftdiamondWest)$);
\draw[rounded corners=20pt, orange] ($(R)+(leftdiamondWest)$) -- ($(-0.8,-1)+(leftdiamondWest)$) -- ($(top)+(leftdiamondWest)$);
\draw[rounded corners=10pt, blue] ($(R)+(leftdiamondWest)$) -- ($(P)+(leftdiamondWest)+(0,-0.6)$) -- ($(P)+(leftdiamondWest)$);
 }
\newcommand\MacroTikzDleftdiamondEast{
 \MacroTikzDFourClustersTriangle{leftdiamondEast}
\draw[gray]  ($(P)+(leftdiamondEast)$) --  node[pos=0.25]{$\mysmallbowtie$} ($(top)+(leftdiamondEast)$) ;
\draw[rounded corners=15pt, orange] ($(R)+(leftdiamondEast)$) -- ($(-0.6,-0.6)+(leftdiamondEast)$) -- ($(top)+(leftdiamondEast)$);
 \draw[rounded corners=10pt, olive] ($(top)+(leftdiamondEast)$) -- ($(P)+(leftdiamondEast)+(0.5,0)$) -- ($(P)+(leftdiamondEast)$);
 }
 \newcommand\MacroTikzDleftdiamondSouth{
 \MacroTikzDFourClustersTriangle{leftdiamondSouth}
\draw[magenta]  ($(R)+(leftdiamondSouth)$)  --  node[sloped,rotate=90,pos=0.8]{$\mysmallbowtie$} ($(P)+(leftdiamondSouth)$);
\draw[gray] ($(P)+(leftdiamondSouth)$) --  node[pos=0.25]{$\mysmallbowtie$} ($(top)+(leftdiamondSouth)$) ;
\draw[rounded corners=15pt, orange] ($(R)+(leftdiamondSouth)$) -- ($(-0.6,-0.6)+(leftdiamondSouth)$) -- ($(top)+(leftdiamondSouth)$);
 } 
 % left diamond ends
 

  \newcommand\MacroTikzDrightdiamondNorth{
 \MacroTikzDFourClustersTriangle{rightdiamondNorth}
\draw[red] ($(L)+(rightdiamondNorth)$)  --  ($(P)+(rightdiamondNorth)$);
\draw[olive] ($(P)+(rightdiamondNorth)$) --  ($(top)+(rightdiamondNorth)$) ;
\draw[rounded corners=15pt,cyan] ($(L)+(rightdiamondNorth)$) -- ($(0.6,-0.6)+(rightdiamondNorth)$) -- ($(top)+(rightdiamondNorth)$);
 }
 
   \newcommand\MacroTikzDrightdiamondWest{
 \MacroTikzDFourClustersTriangle{rightdiamondWest}
\draw [gray] ($(P)+(rightdiamondWest)$) --  node[pos=0.25]{$\mysmallbowtie$} ($(top)+(rightdiamondWest)$) ;
\draw[rounded corners=15pt, cyan] ($(L)+(rightdiamondWest)$) -- ($(0.6,-0.6)+(rightdiamondWest)$) -- ($(top)+(rightdiamondWest)$);
\draw[rounded corners=10pt, olive] ($(top)+(rightdiamondWest)$) -- ($(P)+(rightdiamondWest)+(-0.5,0)$) -- ($(P)+(rightdiamondWest)$);
 }
  \newcommand\MacroTikzDrightdiamondEast{
 \MacroTikzDFourClustersTriangle{rightdiamondEast}
\draw[green]  ($(L)+(rightdiamondEast)$)  -- node[sloped,rotate=90,pos=0.8]{$\mysmallbowtie$}  ($(P)+(rightdiamondEast)$);
\draw[rounded corners=20pt, cyan] ($(L)+(rightdiamondEast)$) -- ($(0.8,-1)+(rightdiamondEast)$) -- ($(top)+(rightdiamondEast)$);
\draw[rounded corners=10pt, red] ($(L)+(rightdiamondEast)$) -- ($(P)+(rightdiamondEast)+(0,-0.6)$) -- ($(P)+(rightdiamondEast)$);
 } 
 \newcommand\MacroTikzDrightdiamondSouth{
 \MacroTikzDFourClustersTriangle{rightdiamondSouth}
\draw[green]  ($(L)+(rightdiamondSouth)$)  --  node[sloped,rotate=90,pos=0.8]{$\mysmallbowtie$} ($(P)+(rightdiamondSouth)$);
\draw[gray]  ($(P)+(rightdiamondSouth)$) -- node[pos=0.25]{$\mysmallbowtie$} ($(top)+(rightdiamondSouth)$) ;
\draw[rounded corners=15pt, cyan] ($(L)+(rightdiamondSouth)$) -- ($(0.6,-0.6)+(rightdiamondSouth)$) -- ($(top)+(rightdiamondSouth)$);
}
\newcommand\MacroTikzDAboveSouth{
\MacroTikzDFourClustersTriangle{abovesouth}
\draw  [green] ($(L)+(abovesouth)$)  --  node[sloped,rotate=90,pos=0.8]{$\mysmallbowtie$}  ($(P)+(abovesouth)$);
\draw [gray] ($(P)+(abovesouth)$) -- node[pos=0.25]{$\mysmallbowtie$} ($(top)+(abovesouth)$) ; 
\draw [magenta] ($(P)+(abovesouth)$) -- node[sloped,rotate=90,pos=0.2]{$\mysmallbowtie$} ($(R)+(abovesouth)$);
 }
\newcommand\MacroTikzDSouth{
\MacroTikzDFourClustersTriangle{south}
\draw[green] ($(L)+(south)$) -- node[sloped,rotate=90,pos=0.8]{$\mysmallbowtie$} ($(P)+(south)$);
\draw[magenta] ($(P)+(south)$) -- node[sloped,rotate=90,pos=0.2]{$\mysmallbowtie$} ($(R)+(south)$);
\draw[rounded corners=10pt, brown] ($(L)+(south)$) --  ($(0,\TikzH*0.5)+(south)$) -- ($(R)+(south)$);
}

 \newcommand\TikzDFourClusters[1]{
 \begin{tikzpicture}[scale=#1]
%\MacroTikzDFourClustersTriangle

\newcommand\TikzX{5cm}
\newcommand\Tikzx{2cm}
\newcommand\TikzY{4.5cm}

\newcommand\TikzBelow{4.5cm}
\newcommand\TikzLeft{4cm}
\newcommand\Tikzleft{3cm}
\newcommand\TikzRight{4cm}
\newcommand\Tikzright{3cm}

  \node (north) {};
 \node [below= 0.8*\TikzBelow of north] (belownorth){};% {belownorth};
  \node [below= \TikzBelow of belownorth,  yshift=-\TikzY, left= \Tikzleft of belownorth] (leftdiamondNorth){};%  {leftdiamondNorth};
    \node [below= \TikzBelow of belownorth,  yshift=-\TikzY, right= \Tikzright  of belownorth] (rightdiamondNorth){};% {rightdiamondNorth};
    
   \node [below= \TikzBelow of leftdiamondNorth,  yshift=-\TikzY, left= \Tikzleft of leftdiamondNorth,xshift=-0] (leftdiamondWest) {};%{leftdiamondWest};
   \node [below= \TikzBelow  of leftdiamondNorth,  yshift=-\TikzY, right= \Tikzright  of leftdiamondNorth,xshift=-2cm] (leftdiamondEast) {};
   
 \node [below= \TikzBelow of rightdiamondNorth, yshift=-\TikzY, right= \Tikzright  of rightdiamondNorth,xshift=-0]  (rightdiamondEast) {};%{rightdiamondEast};
\node  [below= \TikzBelow  of rightdiamondNorth,  yshift=-\TikzY, left= \Tikzleft of rightdiamondNorth,xshift=2cm] (rightdiamondWest) {};

\node  [below= 1.8*\TikzBelow  of leftdiamondNorth] (leftdiamondSouth) {};%{leftdiamondSouth};
\node [below=1.8* \TikzBelow of rightdiamondNorth] (rightdiamondSouth) {};%{rightdiamondSouth};

  \node [right= \TikzRight  of rightdiamondEast]  (east) {};%{east};
    \node [ left= \TikzLeft of leftdiamondWest] (west){};% {west};
\node [below= 3.8*\TikzBelow  of belownorth] (abovesouth) {};%{abovesouth};
\node [below=0.8* \TikzBelow of abovesouth] (south) {};%{south};

\node [right= 9cm of north, yshift=-2cm, brown] {$x_0$};
\node [above=5.5cm of leftdiamondWest, blue] {$x_2$};
\node [above=5.5cm of rightdiamondEast, red] {$x_1$};

\node [below=4cm of belownorth, olive] {$\frac{x_{1} + x_{2}}{x_{0}}$};
\node [above=4cm of abovesouth, gray] {$\frac{x_{0} x_{1} + x_{0} x_{2} + x_{1} + x_{2}}{x_{0} x_{1} x_{2}}$};

\node [below=4.5cm of leftdiamondWest, magenta] {$\frac{x_{0} + 1}{x_{1}}$};
\node [below=4.5cm of rightdiamondEast, green] {$\frac{x_{0} + 1}{x_{2}}$};

\node [below=2.7cm of leftdiamondNorth, xshift=-0.7cm, orange] {$\frac{x_{0} x_{2} + x_{1} + x_{2}}{x_{0} x_{1}}$};
\node [below=2.7cm of rightdiamondNorth, xshift=0.7cm, cyan] {$\frac{x_{0} x_{1} + x_{1} + x_{2}}{x_{0} x_{2}}$};

%  \path[every node/.style={font=\sffamily\small}]

\draw [<->,>=stealth',shorten >=35pt,shorten <=35pt,  ultra thick]
(north) edge node[right] {} (belownorth)
    (south) edge (abovesouth);

\draw [<->,>=stealth',shorten >=45pt,shorten <=45pt,
%auto,node distance=4cm,
  ultra thick]
(belownorth) edge (leftdiamondNorth) (belownorth) edge (rightdiamondNorth)
     (leftdiamondNorth) edge [bend right] (leftdiamondWest) (rightdiamondNorth) edge [bend right] (rightdiamondWest)
     (leftdiamondNorth) edge [bend left] (leftdiamondEast) (rightdiamondNorth) edge [bend left] (rightdiamondEast);
     
       \draw [<->,>=stealth',shorten >=32pt,shorten <=32pt,  ultra thick]  
     (leftdiamondSouth) edge [bend right] (leftdiamondEast)  (rightdiamondSouth) edge [bend right] (rightdiamondEast)
    (leftdiamondSouth) edge [bend left] (leftdiamondWest)  (rightdiamondSouth) edge [bend left] (rightdiamondWest);
    
  \draw [<->,>=stealth',shorten >=43pt,shorten <=43pt,  ultra thick]  
        (abovesouth) edge (leftdiamondSouth) (abovesouth) edge (rightdiamondSouth);
    
   \draw [<->,>=stealth',shorten >=35pt,shorten <=35pt,
%auto,node distance=4cm,
  ultra thick]
    (leftdiamondEast) edge (rightdiamondWest)
        (west) edge (leftdiamondWest)
    (east) edge (rightdiamondEast);

    
   \draw [<->,>=stealth',shorten >=45pt,shorten <=45pt,
%auto,node distance=4cm,
  ultra thick]
    (north) edge[bend right] node[sloped,below] {} (west)
     (north) edge[bend left] node[sloped,below] {}  (east)
    (south) edge[bend left] (west) (south) edge[bend right] (east);

 \newcommand\TikzH{1.5}
 \newcommand\TikzW{2}
 \coordinate (P) at (0,-0.2);
 \coordinate (L) at (-\TikzW,-\TikzH );
 \coordinate (R) at (\TikzW,-\TikzH );
 \coordinate (top) at (0,\TikzH);
% \newcommand\TikzOrigin{(north)}

\MacroTikzDNorth;

\MacroTikzDWest
\MacroTikzDEast

\MacroTikzDBelowNorth

\MacroTikzDrightdiamondNorth
\MacroTikzDleftdiamondNorth

\MacroTikzDrightdiamondWest
\MacroTikzDleftdiamondWest

\MacroTikzDrightdiamondEast
\MacroTikzDleftdiamondEast

\MacroTikzDrightdiamondSouth
\MacroTikzDleftdiamondSouth

\MacroTikzDAboveSouth;
\MacroTikzDSouth;

\node at (abovesouth) {\MacroTikzDAboveSouth};
\draw (south) node{\MacroTikzDSouth};

 \end{tikzpicture}
 }
 
\newcommand\TikzTwoRectanglesFlipsAndThreeDigonsFlips[1]{
 \begin{tikzpicture}[scale=#1]
%\MacroTtwoTriangulationCoordinatesForDrawingQuiver

\begin{scope}[xshift=-3cm, very thick]
\draw [<->, red] (0,2) -- (2,2);
\draw[red] (1,3) node{\small $\mu_r$};
\end{scope}
\begin{scope}[xshift=-6.5cm,thick]
\draw[style = thick]  (0,0) -- (0,4);
\draw [style = thick] (0,0) ..  controls (3.7,1) and (3.8,3.9)   .. (0,4);
\draw[style = thick,blue] (0,4) --  (1.2,2.2);
\draw[style = thick,red] (1.2,2.2) -- (0,0) node[pos=0.5,right]{\small $r$};
\draw [fill=violet] (1.2,2.2) circle (.3ex); %puncture
\draw [fill=green] (0,0) circle (.3ex);
\draw [fill=green] (0,4) circle (.3ex);
\end{scope}

\begin{scope}[thick]
\draw[style = thick]  (0,0) -- (0,4); 
\draw [style = thick] (0,0) ..  controls (3.7,1) and (3.8,3.9)   .. (0,4); % the middle arc 
\draw[style = thick, blue] (0,4) .. controls (0.8,1) and (1,2) .. (1.2,2.2) node[pos=0.5,below] {\small $b$};  % the radius (tagged plain) arc
\draw[style = thick, red] (0,4) -- (1.2,2.2) node[pos=0.5, right=2pt] {\small $r'$};  
\draw[red] (1,2.5) -- (1.1,2.8) -- (1.3,2.5) -- cycle; % the notching (right)
\draw[red] (1,2.5) -- (0.9,2.2) -- (0.7,2.5) -- cycle; % the notching (left)
\end{scope}

 \draw [fill=green] (0,0) circle (.3ex);
\draw [fill=green] (0,4) circle (.3ex);
\draw [fill=violet] (1.2,2.2) circle (.3ex); %puncture

\begin{scope}[xshift=4cm, very thick]
\draw [<->, blue] (0,2) -- (2,2);
\draw [blue] (1,3) node{\small $\mu_b$};
\end{scope}

\begin{scope}[xshift=7cm, thick]
\draw[style = thick]  (0,0) -- (0,4);
\draw [style = thick] (0,0) ..  controls (3.7,1) and (3.8,3.9)   .. %node[pos =0.4,left=2pt] {$2$}  
(0,4);
\draw[style = thick,red] (0,4) -- (1.2,2.2);
\draw[style = thick,blue] (1.2,2.2) -- node[pos =0.4,right=1pt] {\small $b'$}  
(0,0);
\draw [fill=violet] (1.2,2.2) circle (.3ex); %puncture
\draw [fill=green] (0,0) circle (.3ex);
\draw [fill=green] (0,4) circle (.3ex);
 \draw[red] (1,2.5) -- (1.1,2.8) -- (1.3,2.5) -- cycle; % the top notching (right)
\draw[red] (1,2.5) -- (0.9,2.2) -- (0.7,2.5) -- cycle; % the top notching (left)
 \draw[blue] (1,1.9) -- (1.1,1.6) -- (1.3,1.9) -- cycle; % the bottom notching (right)
\draw[blue] (1,1.9) -- (0.9,2.2) -- (0.7,1.9) -- cycle; % the bottom notching (left)
\end{scope}


\newcommand\TikzNorthEast{(3,2)}
\newcommand\TikzNorthWest{(0,2)}
\newcommand\TikzSouthEast{(3,0)}
\begin{scope}[yshift=5.2cm]
\begin{scope}[thick]
\draw (-4,1) node{For example,};
\draw[style = thick]  (0,0) rectangle \TikzNorthEast; 
\draw[magenta] \TikzNorthWest -- \TikzSouthEast node[pos=0.6,above] {\small $k$};
\end{scope}
\draw[magenta, <->, very thick] (4,1) -- (6,1);
\draw[magenta] (5,1.7) node{\small $\mu_k$};
\begin{scope}[thick,xshift = 7cm]
\draw[style = thick]  (0,0) rectangle \TikzNorthEast; 
\draw[magenta] (0,0) -- \TikzNorthEast node[pos=0.4,above]{\small $k'$};
\end{scope}
\end{scope}

\end{tikzpicture}
}