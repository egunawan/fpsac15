\documentclass[a0paper, portrait, 17pt]{tikzposter}
%12pt, 14pt, 17pt, 20pt, or 25pt;

\usepackage{amsmath}
%\usepackage{graphicx}
\usepackage{amssymb}
%\usepackage{subfigure}
%\usepackage{amsthm}
\usepackage{enumerate} % To use roman numerals in enumerate

%\usepackage{capt-of} % to do \captionof
%\usepackage[labelfont=bf]{caption}
%\usepackage{subcaption}
%\usepackage[margin=10pt,font=small,labelfont=bf]{caption}
\usepackage{wrapfig}%,kantlipsum}

%\usepackage{filecontents}% http://ctan.org/pkg/filecontents
\usepackage{lipsum}% http://ctan.org/pkg/lipsum
\usepackage{adjustbox} % to use minipage
\usepackage{multicol}

\usepackage[backend=bibtex,style=numeric,sorting=none]{biblatex} % to use .bib file

%\usetheme{Simple} % red white
%\usetheme{Basic} % green
%\usetheme{Envelope} % blue and blue
\usetheme{Board} % light blue, kind of cute
%\usetheme{Autumn} % like simple but brown and blue

%\usecolorstyle[colorPalette=BrownBlueOrange]{Germany}
%\usetitlestyle{Wave}


%\documentclass{amsart}
%\pagestyle{plain}
%\setlength{\parskip}{0in}
%\setlength{\textwidth}{6.8in}
%\setlength{\topmargin}{-.5in}
%\setlength{\textheight}{9.3in}
%\setlength{\parindent}{.25in}
%\setlength{\oddsidemargin}{-.7cm}
%\setlength{\evensidemargin}{-.7cm}
%
%
%\usepackage{graphicx}
%\usepackage{amsmath, amssymb, amsbsy}
%\usepackage{subfigure}
%%\usepackage{caption}
%%\usepackage{subcaption}
%
%\usepackage{amsthm}
%\usepackage{mdwlist} % To put a sentence in the middle of enumerate
%\usepackage{enumerate} % To use roman numerals in enumerate
%\usepackage{multicol} % to break a list into multiple columns
\usepackage{hyperref}
\hypersetup{
    colorlinks=true,
    linkcolor=blue,
    filecolor=magenta,      
    urlcolor=cyan,
}
 
\urlstyle{same}

%\theoremstyle{plain}
%\newtheorem{theorem}{Theorem}[section]
%
%\theoremstyle{definition}
%\newtheorem{remark}[theorem]{Remark}
%\newtheorem{lemma}[theorem]{Lemma}
%\newtheorem{corollary}[theorem]{Corollary}
%\newtheorem{definition}[theorem]{Definition}
%\newtheorem{notation}[theorem]{Notation}
%\newtheorem{example}[theorem]{Example}
%\newtheorem{proposition}[theorem]{Proposition}

\numberwithin{equation}{section} % If you want equations to be numbered like (5.1), (5.2), ...

\newcommand\notch{^{(p)}}
\newcommand \tilblacktri[1]{\widetilde{\triangle}_{#1}}
\newcommand \blacktri[1]{\triangle_{#1}}
\newcommand \tiltaui[1] {\widetilde{\tau}_{i_{#1}}} 
\newcommand \taui[1] {\tau_{i_{#1}}}


\newcommand \al {\alpha} \newcommand \be {\beta}
\newcommand\ga{\gamma}  \newcommand\Ga{\Gamma}
\newcommand\ta{\tau } \newcommand\si{\sigma}  \newcommand\Si{\Sigma} 
\newcommand \om {w} \newcommand \oom {$w$}
\newcommand\Cn{\mathcal{C}_n}
\newcommand\Surf{S}
\newcommand\Cnn[1]{\mathcal{C}_{#1}}
\newcommand\Z{\mathbb{Z}}
\newcommand\B{\mathcal B} 
\newcommand\AAA {\mathcal A}

\newcommand\tgk{\tau_{[\ga_k]}}
\newcommand\tbk{\tau_{[\be_k]}}
\newcommand\tgl{\tau_{[\ga_\ell]}} 
\newcommand \llrrl {\ell, \ell, r, r, \ell} \newcommand \lrrll {\ell, r, r, \ell, \ell}
\newcommand\rrBak{\underline{r, r}}
\newcommand\rrNon{\mathbf{r, r}}
\newcommand\ttBak{\underline{\tau, \tau}}
\newcommand\ttNon{\boldsymbol{\tau, \tau}}
\newcommand\ttQuasi{\overline{\tau, \tau}}
\newcommand\dotRR{\dot{\textbf{r}}, \textbf{r}}
\newcommand\RddotR{\textbf{r}, \ddot{\textbf{r}}}
\newcommand\GT {G_{T^o,\ga}}
\newcommand\overGT {\overline{G}_{T^o,\ga}}
\newcommand\puncture{\text{\small $P$}}
\newcommand\Quad[1]{Quad{(#1)}}

\newcommand \tilp {\widetilde{p}} \newcommand \tilP {\widetilde{P}} 
\newcommand\tilv{\widetilde{v}} \newcommand\tily{\widetilde{y}} \newcommand\tilz{\widetilde{z}}
\newcommand\tilt{\tilde{t}} \newcommand\tils{\tilde{s}} \newcommand\tilr{\widetilde{r}} \newcommand\tilell {\widetilde{\ell}}
\newcommand\tilS{\widetilde{\Sga}}  \newcommand\Sga{S_\ga} 
\newcommand\tilA{\widetilde{A}} \newcommand\tilD{\widetilde{D}}
\newcommand\tilga{\widetilde{\ga}} 
\newcommand\tilsi{\widetilde{\si}} 
\newcommand\tilw{\widetilde{\om}} 
\newcommand\tilal{\widetilde{\al}}
\newcommand\tilbe{\widetilde{\be}} 
\newcommand\tillam{\widetilde{\lambda}} 
%\newcommand\tiltheta{\widetilde{\theta}} 
%\newcommand\tilrho{\widetilde{\rho}} 
\newcommand\tiltau{\widetilde{\tau}} 
%\newcommand\tilom{\widetilde{\om}}
 \newcommand\tilom{\widetilde{\om}}
\newcommand\overpi{\overline{\pi}} 
\newcommand\tilT{\widetilde{T}}
\newcommand\tilTo{\widetilde{T^o}}
\def\a{\aleph}
\def\b{\text{\tiny $\beth$}}
\newcommand\ccr{r^{cc}}
\newcommand\bijection{\longleftrightarrow}

\newcommand\ts[1]{\tau_{[\si_{#1}]}} 
\newcommand\tg[1]{\tau_{[\ga_{#1}]}} 
\newcommand\tb[1]{\tau_{[\be_{#1}]}} 
\newcommand\tiltg[1]{\widetilde{\tau}_{[\ga_{#1}]}}
\newcommand \pkpkone[1] {[p_k,p_{k+1}]_{#1}}
\newcommand\pipjOm[2]{[p_{#1}, \dots, p_{#2}]_{\om}}
%\newcommand\overom{\overline{\om}}
\newcommand\overom{\bar{\om}}
%\newcommand \pjpjone[1] {[p_j,p_{j+1}]_{#1}} % for consistency always use k!!! not j
\newcommand\disk[1]{\textbf{Disk}_{#1}}
%\newcommand\tildisk[1]{\widetilde{\textbf{Disk}_{#1}}}
\newcommand\tildisk[1]{\textbf{Disk}_{\widetilde{#1}}}
%\newcommand\tildiskplus[2]{\widetilde{\textbf{Disk}_{#1}^{#2}}}
\newcommand\tildiskplus[2]{\textbf{Disk}_{\widetilde{#1}}^{#2}}
\newcommand\outsidedisk[1]{\textbf{C}_{#1}}


%\newcommand\TpathOne{(b_4, 	1, 	b_2, 	2, 	b_3, 		\ell, 	b_3)}
\newcommand\TpathCompleteOne{(b_4, 	1, 	b_2, 	2, 	b_3, 		\ell, 	\underline{r, 		r}, 	\underline{\ell, 		\ell}, 	b_3)}
\newcommand\PMOne{(b_4, 	 	b_2, 	 	b_3, 		 		r, 	\ell, 	b_3)}

%\newcommand\TpathTwo{(b_4, 	2, 	b_3)}
\newcommand\TpathCompleteTwo{(b_4, 	\underline{1,	1}, 	2, 	\underline{\ell, 		\ell}, 	\underline{r, 		r}, 	\underline{\ell, 		\ell}, 	b_3)}
\newcommand\PMTwo{(b_4, 	1, 	\ell, 	r, 	\ell, 	b_3)}

\newcommand\TpathThree{(b_1, 1, 2, \ell, 	2)}
\newcommand\TpathCompleteThree{(b_1, 	1, 	\underline{2, 	2}, 	2, 		\underline{\ell, 	\ell}, 		\underline{r,	r}, 		\ell, 	2)}
\newcommand\PMThree{(b_1,	2,2,\ell, r, 2)}

%\newcommand\TpathFour{(b_1, 1, b_3, \ell, b_3)}
\newcommand\TpathCompleteFour{(b_1, 	1, 	\underline{2, 	2}, 	b_3, 		\ell, 	\underline{r, 		r},	\underline{\ell, 		\ell}, 	b_3)}
\newcommand\PMFour{(b_1, 2, b_3, r,\ell, b_3)}

%\newcommand\TpathFive{(b_4, 1, b_2, \ell, 2)}
\newcommand\TpathCompleteFive{(b_4, 	1, 	b_2, 	\underline{2, 	2}, 		\underline{\ell, 	\ell}, 		\underline{r,	r}, 		\ell, 	2)}
\newcommand\PMFive{(b_4, b_2, 2, \ell,r,2)}

%\newcommand\TpathSix{(b_4, 1, b_2, {\bf r, {r}}, \ell, b_3)}
\newcommand\TpathCompleteSix{(b_4, 	1, 	b_2, \underline{2, 	2}, 		\underline{\ell, 	\ell},	\rrNon, 	\ell, 	b_3)} % \ddot{r}, 	\ell, 	b_3),}
\newcommand\TpathCompleteSixDollar{$(b_4$,$1$,$b_2$,\underline{$2$,$2$},\underline{$\ell$,$\ell$},$\rrNon$,$\ell$,$b_3)$} % \ddot{r}, 	\ell, 	b_3),}
\newcommand\tilSpathCompleteSix{(b_4, 	1, 	b_2, \underline{2, 	2}, 		\underline{\ell, 	\ell},		\RddotR, 	\ell, 	b_3)} % \ddot{r}, 	\ell, 	b_3),}
\newcommand\PMSix{(b_4, 	b_2,2,	\ell,	\ddot{r},	\ell, 	b_3)} 

%\newcommand\TpathSeven{(b_4, 1, b_2, \ell,{\bf {r}},{\bf {r}}, b_3)}
\newcommand\TpathCompleteSeven{(b_4, 	1, 	b_2, \underline{2, 	2}, 		\ell, 	\rrNon,	\underline{\ell, 		\ell}, 	b_3)} %, %\dot{r}, 	r,	\ell, 		\ell, 	b_3)}
\newcommand\TpathCompleteSevenDollar{$(b_4$,$1$,$b_2$,\underline{$2$,$2$},$\ell$,$\rrNon$,\underline{$\ell$,$\ell$},$b_3)$} %, %\dot{r}, 	r,	\ell, 		\ell, 	b_3)}
\newcommand\tilSpathCompleteSeven{(b_4, 	1, 	b_2, \underline{2, 	2}, 		\ell, 	\dotRR,	\underline{\ell, 		\ell}, 	b_3)}
\newcommand\PMSeven{(b_4, b_2, 2, \dot{r}, \ell, b_3)}


%\newcommand\TpathEight{(b_1, 1, 2, {\bf r, {r}}, \ell, b_3)}
\newcommand\TpathCompleteEight{(b_1,1,	\underline{2, 2}, 	2, 		\underline{\ell, 	\ell}, 		\rrNon, \ell, 	b_3)}
\newcommand\TpathCompleteEightDollar{$(b_1$,$1$,\underline{$2$,$2$},$2$,\underline{$\ell$,$\ell$},$\rrNon$,$\ell$,$b_3)$}
%\newcommand\TpathCompleteEight{($b_1$, 	$1$,	\underline{$2$, $2$}, 	$2$, 		\underline{$\ell$, 	$\ell$}, 		$\rrNon$, $\ell$, 	$b_3$)} %, % \ddot{r}, 	\ell, 	b_3)}

%\newcommand\tilSpathCompleteEight{($b_1$, 	$1$, 	\underline{$2$, $2$}, 	$2$, 		\underline{$\ell$, 	$\ell$}, 		$\RddotR$, 	$\ell$, 	$b_3$)}
\newcommand\tilSpathCompleteEight{(b_1, 	1, 	\underline{2, 2}, 	2, 		\underline{\ell, 	\ell}, 		\RddotR, 	\ell, 	b_3)}

\newcommand\PMEight{(b_1, 2,2, \ell,	\ddot{r}, b_3)}

%\newcommand\TpathNine{(b_1, 1, 2, \ell, {\bf{r}, r}, b_3)}
\newcommand\TpathCompleteNine{(b_1, 	1, 	\underline{2, 	2}, 	2, 		\ell, 	{\bf {r}, 	r},	\underline{\ell, 		\ell}, 	b_3)} %. %\dot{r}, 	r,	\ell, 		\ell, 	b_3)}
\newcommand\TpathCompleteNineDollar{$(b_1$,$1$,\underline{$2$,$2$},$2$,$\ell$,$\rrNon$,\underline{$\ell$,$\ell$},$b_3)$} %. %\dot{r}, 	r,	\ell, 		\ell, 	b_3)}
\newcommand\tilSpathCompleteNine{(b_1, 	1, 	\underline{2, 	2}, 	2, 		\ell, 	\dotRR,	\underline{\ell, 		\ell}, 	b_3)} %. %\dot{r}, 	r,	\ell, 		\ell, 	b_3)}

\newcommand\PMNine{(b_1, 2, 2, \dot{r}, \ell, b_3)}  %%%%
%%%%%%% TIKZ %%
\input{tpath_poster_figures}  %%%%
%%%%%%% TIKZ %%
%%%%

%%%%%
 %%%%%% TIKZ %%%
 %%%%%% TIKZ %%%

%%%%%%%%%%%%%
%%%%%%%% shortcuts %%%

\newcommand\TpathSize{0.8}


%%%%% shortcuts %%%%%
%%%%%%%%%%%%%%%

\title{Atomic Bases and $T$-path Formula for Cluster Algebras of Type $D$} \author{Emily Gunawan* and Gregg Musiker} 
\institute{University of Minnesota, School of Mathematics, Minneapolis, USA}

%\titlegraphic{\includegraphics{logo}}

\addbibresource{references.bib}
%\bibliography{references}


\begin{document} \maketitle
\begin{columns}
\column{0.65}
%
%\block{Abstract}{
%We extend a $T$-path expansion formula for arcs on an unpunctured surface 
%to the case of arcs on a once-punctured polygon 
%and use this formula to give a combinatorial proof
%that cluster monomials form the atomic basis of a cluster algebra of type $D$.
%}

\block{What is a cluster algebra? (Fomin and Zelevinsky, 2000)} {
\begin{itemize}

\item 
A (coefficient-free) \emph{cluster algebra}  of rank $n$  is a $\mathbb{Z}$-subalgebra of $\mathbb{Q}(x_1, \dots, x_n)$
% the field of rational functions in n variables with rational coefficients
generated by elements called \emph{cluster variables}:

\begin{enumerate}[-]
\item Start with an \emph{initial seed}: a \emph{cluster} $\mathbf{x}=\{ x_1, x_2, \dots, x_n\}$
and a skew-symmetrizable matrix $B=(b_{ij})$.
\item For each $k=1,\dots,n$, we can \emph{mutate} in the $k$-th direction
$\left( \{ x_1, \dots, {x_k}, \dots, x_n\},B \right) \overset{\mu_k}{\longrightarrow} 
\left( \{ x_1, \dots, {x_k'}, \dots, x_n\}, \mu_k(B) \right)$
to obtain a new seed
where 
\[x_k' =\frac{1}{x_k}\left( {\prod_{b_{ik}>0} x_i^{b_{ik}} + \prod_{b_{ik}<0} x_i^{-b_{ik}}}\right)
\, \text{ and }\, \mu_k(B)_{ij}=
 \begin{cases}
 - b_{ij}&\text{if $k=i$ or $k=j$} \\
b_{ij} + b_{ik}b_{kj} &\text{if $b_{ik}>0$ and $b_{kj}>0$}\\
b_{ij} - b_{ik}b_{kj} &\text{if $b_{ik}<0$ and $b_{kj}<0$}\\
b_{ij} &\text{otherwise}
\end{cases}\]
\item Apply all possible sequences of mutations to produce all cluster variables (usually inifinite).
\end{enumerate}

%\item Construction of cluster variables:
%\begin{itemize}
%\item Specify an initial \emph{cluster}
%$\{ x_1, x_2, \dots, x_n\}$ %of cluster variables
%
%\item Produce new clusters $\{ x_1,  \dots, \hat{x_k}, \dots, x_n\}$ by \emph{mutation} at $k$
%via certain exchange relations
%$$ x_k' = \frac{\prod x_i^{d_i+} + \prod x_i^{d_i-}} { x_k }$$
%where the products run through other elements of the cluster
%$\{ x_1, \dots, {x_k}, \dots, x_n\} \longleftrightarrow \{ x_1, \dots, \hat{x_k}, \dots, x_n\}$. 
%
%\item Repeat this procedure (usually infinitely many times).
%\end{itemize}
\end{itemize}

    \begin{itemize}
    \item \textbf{Laurent Phenomenon:} each cluster variable can be expressed as a Laurent polynomial in $\{ x_1, \dots, x_n\}$.
    \item \textbf{Positivity:} this Laurent polynomial has positive coefficients (2014, [Lee and Schiffler], [Gross, Hacking, Keel, and Kontsevich], and others).
    \end{itemize}
}

\block{Cluster Algebras from Surfaces} {

\begin{multicols*}{2}
\innerblock{Definition: ordinary arcs}{
\begin{itemize}
\item An \emph{ordinary ar}c $\gamma$ is a non-contractible curve between marked points
such that $\gamma$ does not cross itself or the boundary, and $\gamma$ is not homotopic to a boundary edge.
%\begin{itemize}
%\item $\gamma$ does not cross itself or the boundary of $S$
%\item $\gamma$ is not homotopic to a boundary edge.
%\end{itemize} 
\item A loop \TikzNoose{0.4} that cuts out a monogon with $1$ puncture is called a \emph{noose}.

\item An \emph{ideal triangulation} is a maximum collection of distinct arcs that pairwise do not cross.
\end{itemize}

%\OncePuncturedSquare{0.6}
\TikzTtwoIdealTriangulation{0.6}
\TikzToneOnly{0.6}
%\caption{Some ideal triangulations of a once-punctured disk}
%\end{figure}
}

\innerblock{Definition: tagged arcs}{

\begin{itemize}
\item A \emph{tagged arc} 
is an ordinary arc (not a noose) decorated (plain or with a $\bowtie$) at each endpoint.
%is obtained by making each endpoint of an ordinary arc (that is not a noose)
%either plain or notched (indicated by a bow tie).

\item A \emph{tagged triangulation} is a maximum collection of distinct tagged arcs
that are pairwise ``compatible''.
%satisfying several criteria including pairwise non-crossing
\end{itemize}
\TikzTtwoTaggedTriangulation{0.6}
\TikzToneOnly{0.6}
\TikzToneBowtiesOnly{0.6}
}
\end{multicols*}

\innerblock{Theorem (Fomin, Shapiro, and Thurston, 2006)}{

One can define a cluster algebra from a Riemann 
surface $+$ interior points (called \emph{punctures}) and/or
marked points on the boundary:
%corresponding 
%to a given Riemann surface with marked points:
% Given a Riemann surface with marked points, we can define a corresponding cluster algebra
%\begin{itemize}
\begin{align*}
%\item 
%cluster + exchange relations 
\text{seed $\left( \mathbf{x}_T, B_T\right)$} &\longleftrightarrow \text{tagged triangulation $T=\{\tau_1,\dots,\tau_n\}$}\\
%\item 
\text{cluster variable $x_\ga$} &\longleftrightarrow \text{tagged arc $\ga$}\\
%\item 
%exchanging clusters 
\text{cluster mutation} &\longleftrightarrow \text{``flipping diagonals"}
%  \text{cluster monomials}  && \longleftrightarrow  & &\text{multi-tagged dissections}
%\end{itemize}
\end{align*}
}
}
\note[width=11cm, targetoffsetx=0.09\textwidth,targetoffsety=-8.5cm,innersep=-.2cm,angle=0,connection]{
\begin{tikzfigure}
\TikzTwoRectanglesFlipsAndThreeDigonsFlips{0.5}
\end{tikzfigure}
}
\column{0.35}
\block{Example: once-punctured $4$-gon \OncePuncturedSquare{0.25}}{
A tagged triangulation of a once-punctured square $\longleftrightarrow$ A quiver that is mutation equivalent to an orientation of a type $D_4$ Dynkin diagram.

\TikzTtwoTaggedNextToQuiver{0.7}
\begin{tikzpicture}[scale =0.4]
\begin{scope}[->,>=stealth',shorten >=1pt,auto,node distance=1.5cm, ultra thick, red]
  \node (2) {};
  \node (1) [ left of=2] {};
  \node (r) [ above of=2, right of=2] {};
  \node (rp) [ below of =2, right of=2] {};

  \path[every node/.style={font=\sffamily\small}]
    (2) edge (1)
    (2) edge (rp)
    (2) edge (r);
    %(r)  edge [bend right]  (2);
  \draw (1) node{$1$};
  \draw (2) node{$2$};
  \draw (r) node{$r$};
   \draw (rp) node{$r^{\bowtie}$};
  % \draw (10,0) node{};
\end{scope}
\end{tikzpicture}
 % \caption  {Finite $D_4$ Cluster algebra $\leftrightarrow$ finite $D_4$ Dynkin diagram}
%\begin{tikzpicture}
%  \draw (0:0) node{
%  \begin{math}\bordermatrix{
%\text{\small arcs} & \text{\small $1$} & \text{\small $2$} & \text{\small $r$} & \text{\small $r^{\bowtie}$}\cr
% \text{\small $1$} & 0 & -1 & 0 & 0 \cr
%\text{\small $2$} &1 & 0 & 1 & 1 \cr
%\text{\small $r$} & 0 & -1 & 0 & 0 \cr
%\text{\small $r^{\bowtie}$} & 0 & -1 & 0 & 0}\end{math}};
%\end{tikzpicture} 
}
\note[width=9cm,targetoffsetx=1.5cm,targetoffsety=-1cm,innersep=.5cm,connection]{
This example corresponds to \\
  \begin{math}
  B = \bordermatrix{
%\text{\small arcs} 
& \text{\small $1$} & \text{\small $2$} & \text{\small $r$} & \text{\small $r^{\bowtie}$}\cr
 \text{\small $1$} & 0 & -1 & 0 & 0 \cr
\text{\small $2$} &1 & 0 & 1 & 1 \cr
\text{\small $r$} & 0 & -1 & 0 & 0 \cr
\text{\small $r^{\bowtie}$} & 0 & -1 & 0 & 0}\end{math}
}

\block{Example: once-punctured $3$-gon}{
\begin{tikzfigure}
[There are ${{2n}\choose n}-{{2n-2}\choose{n-1}}$ tagged triangulations of a once-punctured $n$-gon.]
%[The exchange graph for tagged triangulations of a once-punctured triangle]
\TikzDFourClusters{0.6}
\end{tikzfigure}
}

%\block{}{
%$n^2$ cluster variables 
%\begin{enumerate}[]
%\item $ \frac{x_{0} x_{1} + x_{0} x_{2} + x_{1} + x_{2}}{x_{0} x_{1} x_{2}} $ %
%\item $ \frac{x_{0} x_{2} + x_{1} + x_{2}}{x_{0} x_{1}} $ %
%\item $ \frac{x_{1} + x_{2}}{x_{0}} $ %
%\item $ \frac{x_{0} + 1}{x_{1}} $ %
%\item $ \frac{x_{0} x_{1} + x_{1} + x_{2}}{x_{0} x_{2}} $ %
%\item $ \frac{x_{0} + 1}{x_{2}} $ %
%\item $ x_{2} $
%\item $ x_{1} $
%\item $ x_{0} $
%\end{enumerate}
%}
%\end{subcolumns}
\end{columns}


\begin{columns}
\column{0.55}
\block{Result 1: $T$-path formula for cluster variables for type $D$ cluster algebras}
{
%\coloredbox{
We extend Schiffler - Thomas' $T$-path definition and formula for unpunctured surfaces (2009). 
Let $T^o$ be an ideal triangulation (of an unpunctured surface or a once-punctured disk) and an arc $\ga$ that crosses $T^o$.
%We expect to generalize this to other punctured surfaces.
%}

%\begin{definition}[Complete $(T^o,\ga)$-path]\label{def:Tpath}
\innerblock{Definition: quasi-arc}{
If $\tau$ is an ordinary arc,
let an associated \emph{quasi-arc} $\tau'$ be a curve (not passing through the puncture $\puncture$) which agrees with $\tau$ outside of a small radius-$\epsilon$ disk $D_\epsilon$ around $\puncture$.
If $\tau$ is not adjacent to the puncture, let the associated quasi-arc be $\tau$ itself. 
%If $\tau$ is an ordinary radius of $\Cn$ between a marked point $v$ on the boundary and the puncture $\puncture$,
%let an associated \emph{quasi-arc} $\tau'$ be a curve (not passing through $\puncture$) which agrees with $\tau$ outside of a radius-$\epsilon$ disk $D_\epsilon$ around $\puncture$, where $\epsilon$ is chosen small enough so that the intersection of $\tau$ with any other arc is outside of $D_\epsilon$.
%Otherwise, we let the associated quasi-arc be $\tau$ itself.
}
\innerblock{Definition: $T$-path}{
%Let $T^o$ be an ideal triangulation (of an unpunctured surface or a once-punctured disk) and an arc $\ga$ that crosses $T^o$.
A \emph{complete $(T^o,\gamma)$-path} (or \emph{$T^o$-path} for short) $\om = (\om_1, \dots, \om_{2d+1})$
is a concatenation of quasi-arcs and boundary edges such that:
\begin{itemize}
\item [(T1)]
Each even step $\om_{2k}$ ($k=1,\ldots,d$) is the $k$-th arc that $\gamma$ crosses.
\item[(T2)] 
The path $\om$ is homotopic to $\ga$, and satisfies the following:\\
Let $p_1,\dots,p_d$ be the intersection points of $\gamma$ and $T^o$.
Let $\gamma_k$ be the segment along $\gamma$ between $p_k$ and $p_{k+1}$.
Then the segment $\gamma_k$ is homotopic to the segment from $p_k$ following $\om_{2k}$,
following $\om_{2k+1}$, following $\om_{2k+2}$ until $p_{k+1}$.
%
%For $k=0,\ldots,d$, each $\om_{2k+1}$ traverses a side of the ideal triangle $\blacktri{k}$.
%In addition, $\om_1$ traverses the edge $\tg{0}$ or $\tg{-1}$ (which is adjacent to $s$)
%and $\om_{2d+1}$ traverses the edge $\tg{d}$ or $\tg{d+1}$ (which is adjacent to $t$).
%\begin{enumerate}[i.)]
%\item
%Moreover, for $k=1,2, \ldots, d-1$, let $\pkpkone{\om}$ 
%denote the segment of $\om$ starting at the point $p_k$
%following $\om_{2k}$, continuing along $\om_{2k+1}$ and $\om_{2(k+1)}$ until the point $p_{k+1}$. 
%Then the segment $\ga_k$ is homotopic to $\pkpkone{\om}$.
%If $S=\Cn$, then we mean homotopy in the disk minus the puncture.
%
%\item
%The segment $\ga_0$ is homotopic to the segment $[s,p_1]_{\om}$
%of the path starting at
%the point $s=p_0$ following $\om_1$ and $\om_{2}$ until the point $p_1$;
%
%\item
%The segment $\ga_d$ is homotopic to the segment $[p_d, t]_{\om}$
%of the path starting at
%the point $p_d$ following $\om_{2d}$ and $\om_{2d+1}$ until the point $p_{d+1}=t$.
%\end{enumerate}
\item[(T3)] 
The step $\om_{2k+1}$ starts and finishes in the interior of $\blacktri{k}$ or at a boundary marked point. 
%This means that, if $\om_{2k+1}$ goes along a quasi-arc $\tau'$ associated to a radius, $\tau'$ must be chosen so that its endpoint $\puncture'$ near the puncture is located in the interior of $\blacktri{k}$.
\end{itemize}
}%\end{definition}
%}


%\block{$T$-path formula}{
%\innerblock{Definition: \emph{Laurent monomial} from a $T$-path $\alpha$}{
%the \emph{Laurent monomial} corresponding to a $T$-path $\alpha = (\alpha_1, \dots, \alpha_{2d+1})$

%}
%\end{definition}
\innerblock{Theorem ($T$-path formula)}{
The cluster variable $x_\gamma$ expressed 
in the variables corresponding to $T^o$ is 
%\coloredbox{
$$x_\gamma = \sum_\om x(\om)$$%}
over all $(T^o,\gamma)$-paths $\om = (\om_1, \dots, \om_{2d+1})$,
where 
$$x(\om) := \left( {\prod_{i\text{ odd}} x_{\om_i}} \right) 
\left({\prod_{i\text{ even}} x_{\om_i}}^{-1}\right).$$
%\end{theorem}
}}
%\note[targetoffsetx=4cm, targetoffsety=3cm, width=16cm]{We expect to generalize this to other punctured surfaces.}
\note[targetoffsetx=7cm, targetoffsety=-0.5cm, innersep=1cm,connection, width=4cm]{new}
\note[width=10cm,targetoffsetx=7cm, targetoffsety=-3cm, innersep=1cm,angle=-20,connection]
{We expect to generalize this to other punctured surfaces.}

\column{0.45}
\block{}{
\newcommand\TikzFourTpathsSize{0.41}
\newcommand\TikzNooseTpathsSize{0.47}

%\mbox{
%\subfigure[An arc $\ga$ of $\Cnn{4}$ which crosses $T^o$ $5$ times.]{
\begin{tikzfigure}
[Four of nine $(T^o,\ga)$-paths. %(of $T^o$ and $\ga$ from Figure \ref{fig:T2_triangulated_polygon_original}) which contain a (counterclockwise) non-backtrack cycle $(\rrNon)$.
All backtracks $(\underline{2,2})$ and $(\underline{\ell,\ell})$ have been omitted.]
\label{fig:T2_triangulated_polygon_original}
\TikzTtwoIdealTriangulationWithNumberLabels{1.8*\TikzFourTpathsSize}
\TikzTpathSix{\TikzFourTpathsSize}{ \TpathCompleteSixDollar}
\TikzTpathSeven{\TikzFourTpathsSize}{ \TpathCompleteSevenDollar}
\TikzTpathEight{\TikzFourTpathsSize}{ \TpathCompleteEightDollar}
\TikzTpathNine{\TikzFourTpathsSize}{ \TpathCompleteNineDollar}
\end{tikzfigure}

\begin{tikzfigure}
[The four $(T^o,\ga)$-paths of the ideal triangulation $T^o$ and the $\ell$-loop $\ga$ of the first figure.]
\label{fig:TikzNooseGammaIdealTriangulation}
%Figure \ref{fig:TikzNooseGammaIdealTriangulation}.
%Each path contains a quasi-backtrack $(\overline{1,1})$, $(\overline{2,2})$, or $(\overline{3,3})$]
%\mbox{
%\subfigure[An $\ell$-loop $\ga$ of $\Cnn{4}$ which crosses $T^o$ $3$ times.]{
\TikzNooseGammaIdealTriangulation{1.5*\TikzNooseTpathsSize}
%\label{fig:TikzNooseGammaIdealTriangulation}
\TikzNooseTpathOne{\TikzNooseTpathsSize}{$(b_1,1,\overline{2,2},\overline{3,3},\tau)$}
\TikzNooseTpathTwo{\TikzNooseTpathsSize}{$(\tau,1,b_2,2,\overline{3,3},\tau)$}
\TikzNooseTpathThree{\TikzNooseTpathsSize}{$(\tau,\overline{1,1},2,b_3,3,\tau)$}
\TikzNooseTpathFour{\TikzNooseTpathsSize}{$(\tau,\overline{1,1},\overline{2,2},3,b_4)$}
%}\label{fig:TikzNoose}
\end{tikzfigure}

\begin{tikzfigure}
[Three of the five $(T^o,\ga)$-paths of the ideal triangulation $T^o$ and the arc $\ga$ of the first figure]
\label{fig:TikzNotNooseGammaIdealTriangulation}
%\mbox{
%\subfigure[An arc $\ga$ of $\Cnn{4}$ which crosses $T^o$ $3$ times.]{
\TikzNotNooseGammaIdealTriangulation{1.5*\TikzNooseTpathsSize}
%\label{fig:TikzNotNooseGammaIdealTriangulation}}
%\subfigure[\tiny$(b_1,1,\overline{2,\,}\underline{\overline{2},2},3,b_4)$]{
\TikzNotNooseTpathOne{\TikzNooseTpathsSize}{$(b_1,1,\overline{2,\,}\underline{\overline{2},2},3,b_4)$}
%\subfigure[\tiny$(0,1,b_2,\overline{2,2},3,b_4)$]{
\TikzNotNooseTpathTwo{\TikzNooseTpathsSize}{$(0,1,b_2,\overline{2,2},3,b_4)$}
%\subfigure[\tiny$(0,\overline{1,1},2,\underline{3,3},b_3)$]
%{
\TikzNotNooseTpathThree{\TikzNooseTpathsSize}{$(0,\overline{1,1},2,\underline{3,3},b_3)$}
%\caption{
% \ref{fig:TikzNotNooseGammaIdealTriangulation}.
%}\label{fig:TikzNotNoose}
\renewcommand\TikzNooseTpathsSize{0.5}
\end{tikzfigure}

\begin{tikzfigure}[Examples of \emph{non}-$(T^o,\ga)$-paths for the situations in Figures \ref{fig:T2_triangulated_polygon_original}, \ref{fig:TikzNooseGammaIdealTriangulation}, and \ref{fig:TikzNotNooseGammaIdealTriangulation}.
Each path is homotopic to $\ga$ and satisfies (T1) but fails (T2) or (T3).
]
%\mbox{
%\subfigure[A path with the same labels {\tiny\TpathCompleteSixDollar} as Figure \ref{fig:path_six}
% but fails (T2) because ${[p_4,p_5]_{\om}}$ is not homotopic to $\ga_4$.]{
\TikzTpathSixBad{0.4}
%\label{fig:path_six_bad}
%}\quad
%\subfigure[A path with the same labels {\tiny$(b_1,1,\underline{2,2},\overline{3,3},\tau)$} as Figure \ref{fig:TikzNooseTpathOne} but fails (T3) because $\om_5=3$ starts outside $\blacktri{2}$.]{
\TikzNooseTpathOneBad{\TikzNooseTpathsSize}
%\label{fig:TikzNooseTpathOneBad}
%}\quad
%\subfigure[A path with the same labels {\tiny$(0,\overline{1,1},2,\underline{3,3},b_3)$} as Figure \ref{fig:TikzNotNooseTpathThree} but fails (T3) because $\om_{1}=0$ starts outside $\blacktri{0}$ and $\om_3=2$ finishes outside $\blacktri{1}$.]{
\TikzNotNooseTpathThreeBad{\TikzNooseTpathsSize}
\end{tikzfigure}
%\label{fig:TikzNotNooseTpathThreeBad}
%\caption{Examples of \emph{non}-$(T^o,\ga)$-paths for the situations in Figures \ref{fig:T2_triangulated_polygon_original}, \ref{fig:TikzNooseGammaIdealTriangulation}, and \ref{fig:TikzNotNooseGammaIdealTriangulation}.
%Each path is homotopic to $\ga$ and satisfies (T1) but fails (T2) or (T3).}
%\label{fig:non_tpaths}
%\end{figure}}
}
\note[targetoffsetx=-22cm, targetoffsety=-7cm, width=5.5cm]
{The $T$-paths are in natural bijection with Musiker, Schiffler, and Williams' snake graph matchings.}
%\note[targetoffsetx=7cm, targetoffsety=-7cm, width=5.5cm]{Steps are not drawn exactly along the arcs/boundary edges for \mbox{illustration} purposes.}
\end{columns}


\begin{columns}

\column{0.45}
%\block{Application of $T$-paths: atomic basis proof} {
%\block{What is the atomic basis?}
\block{Definition: Atomic Bases and Cluster Monomials}
%\innerblock{What is the atomic basis?}%[Proper Laurent monomial property of cluster monomials]
{
%\innerblock{Definition: atomic bases and cluster monomials}{
Let $\mathcal{A}$ be a (coefficient-free) cluster algebra.
\begin{itemize}
 \item The \emph{positive cone} of $\mathcal{A}$ is
 $\{$positive elements$\}$ = 
  $\{$elements that are positive Laurent polynomials %when expanded
 with respect to every cluster$\}$.
  
\item The subset $\mathcal{B}$ of all indecomposable positive elements (\textit{i.e.}, those that cannot be written as a sum of two positive elements)
is called the \emph{atomic basis} if it forms a $\mathbb{Z}$-basis of $\mathcal{A}$.

 
 \item A \emph{cluster monomial} is a product of cluster variables all coming from the same cluster,
%\item A \emph{cluster monomial} is a monomial in a cluster,
e.g. $a^{5}be^2$ is a cluster monomial if $\{a,b,c,d,e \}$ is a cluster.

%If $\mathcal{A}$ comes from a surface, a cluster monomial corresponds to a multi-tagged dissection D (i.e. a partial tagged triangulation allowing multiple copies of tagged arcs).
% We define xD = ??D x?.
 \end{itemize}
% }
 }
 \note[width=18.5cm, angle=-25,%connection, 
 targetoffsety=-0.5cm,targetoffsetx=2cm]{%If $\mathcal{A}$ comes from a surface, 
 A cluster monomial corresponds to a multi-tagged dissection D (\textit{i.e.} a partial tagged triangulation allowing multiple copies of tagged arcs).}
\column{0.55}

\block{Result 2: %combinatorial 
$T$-path proof for type $D$ cluster algebras}{

%Result 2: Combinatorial proof of atomic basis for type $D$% cluster algebras
%Application of $T$-path formula: a purely combinatorial proof of the proper Laurent monomial property of a cluster algebra of type $D$.

%We give a combinatorial proof (using the $T$-path formula) for the following.
%\end{result2}
\innerblock{Theorem (atomic basis)}{% (Cerulli Irelli, 2011)}{
For a cluster algebra of type $A$, $D$, or $E$, the basis of cluster monomials is atomic.}
%\end{theorem}
%\innerblock{remark}{
\begin{enumerate}[-]
%\item Cerulli Irelli (2011) gives a representation theory proof which works for all finite type cluster algebras.
\item %\cite[Cerulli Irelli, 2011]{Cer11} and 
Representation theory proof by {[Cerulli Irelli, 2011]} and
{[Cerulli Irelli, Keller, Labardini-Fragoso, and Plamondon, 2012]}
\item We give a combinatorial proof (using the $T$-path formula) for type $D$,
inspired by work on types $A$ and $\widetilde{A}$ by {[Dupont and Thomas, 2011]}.

%\item Our proof is inspired by {[Dupont and Thomas, 2011]} for type $A$ and $\widetilde{A}$ cluster algebras.
\end{enumerate}
%\end{remark}
%}
}

\end{columns}

%\begin{columns} \column{0.5} 

\block{Atomic bases for other surfaces (with $2$ or more marked points)}
{\begin{multicols*}{2}
\innerblock{Conjecture (Fomin, Shapiro, and Thurston, 2006)}{
A candidate for atomic bases:
the \emph{``bracelets collection"} consisting of all cluster monomials %plus
$+$ a class of elements.
A \emph{bracelet} is a 
closed loop in the interior of the surface which avoids marked points.
% is the atomic basis for a cluster algebra from a surface. %arising from a surface.
%(True for disks with $\leq 1$ puncture and annuli, types $A$, $D$, and $\widetilde{A}$).
% (type $\widetilde{A}$), and once-punctured disks (type $D$)).)
%(Only proven for disks (type $A$), annuli (type $\widetilde{A}$), and once-punctured disks (type $D$)).)
%\end{conjecture}
}

\begin{enumerate}[-]
\item True for annuli, type $\widetilde{A}$ (Dupont and Thomas, 2011).
\item The bracelets collection forms a basis for 
unpunctured surfaces (Musiker, Schiffler, and Williams, 2011).
\end{enumerate}
%}

\innerblock{Current work}{
Type $\tilD_{n-1}$ cluster algebras ($(n-3)$-gons with $2$ punctures), \textit{e.g.} type $\tilD_{6}$ cluster algebra comes from a square with $2$ punctures.

\TikzAffineDSixTaggedNextToQuiver{0.7}
\begin{tikzpicture}[scale=0.4]
\begin{scope}[->,>=stealth',shorten >=1pt,auto,node distance=1.2cm, ultra thick, red]
\node (2) {};
\node (1) [ right of=2] {};
\node (r) [ above of=2, left of=2] {};
\node (rp) [ below of =2, left of=2] {};
\node (3) [right of=1]{};
\node (t) [ above of=3, right of=3] {};
\node (tp) [ below of =3, right of=3] {};
  
  \path[every node/.style={font=\sffamily\small}]
    (2) edge (1)
    (2) edge (rp)
    (2) edge (r)
    (3) edge (1)
    (3) edge (tp)
    (3) edge (t);
\draw (1) node{$1$};
\draw (2) node{$2$};
\draw (r) node{$r$};
\draw (rp) node{$r^{\bowtie}$};
\draw (3) node{$3$};
\draw (t) node{$t$};
\draw (tp) node{$t^{\bowtie}$};
\end{scope}
\end{tikzpicture}
}
\end{multicols*}
}
%\item A version of the $T$-path formula for tagged triangulations.

%% Text and figure Inside the Block
%\block{Text and figure Block}{
%\begin{minipage}[t]{0.7\linewidth}
%%\lipsum[1]
%\itemize{\item one \item two}
%\end{minipage}%
%\begin{adjustbox}{valign=t}
%\begin{minipage}[t]{0.3\linewidth}
%\TikzToneOnly{0.6}
%\end{minipage}
%\end{adjustbox}
%}
%\column{0.5}
%\block{Acknowledgements.~} {
\note[targetoffsetx=22cm,targetoffsety=-2.5cm, width=22cm,innersep=.5cm]{
%We would like to thank Pasha Pylyavskyy, Vic Reiner, and Peter Webb for looking over an early version of this paper, Hugh Thomas for helpful discussions, and the referees for many useful comments. 
\begin{enumerate}[-]
\item The authors were supported by NSF Grants DMS-1067183 and DMS-1148634,
and would like to thank P. Pylyavskyy, V. Reiner, H. Thomas, and P. Webb.
\item Preprint arXiv:1409.3610 (2014) to appear in SIGMA.
\item Email: egunawan@umn.edu 
\item Home page: umn.edu/home/egunawan
\end{enumerate}}


%\block{References}{
%%\bibliographystyle{authordate1}
%    \printbibliography[heading=none]    
%   
%}
\end{document}